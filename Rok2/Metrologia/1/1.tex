\documentclass[11pt]{article}
\usepackage[utf8]{inputenc}
%\usepackage[T1]{fontenc}
\usepackage{amssymb}
\usepackage{amsmath}
\usepackage{enumerate}
\usepackage{fullpage}
\usepackage{polski}  
\usepackage{indentfirst} 
\usepackage[pdftex]{graphicx}
\usepackage{multirow}
\usepackage{placeins}

\author{Łukasz Dubiel}
\title{Metrologia-wstęp}

\begin{document}

\maketitle

\section{Literatura}
Zatorski, Sroka  . Podstawy metrologi elektrycznej.

Zatorski, Rozkrut , Miernictow elektryczne. Materiały do ćwiczeń.

Macryniuk. Postawy metrologi elektycznej.

Stabrowski. Cyfrowe przyrzady pomiarowe.

Rydzewski. Pomiary Oscyloskopowe

Kulka Liubura, Nadachowski. Przetworniki analogowo-cyfrowa i cyfrowo analogowe.

Tumański : Technika pomiarowa.

Skubis T. Opracowanie wyników pomiarów - przykładów.

Pomiary dynamiczne - hagel

\section{Przyrzady magneto-elektryczne}

Symbol półokrąg i

Większość multimetrów analogowych tworzonych jest w ten sposób. Czasami jest jeszcze dioda.

\section{Klasa przyrządu}
Dana z szerogu 0.1, 0.2 , 0.5, 1.0 , 1.5

Klasa, błąd względy procentowy.

$$ k  = \frac{\Delta_{max}}{z} \cdot 100\% $$
gdzie k to klasa, $ \Delta_{max}$ to maksymalna odchyłka, a $z$ to zakres 

\end{document}
