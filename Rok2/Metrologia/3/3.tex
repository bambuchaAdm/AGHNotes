\documentclass[11pt]{article}
\usepackage[utf8]{inputenc}
%\usepackage[T1]{fontenc}
\usepackage{amssymb}
\usepackage{amsmath}
\usepackage{enumerate}
\usepackage{fullpage}
\usepackage{polski}  
\usepackage{indentfirst} 
\usepackage[pdftex]{graphicx}
\usepackage{multirow}
\usepackage{placeins}

\author{Łukasz Dubiel}
\title{Metrologia\\ wykład 3}

\begin{document}

\maketitle

\section{Błędy metody}

\section{Błąd graniczny}
Można wyznaczyć z kasy przyrządu, lub podanego przez producenta przyrządu.

\section{Błąd graniczny}
Błąd wyznaczony na moduł. Określa gdzie przedział w jakim może być rzeczywista 

\section{Poprawka}
Gdy mamy błąd metody, musimy go znać 

\section{Właściwości obserwatora}
Na przyrządach cyfrowych ma mniejsze znaczenie. Na przyrządach analogowych istniała duża gama błędów np. błąd paralaksy czy coś innego.

\section{Przyczyny losowe}

\section{Sposoby zapisu wyników pomiarowych}
Sposób wyznaczania wyniku 
$$ \hbox{Wynik} = \hbox{Przybliżenie} \pm \hbox{Bład  [Jednostka]} $$

\subsection{Rozdzielczość}
Maksymalna ilość cyfr na przyrządzie. Jeśli wynik jest dokładny to musi być rozdzielczy. Jeśli jest rozdzielczy to nie musi być dokładny.

\subsection{Cyfry niepewne}
Cyfrę uważamy za niepewną gdy niepewność pomiaru jest większa niż połowa pojemności.

\section{Niepewność deterministyczna}
Właściwości przyrządów pomiarowych. Rozkład prawdopodobieństwa jest równomierne. 

\subsection{Niepewność złożona}
Niepewność złożona standardowa
$$ u_c =\sqrt{ u_a^2 + u_r^2 } $$
$$ k_b = \sqrt{3} p $$
$$ U_b = k_b \cdot u_b $$

Powinniśmy stosować miarę granicznego błędu względnego.

\subsection{Błędy}
Małe błędy występują częściej. Znaki występują jednakowo często.

\end{document}
