\documentclass[11pt]{article}
\usepackage[utf8]{inputenc}
%\usepackage[T1]{fontenc}
\usepackage{amssymb}
\usepackage{amsmath}
\usepackage{enumerate}
\usepackage{fullpage}
\usepackage{polski}  
\usepackage{indentfirst} 
\usepackage[pdftex]{graphicx}
\usepackage{multirow}
\usepackage{placeins}

\author{Łukasz Dubiel}
\title{Oscyloskop}

\begin{document}

\maketitle

\section{defnicja}
Z łac soliare ( kwiać się) oraz gr. skopei ( patrzeć ) 

\section{Co mierzymy oscyloskopem}
Badania zależności między wielkościami elektrycznymi lub wielkościami przekadanymi na wielkości napięciowe .

\section{Zastosowania}
\begin{enumerate}
\item{Obserwacji sygnałów}
\item{Pomiary napięć (prądów)}
\item{Okres sygnałów}
\item{Częstotliwość}
\item{Kąt przesunięcia fazowego}
\item{Obserwacji charakterystyk statystycznych elementów}
\end{enumerate}
Jest jednym z najbardziej uniwersalnych narzędzi pomiarowych (duża czułość napięciowa (pomiary napięć do mV), szerokie pasmo częstotliwości (do MHz, a nawet GHz), możliwość obserwacji wielu przebiegów jednocześnie, w oscyloskopach cyfrowych pomiar automatyczny wielu parametrów).


\end{document}
