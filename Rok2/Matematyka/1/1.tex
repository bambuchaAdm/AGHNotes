\documentclass[11pt]{article}
\usepackage[utf8]{inputenc}
%\usepackage[T1]{fontenc}
\usepackage{amssymb}
\usepackage{amsmath}
\usepackage{enumerate}
\usepackage{fullpage}
\usepackage{polski}  
\usepackage{indentfirst} 
\usepackage[pdftex]{graphicx}
\usepackage{multirow}
\usepackage{placeins}

\author{Łukasz Dubiel}
\title{Macierz odwzorowania liniowego}

\begin{document}

\maketitle

$\mathcal{V,W}$ - przestrzenie wektorowe $ \mathcal{V} = \mathbb{R}^n$ $\mathcal{W} = \mathbb{R}^m$ 

$ f : \mathcal{V} \to \mathbb{W} $

$ E = ( e_1 , \ldots , e_n ) $ - baza w przestrzeni $\mathcal{V}$

$B = (b_1, \ldots , b_m) $ - baza przestrzeni w $\mathcal{W}$

$$ \mathcal{W} \ni f(e_1) = a_{11} b_1 + \ldots +  a_{m1} b_m  = [a_{11},\ldots, a_{m1} ] $$
$$ \vdots $$
$$ \mathcal{W} \ni f(e_n) = a_{1n} b_1 + \ldots +  a_{mn} b_m  = [a_{1n},\ldots, a_{mn} ] $$

$$ A = M_f = M_f(E,B) = \begin{bmatrix}
	a_{11} & \ldots & a_{1n}\\
	\vdots &  & \vdots\\
	a_{m1} & \ldots & a_{mn}\\
\end{bmatrix}_{m \times n} \quad n = \dim{\mathcal{V}} \quad m = \dim{\mathcal{W}}$$

\section{Twierdzenie o postaci odwzorowania liniowego}

$A = M_f(E,B) \ $
$ x \in V \ $ 

$ x = [x_1,x_2,\ldots,x_n]_E $

$ \mathcal{W} \ni y = f(x) = [y_1,y_2,\ldots,y_m]_B $

$$ Y = A \cdot X $$

\section{Twierdznie o rzędzie macierzy odwzorowania liniowego}

$$ r(M_f) = r(f) $$ gdzie $r(f) = \dim{\Im{f}}$

\subsection{Wniosek}
Rząd macierzy $M_f$ nie zależy od wciętych baz

\section{Twierdzenia o macierzach działań na odwzorowań liniowych}

$f,g : \mathcal{V} \to \mathcal{W} $ gdzie, $E$-baza przestrzeni $\mathcal{V}$, $B$ - baza przestrzeni $\mathcal{W}$ , $\alpha \in \mathbb{K}$ ( $\mathbb{K} = \mathbb{R} \vee \mathbb{K} = \mathbb{C}$)

\subsection{Suma odwzorowań}
$$ M_{f+g} = M_f + M_g$$
\subsection{Iloczyn odwzorowań z skalarem}
$$ M_{\alpha f} = \alpha M_f $$
\subsection{Złożenie odwzrowań}
$ f : \mathcal{V} \to \mathcal{W} $

$ g : \mathcal{W} \to \mathcal{U} $

$$ M_{g \circ f} = M_g \cdot M_f $$

\subsection{Odwzrowanie odwrotne}
$ f : \mathcal{V}_{B} \to \mathcal{V}_{B'} $

$ \exists f^{-1} : \mathcal{V}_{B'} \to \mathcal{V}_{B} $
$$ M_{f^{-1}} = (M_f)^{-1} $$

\section{Odwzorowanie identycznościowe}
\subsection{Definicja} 
Odwzrorowanie identycznościowe oznaczane $$ Id : \mathcal{V} \to \mathcal{V} $$ takie że 
$$ \forall v \in \mathcal{V} \quad Id(v) = v $$
\subsection{Własności}
\begin{enumerate}
\item{Liniowość}
\item{Bijektywność}
\end{enumerate}

\section{Macierz przejścia}

$B = (b_1,\ldots,b_n)$ - stara baza 

$B' = (b'_1 , \ldots ,b'_n ) $ -nowa baza

\section{Definicja}
Macierzą przejścia $P$ od bazy $B$ do bazy $B'$ nazywamy macierz odwzorowania, identycznościowego przestrzeni $ \mathcal{V}_{B'}$ w przestrzeń $\mathcal{V}_{B}$.

$$ P = M_{Id}(B',B) $$

$$ Id(b'_1) = b'_1 = [p_{11},\ldots,p_{n1}]_B $$
$$ \vdots $$
$$ Id(b_n) = b_n = [p_{n1}, \ldots, p_{nn}]_B $$

$$ P = \begin{bmatrix}
	p_{11} & \ldots & p_{1n}\\
	\vdots &  & \vdots \\
	p_{n1} & \ldots & p_{nn}\\
\end{bmatrix}$$

\subsection{Uwaga 1}

$$ Id \hbox{ - bijekcja} \Rightarrow  \dim{ \Im{Id}}  = n \implies \det{P} \not = 0 $$

\subsection{Wniosek 1}
 $$ P \hbox{ - nieosobliwa} $$
 $$ P^{-1} - \hbox{ macierz odwzorowania Id i bazach } B \hbox{ i } B' \hbox{ czyli macierz przejścia z  bazy } B \hbox{ do } B' $$
 
\subsection{Wniosek 2}
$B$ - stara baza przestrzeni $\mathcal{V}$ 

$B'$ - nowa baza w $\mathcal{V}$

$ x \in \mathcal{V}$

$x = [x_1,\ldots,x_n]_B$

$x = [x'_1, \ldots, x'_n]_{B'}$

$$ X = \begin{bmatrix}
	x_1\\
	\vdots\\
	x_3\\
\end{bmatrix} \quad  
X' = \begin{bmatrix}
	x'_1\\
	\vdots\\
	x'_3\\
\end{bmatrix} $$
$$ X = P \cdot X' $$
$$ X' = P^{-1} \cdot X $$ 

\subsection{Uwaga}
$Id : \mathcal{V}_B\to \mathcal{V}_B$

$$ P = \begin{bmatrix}
	1 & \ldots & 0\\
	\vdots & 1 & \vdots\\
	0 & \ldots & 1\\
\end{bmatrix} = \mathbf{I}$$

\section{Twierdzenie o zamianie baz}
$ f : \mathcal{V} \to \mathcal{W} $

$ E,E' $ - baza w przestrzeni $\mathcal{V}$ 

$ B,B' $ - baza w przestrzeni $\mathcal{W}$

$$ M_f(E',B') = Q^{-1} \cdot M_f(E,B) \cdot P $$

$P$ - macierz przejścia od bazy $E$ do $E'$

$Q$ - macierz przejścia od bazy $B$ do $B'$

\subsection{Dowód}
$ x \in \mathcal{V} $
$$ X = P \cdot X' $$
$$ Y = P \cdot Y' $$
$$ Y = M_f(E,B) \cdot X $$

$$ Q \cdot Y' = M_f(E,B) \cdot P \cdot X' \quad / Q^{-1} \cdot $$
$$ Q^{-1} \cdot Q \cdot Y' = Q^{-1} \cdot M_{E',B'} \cdot P \cdot X'$$
$$ Y'  =  \underbrace{Q^{-1} \cdot M_f(E,B) \cdot P}_{M_f(B',E')} \cdot X'$$

\subsection{Przykład}
$ f : \mathbb{R}^3 \to \mathbb{R}^2 $

$ E = ((1,0,0),(1,1,0),(1,1,1)) $

$ B = ((1,0),(1,1)) $

$$ M_f(E,B) = \begin{bmatrix}
	1 & -1 & 2\\
	0 & -3 & -5\\
\end{bmatrix}$$

$ E' = ( (0,0,1) , (0,1,0), (1,0,0) ) $

$ B' = ( (0,-1),(-2,-1)) $

$$ P = \begin{bmatrix}
	0 & -1 & 1\\
	-1 & 1 & 0\\
	1 & 0 & 0\\
\end{bmatrix}$$

$$ Q^{-1} = 
\begin{bmatrix}
\frac{1}{2} &  \frac{-1}{2} \\
\frac{-1}{2} & \frac{-1}{2} \\
\end{bmatrix} $$

$$ M_f(E',B') =
\begin{bmatrix}
\frac{1}{2} &  \frac{-1}{2} \\
\frac{-1}{2} & \frac{-1}{2} \\
\end{bmatrix}
\cdot
\begin{bmatrix}
	1 & -1 & 2\\
	0 & -3 & -5\\
\end{bmatrix}
\cdot
\begin{bmatrix}
	0 & -1 & 1\\
	-1 & 1 & 0\\
	1 & 0 & 0\\
\end{bmatrix} $$

\subsection{Wniosek}
$f$ jest endomorfizmem
$B,B'$ - bazy przestrzeni V

$$M_f(B',B') = P^{-1} \cdot M_f(B,B) \cdot P $$



\end{document}
