\documentclass[11pt]{article}
\usepackage[utf8]{inputenc}
%\usepackage[T1]{fontenc}
\usepackage{amssymb}
\usepackage{amsmath}
\usepackage{enumerate}
\usepackage{fullpage}
\usepackage{polski}  
\usepackage{indentfirst} 
\usepackage[pdftex]{graphicx}
\usepackage{multirow}
\usepackage{placeins}

\author{Łukasz Dubiel}
\title{Diagonalizacja endomorfizmów}

\begin{document}

\maketitle

\section{Wartość i wektory własne endomorfizmu}

$\mathcal{V}$ - przestrzeń wektorowa nad ciałem $\mathbb{K}$
$f$ - endomorfizm przestrzeni $V \quad (f : \mathcal{V} \to \mathcal{V}$ - liniowe

$ \lambda \in \mathbb{K} $

Skalar $\lambda$ nazywamy wartości własnej endmofizmu $f$ gdy $$ \exists v \in \mathcal{V}, v \not = \overline{0} \quad f(v) = \lambda v $$

Następnie wektor $v$ spełniający warunek $$ f(v) = \lambda v$$ nazywamy wektorem własnym endomorfizmu odpowiadającym wartości własnej $\lambda$. 

\subsection{Przykład}
$ V = \mathbb{R}^3 $ \\
$f$ - symetria względem płaszczyny $OXY$ 
$$ f(v) = -v \implies \lambda =  -1 \hbox{ wartość własna} $$
$$ f(v') = v' \Longrightarrow \lambda =  1 \hbox{ wartość własna } $$

\subsection{Uwaga}
$$ v = \overline{0}  \implies f(v) = \overline{} = \lambda \overline{0} = \lambda v $$
Co oznacza że własnościami własnymi jest cały zbiór liczb rzeczywistych, jednak nie o to chodzi. 

\section{Widmo endomorfizmu}
$ \lambda - wartość własna endomorfizmu $
$$ V_{\lambda} := \left\{ v \in V : f(v) = \lambda v \right\} $$
Jest to zbiór wszystkich wektorów własnych odddział wartośći własnej $\lambda$ wraz z wektorem zerowym.

\subsection{Twierdzenie} 
$$ V_\lambda \hbox{ - podprzestrzeń wektorowa przestrzeni } V $$
\subsubsection{Dowód}
$ \\
\forall \alpha, \beta \in \mathbb{K} \\
\forall v_1,v_2 \in V_\lambda$
$$ f( \alpha v_1 + \beta v_2 ) = \alpha f(v_1) + \beta f(v_2) = \alpha \lambda v_1 + \beta \lambda v_2 = \lambda (\alpha v_1 + \beta v_2)$$ 
\subsection{Uwaga}
$$V_\lambda - \hbox{ podrzestrzeń własna (odpowiadającej wartości własnej) endomorfizmu f }$$

\subsection{Własności}
\begin{enumerate}
\item{$\dim{V_\lambda} \geq 1$}
\item{Każdy wektor własny odpowiada dokładnie jednej wartości własnej
$$ V_{\lambda_1} \cap V_{\lambda_2} = \overline{0} \quad \lambda_1 \not = \lambda_2 $$}
\item{$V_\lambda = \ker{f - \lambda Id_V}$}
\end{enumerate}
\subsubsection{Dowód 1}
$$ \exists v\not = 0 : \quad f(v) = \lambda v \implies v \in V_\lambda$$

\subsubsection{Dowód 2}
Dowód nie wprost 
Hp. $v$ - wektor własny odpowiadający wartości własnym $\lambda_1 i \lambda_2$, gdzie $\lambda_1 \not = \lambda_2$
$$
\begin{cases}
f(v) = \lambda_1 v \\
f(v) = \lambda_2 v \\
\end{cases}
\implies
\lambda_1 v = \lambda_2 v
\implies
(\lambda_1 - \lambda_2) v = \overline{0}
\implies
\lambda_1 - \lambda_2 = 0
$$

\subsubsection{Dowód 3}
$$ V_\lambda = \{ v \in V : f(v) = \lambda v \} = \{ v \in V : f(v) - \lambda v = \overline{0} \} = \{ v \in V : (f - \lambda Id)(v) = \overline{0} \} = \ker{(f - \lambda Id)} $$

\newpage
\section{Twierdzenie o wartościach własnych endomorfizmu}
$f$ - endomorfizm przestrzeni $V$ 
$B$ - baza przestrzeni $V$
$\mathbf{A} = M_f(B,B)$
$\lambda \in \mathbb{K} $

$$ \lambda - \hbox{ wartość własna endomorfizmu } f \iff \det(\mathbf{A} - \lambda \mathbf{I}) = 0 $$

\subsection{Dowód}
$$ \lambda - \hbox{ wartość własna } f 
\iff \exists v \not \overline{0} : f(v) = \lambda v
\iff \exists v \not = \overline{0} : f(v) - \lambda v = \overline{0}
\iff \exists v \not = \overline{0} : f( - \lambda Id)(v) = \overline{0} \iff $$
Co przechodząc na macierze odwzorowań daje 
$$ (\mathbf{A} - \lambda \mathbf{I}) v = \overline{0}
\iff r(\mathbf{A} - \lambda \mathbf{I}) < n
\iff \det(\mathbf{A} - \lambda \mathbf{I}) = 0 $$

\subsection{Wnioski}
$v$ - wektor własny endomorfizmu $f \iff v $ jest niezerowym rozwiązaniem układu jednorodnego $(\mathbf(A) - \lambda \mathbf{I}) v = \overline{0}$ odpowiedniej wartości własnej $\lambda$

\subsection{ Twierdzenie o niezależności wartośći własnej od wyboru baz }
Wartości własne nie zależą od wybranych baz odwzorowania

\section{Wielomian charakterystyczny}
Wielomianem charakterystycznym endomorfizmu $f$ nazywamy wielomian 
$$ \Delta (\lambda) := \det{(\mathbf{A} - \lambda \mathbf{I})} $$
gdzie  $\mathbf{A}$ - jest macierzą endomorfizmu $f$ w dowolnej bazie

\section{Równanie charakterystycznym}
Równaniem charakterystycznym endomorfizmu $f$ nazywamy równanie
$$ \Delta (\lambda) = 0 $$

\newpage
\section{Twierdzenie o liniowej niezależności wektorów własnych}

Wektory własne endomorfizmu $f$ odpowiadające różnym wartościom własnym są liniowo niezależne.

\subsection{Dowód}
ć.w.

\subsection{Wniosek}
$f$ - endomorfizm przestrzeni $V, \quad \dim{V} = n $

$f$ posiada $n$ różnych (parami) wartości własnych 

z tego wynika $$ \exists n \hbox{ liniowo niezależnych wektorów własnych przestrzeni V } \implies \exists \hbox{Istnieje baza złożona z wektorów bazowych} $$

\section{Endomorfizm diagonalizowalny}
Endomorfizm $f$ przestrzeni $V$ nazywamy diagonalizowalnym $:\iff \exists$ w przesrzenie $V$ taka baza, że macierz odwzorowania $f$ w tej bazie jest macierzą diagonalną. 

\section{WKW diagonalizowalnośći macierzy endomorfzimu}
$f$  - endomorfizm przestrzeni $V$
$$ f \hbox{ jest diagonalizowalny} \iff \exists \hbox { baza przestrzeni } V \hbox{ stworzona z wektorów własnych} $$

\subsection{Dowód $\Rightarrow$}
$f$ - diagonalizowalny
$$ \exists B = (b_1,\ldots,b_n) : \quad M_f(B,B) - \hbox{ Jest diagonalna } $$

$f(b_1) = [\alpha_1 , 0 , \ldots, 0]  = \alpha_1 b_1 \Rightarrow \alpha_1 - \hbox{ wartość własna endomorfizmu } f i b \hbox{ wektor własny odpowiadający } \alpha_1$

$ \vdots $

$f(b_n) = [0,0,\ldots,\alpha_n] = \alpha_n b_n$ 

\subsection{Dowód $\Leftarrow$}
ćw.

\subsection{Wnioski 1}
$f$ - endomorfizm diagonalizowalny 

$B = (v_1,\ldots,v_n)$ baza przestrzeni $V$ 
$ v_1,\ldots,v_n$ - wektory własne odpowiadające wartościom własnym $\lambda_1,\ldots,\lambda_n$ 

To 
$$ M_f(B,B) = \begin{bmatrix}
	\lambda_1 & 0  & 0\\
	0 & \ddots & 0 \\
	0 & 0 & \lambda_n\\
\end{bmatrix}$$

\subsection{Wniosek 2}
Endomorfizm $f$ posiada $n$ różnych wartości własnych oraz $\dim{V} = n \implies f$ jest diagonalizowalna.

\section{Twierdzenie o diagonalizowalności} 
$f$ - endomorfizm $V$

$\dim{V} = n $

$ \lambda_1,\ldots,\lambda_p $ - różne wartości własne endomorfizmu $f, \quad p \leq n$

$k_1,\ldots,k_p$- krotności $\lambda_1,\ldots,\lambda_p$ jako pierwiastków wielomianu charakterystycznego 
\end{document}
Wtedy
\begin{enumerate}
\item{
$$ \forall i=1, \ldots , p \quad \dim{V_{\lambda_i}} = k_i $$
}
\item{
$$f - diagonalizowalny \iff \forall i = 1,\ldots , p  \dim{V_{\lambda_i}} = k_i $$

\section{Przykłąd}
Sprawidzić że endomorfizm jest diagnoalizowalny, wyznaczyć macierz endomorfizmu $\mathbf{D}$ w bazie $\mathbf{B}$ w której macież $\mathbf{D}$ jest diagonalna oraz podać bazę $\mathbf{B}$. 

$f : \mathbb{R}^3 \to \mathbb{R}^3 $

$f(x,y,z) = (x-z,x+2y+z,z,y) $