\documentclass[11pt]{article}
\usepackage[utf8]{inputenc}
%\usepackage[T1]{fontenc}
\usepackage{amssymb}
\usepackage{amsmath}
\usepackage{enumerate}
\usepackage{fullpage}
\usepackage{polski}  
\usepackage{indentfirst} 
\usepackage[pdftex]{graphicx}
\usepackage{multirow}
\usepackage{placeins}

\author{Łukasz Dubiel}
\title{Matma semestr drugi powtorka}

\begin{document}

\maketitle

\section{Wartość włsana}
Wartością własną nazywamy każdą liczbę $\lambda \in \mathbb{R}$ spełniającą równanie
$$ \det(\mathrm{A} - \lambda \mathrm{I}) = 0 $$

\section{Wektor własny}
Wektor $ v \in \mathbb{R}^n \backslash \left\{\overline{0}\right\}$ nazywamy wektorem związanym z wartością własną $\lambda$, gdy spełnia równanie.
$$ \mathrm{A} v = \lambda v $$

\section{WKW diagonalizowalności}
Endomorfizm $f : \mathbb{R^n} \to R^{m}$ jest diagonalizowalny, wtw gdy $$\det {(\mathrm{A}_f -\lambda \mathrm{I}) = \sum_{i=1}(\lambda - \lambda_i)^{k_i}}$$
$$ \dim{(\mathrm{A}_f - \lambda_i \mathrm{I})} = k_i $$ 

\section{WW na diagonalizowalność}
Endomorfizm jest diagonalizowalny, gdy posiada $n$ różnych wartości własnych.

\section{Iloczyn Couchy'ego szeregów}
Jeśli $\sum{a_n}$ i $\sum{b_n}$ jest szeregami bezwzględnie zbieżnymi to szereg $\sum{c_n}$ gdzie 
$$ c_n = \sum_{k=0}^{n}{a_k \cdot b_{n-k}}$$
też jest zbieżny a jego granica 
$$ \sum{c_n} = \left( \sum{a_n} \right)\left( \sum{b_n} \right) $$

\section{Kryterium zbieżności}
\subsection{Porównawcze}
Niech $n_0 \in \mathbb{N}$ takie, że $ 0 \leq a_n \leq b_n \forall n \in \mathbb{N}$
$$ \sum{b_n} - \hbox{ zbieżny } \implies \sum{a_n} - \hbox{ zbieżny}$$
$$ \sum{a_n} - \hbox{ rozbieżny } \implies \sum{b_n} - \hbox{ rozbiezny }$$

\subsection{Ilorazowe}
$$ \lim_{n \to \infty}{\frac{a_n}{b_n}} \in (0,\infty) \implies \left( \sum{a_n} - \hbox{ zbiezny } \iff \sum{b_n} - \hbox{ zbiezny } \right) $$

\subsection{Kryterium d'Alamberta}
\begin{enumerate}
\item{
$$ \lim_{n \to \infty}{\left|\frac{a_{n+1}}{a_n}\right| } < 1 \implies \sum{a_n} < \infty $$}
\item{
$$ \lim_{n \to \infty}{\left|\frac{a_{n+1}}{a_n}\right| } > 1 \implies \sum{a_n} = \infty $$} 
\end{enumerate}
Dla ilorazu równego $1$ kryterium nie rozstrzyga

\subsection{Kryterium Couchy'ego}
\begin{enumerate}
\item{
$$ \lim_{n \to \infty}{\sqrt[n]{a_n}} < 1 \implies \sum{a_n} < \infty $$}
\item{
$$ \lim_{n \to \infty}{\sqrt[n]{a_n}} > 1\implies \sum{a_n} = \infty $$} 
\end{enumerate}
Dla wartości granicy równej $1$ kryterium nie rozstrzyga

\subsection{Kryterium całkowe}
$\sum_{n = n_0}{a_n}$ - szereg liczbowy, $ a_n \geq 0 $
$f : <n,\infty) \to \mathbb{R}, f \in \mathcal{C}(<n,\infty)) f - $ malejąca $ f(n) = a_n $

$$ \sum_{n = n_0}{a_n} - \hbox{ zbiezny } \iff \int_{n_0}^{\infty} f(x)dx - \hbox{ zbieżna} $$

\subsection{Kryterium Laibnitza}
Jeśli $a_n >0 \forall n \in \mathbb{N}, (a_n)_{n \in \mathbb{N}^+}$ - ciąg malejący oraz
$\lim_{n \to \infty}{a_n} = 0$ 
$$ \sum{a_n} \hbox{ jest zbieżny }$$

\section{Ciąg funkcyjny}
Zbiór funkcji
$$ F = \left\{ f : X \to \mathbb{R}, X \subset \mathbb{R} \right\} $$
$$(f_n)_{n \in \mathbb{N}} - \hbox{ ciąg funkcyjny } \iff f_n \in F \quad \forall n \in \mathbb{N} $$

\section{Zbieżność punktowa}
Mówimy, że ciąg $(f_n)$ funkcyjny jest zbieży punktowo do funkcji $f: X \to \mathbb{R}$ na zbiorze $X \subset \mathbb{R}$, wtw gdy $$\forall x \in X \quad \lim_{ n \to \infty}{|f_n(x) - f(x)| = 0}$$
innymi słowy
$$ \lim_{n \to \infty}{f_n(x)} = f(x) $$
\subsection{Duga wersja}
$$ f_n \xrightarrow{x \to \infty} f \iff \forall \epsilon > 0 \ \forall x \in X \ \exists n_0 \in \mathbb{N}\ \forall n \geq n_0 \quad |f_n(x) - f(x)| < \epsilon$$

\section{Zbieżność jednostajna}
$$f_n \mathop{\rightrightarrows}_{n \to \infty} f \iff \forall \epsilon > 0\ \exists n \in \mathbb{N}\ \forall x \in X\ \forall n > n_0\quad |f_n(x) - f(x)| < \epsilon$$
\subsection{definicja dla ludzi}
$$f_n \mathop{\rightrightarrows}_{n \to \infty} f \iff \lim_{n\to \infty}{\sup_{x \in X}{f_n(x)}} = 0$$

\section{Szereg funkcyjny}
Niech $(f_n)\ n \in \mathbb{R}$ - ciąg funkcyjny, $f_u X \to \mathbb{R}, X \subset \mathbb{R}$

Szeregiem funkcyjnym $\sum{f_n}$ nazywamy ciąg sum częściowych ciągu funkcyjneg $(f_n)_{n \in \mathbb{N}}$ 
$$ s_0 = f_0 \quad s_1 = f_0 + f_1 \quad s_2 = f_0 + f_1 + f_3 \quad s_n = f_0 + f_1 + \ldots + f_n $$

Suma szeregu $\sum{f_n}$ to granica ciągu sum częściowych $(s_n)_{n \in \mathbb{N}}$

\section{Zbieżność szeregów funkcyjnych}
\subsection{Punktowa}
$\sum{f_n}$ 0 zbieżny punktowo do $S \iff$ ciąg
$(s_n)_{n \in \mathbb{N}}$ jest zbieżny punktowo do $S$ na $X$. 
$$ \sum_{f_n} = s \iff s_n \mathop{\longrightarrow}_{n \to \infty}^{x} s $$

\subsection{Jednostajna}
$\sum{f_n}$ jest zbieżny jednostajnie do $S$ na $X \iff
s_n \mathop{\rightrightarrows}_{n \to \infty}^{x} s$

\subsection{Bezwzględna}
$\sum{f_n}$ jest zbieżny bezwzględnie do $S$ na $X \iff$ $ \sum{|f_n|}$ jest zbieżny punktowo do $S$ 
\section{Szereg potęgowy}
Szereg funkcyjny w postaci $$ \sum_{n = 0}{a_n (x - x_0)^n} $$
gdzie $a_n \in \mathbb{R} \quad \forall n \in \mathbb{N}$
oraz $ x,x_0 \in \mathbb{R}$ oraz $x_0$ jest jego środkiem.

\section{Warunek konieczny szeregów funkcyjnych}
\begin{enumerate}
\item{
szereg zbieżny punktowo do $f(x)$ na $X$ $\implies f_n(x) \to 0 $}
\item{Jak wyżej tylko bezwzględnie}
\item{Jak wyżej tylko że w obu stronach jednostajnie}
\end{enumerate}

\section{Kryterium Weistrassa}
Niech $X \subset \mathbb{R} f_n : X \to \mathbb{R} \forall n \in \mathbb{N}$
Jeśli "da się ograniczyć"
Jeśli $\forall n \in \mathbb{N} \forall x \in X |f_n(x)| < a_n$ oraz $ \sum{a_n} < \infty $ to szereg funkcyjny jest zbieżny bezwzględnie i jednostajnie na $X$.
\section{Promien zbiezności szeregu}
Niech $ \sum_{n = 0}^{\infty}{a_n (x - x_0)^n}$ - szereg potęgowy. Promieniem zbieżności szeregu nazywamy 
$$ R := \sup{ \left\{r : \sum_{n = 0}^{\infty}{a_n (x - x_0)^n} \hbox{ jest zbieżny punktowo w } (x_0 - r, x_0 +r)\right\}}$$

\section{Lemat Abela}
Jeśli szereg potęgowy jest zbieżny w punkcie $ \alpha \in \mathbb{R} \backslash \{ x_0 \}$, to szereg jest zbieżny bezwzględnie w przedziale $(x_0 - r , x_0 + r)$ oraz zbieżny jednostajnie w każdym przedziale domkniętym zawartym w $(x_0 - r, x_0 + r)$ gdzie $r = |\alpha - x_0|$ 

\section{Twierdzenie d'Alamberta}
Niech $ \sum_{n = 0}^{\infty}{a_n (x - x_0)^n}$ - szereg potęgowy oraz $ a_n \not = 0 \quad \forall n \in \mathbb{N}$
Jeśli $$ \lambda = \left| \frac{a_{n+1}}{a_n} \right| \implies R = \frac{1}{\lambda} $$
Gdy $ \lambda = 0 \implies R = \infty$ oraz $ \lambda = \infty \implies R = 0 $

\section{Couchy'ego - Hadamara}
Niech $ \sum_{n = 0}^{\infty}{a_n (x - x_0)^n}$ - szereg potęgowy.
$$ \lambda = \lim_{n \to \infty}{\sqrt[n]{|a_n|}} \implies R = \frac{1}{\lambda} $$
Przypadki skrajne jak w twierdzeniu $d'Alamberta$

\section{Twierdzenie o różniczkowalności szeregu potęgowego}
Niech $ \sum_{n = 0}^{\infty}{a_n (x - x_0)^n}$ - szereg potęgowy o promieniu zbieżnosci $R$.
Jeśli  $$ \sum_{n = 0}^{\infty}{a_n (x - x_0)^n} = f(x) \quad \forall x \in (x_0 - R , x_0 + R) $$ 
to
$$ f \in \mathcal{C}^{\infty}$$
$$ f^{(k)}(x) = \sum_{n = k}{k! {n \choose k} a_n(x-x_0)^{n-k}}$$



\section{Twierdzenie o całkowaniu szeregu}

Niech $ \sum_{n = 0}^{\infty}{a_n (x - x_0)^n}$ - szereg potęgowy o promieniu zbieżnosci $R$. Wtedy dla $x \in (x_0 - R , x_0 + R)$
$$ \int_{x_0}^x ( \sum_{n = 0}^{\infty}{a_n (x - x_0)^n}) dt = \sum_{n = 0}^{\infty}{\frac{a_n}{n+1} (x - x_0)^{n+1}}$$


\end{document}
