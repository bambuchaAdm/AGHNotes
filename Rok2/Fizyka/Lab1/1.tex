\documentclass[11pt]{article}
\usepackage[utf8]{inputenc}
%\usepackage[T1]{fontenc}
\usepackage{amssymb}
\usepackage{amsmath}
\usepackage{enumerate}
\usepackage{fullpage}
\usepackage{polski}  
\usepackage{indentfirst} 
\usepackage[pdftex]{graphicx}
\usepackage{multirow}
\usepackage{placeins}

\begin{document}

\section{Wiadomości teoretyczne}

\subsection{Prawo Ohma}
Natężenie prądu na rezystorze jest proporcjonalne do płynącego prądu $$ I = \frac{U}{R} $$
\subsection{Prawo indukcji Faradaya}
Siła elektromotoryczna pojawiająca się na zaciskach cewki zależy od tempa zmian strumienia elektoromagnetycznego.
$$ \varepsilon = \frac{-d\Phi_B}{dt} $$
\subsection{Samoindukcja cewki}
Zjawisko przeciwstawiania się zmianą prądu przez cewkę
$$ \varepsilon = L \frac{dI}{dt} $$
\subsection{Rezystancja}
Siła z jaką ciało przeciwstawia się przepływowi prądu stałego oznaczane jako $R$ 
\subsection{Reaktancja}
Rezystancja dla elementów o charakterze pojemnościowym lub indukcyjnym, oznaczane jako $X$
\subsection{Impedancja}
$$ Z = R + jX $$
\subsection{Konduktancja}
Zdolność do przewodzenia prądu. Wiąże się z rezystancją $$ G = \frac{1}{R}$$
\subsection{Susceptancja}
Zdolność do przewodzenia prądu w ciałach o charakterze pojemnościowym lub indukcyjnym
$$ B = \frac{1}{X} $$
\subsection{Admitancja}
Wielkość podobna do impedancji
$$ Y = \frac{Z} \quad Y = G + jB $$
\subsection{Przesunięcie fazowe}
W elementach 
\subsection{Krzywa namagnesowania ferromagnetyka}

\section{Sposób wykonania ćwiczenia}



\end{document}