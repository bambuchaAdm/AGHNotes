\documentclass[11pt]{article}
\usepackage[utf8]{inputenc}
%\usepackage[T1]{fontenc}
\usepackage{amssymb}
\usepackage{amsmath}
\usepackage{enumerate}
\usepackage{fullpage}
\usepackage{polski}  
\usepackage{indentfirst} 
\usepackage[pdftex]{graphicx}
\usepackage{multirow}
\usepackage{placeins}

\author{Łukasz Dubiel}
\title{Błędy pomiarowe}
\begin{document}

\maketitle

\section{Rodzaje błędu}
\begin{enumerate}
\item{Błędy grube - czyli pomyłki, należy je eliminować, takie jak błądy przeliczenia jednostek i inne}
\item{Błędy systematyczne - które można oraniczyć udoskonalając pomiar}
\item{Błędy przypadkowe - które podlegają prawą statystyki i rachunku prawdopodobieństwo  wynikające z zmiennych losowych przyczynków i nie da się wyeliminować}
\end{enumerate}

\section{Błędy systematycznej}
Gdy wartość średnia pomiarów jest przesunięta względem wartości oczekiwanej. Istnieje coś takiego jak minimalny błąd systematyczny jest określany przez błąd przyrządu pomiarowego.
\section{Błędy przypadkowy}
Gdy średnia jest bardzo blisko wartości oczekiwnej a jednak. Im większe odchylenie standardowe tym gorszy pomiar.

$$ \Phi(x) = \frac{dP(x)}{dx} $$

Wszystkie wykresy podaje się jako funkcję gęstości prawdopodobieństwa, dzięki temu prawdopodobieństwo możemy wyrazić 
$$ P(x) = \int_{x}^{x+dx} \Phi(x)dx $$

\section{Źródła błędu systematycznego}
\begin{enumerate}
\item{Skale mierników}
\item{Nieświadomy wpływ czynników zewnętrznych (temperatury,wilgotność) na wielkości mierzone}
\item{Niewłaściwy sposób odczytu (błąd paralaksy) lub pomiaru}
\item{Przybliżone charakter wzorów stosowanych do wyznaczenia wielkości złożonej.}
\end{enumerate}
\subsection{Poprawki}
Błędy systematyczne czasami można ograniczyć wprowadzając poprawkę, np.
$$ F = 6\pi \eta v(1 + 2,4\frac{r}{R}) $$

\subsection{Źródła błędu pomiarowego}
\begin{enumerate}
\item{Własności obiektu mierzonego - np. wahania średnicy drutu na całej jego długości}
\item{Własności przyrządu pomiarowego - np. zależą od przypadkowych drgań budynku, fluktuacja ciśnienia czy temeratury, czy docisku}
\item{Podłoże fizjologiczne - refelks, subiektywną ocenę maksimum czy równomierność oświetlenia poszczególnych części}
\end{enumerate}

\section{Typy oceny niepewności}
\subsection{Typ A}
Metody wykorzytujące statystyczną analizę serii pomiarów.
\begin{enumerate}
\item{Używamy metod statystycznych by je wyeliminować}
\end{enumerate}
\subsection{Typ B}
Operia się na naukowym osądzie eksperymentatora wykorzystującym wszystkie informacje o pomiarze i źródłach jego niepewności.
\begin{enumerate}
\item{Stosuje się gdy statystyczna analiza nie jest możliwa}
\item{Dla błędu systematycznego lub dla jednego wyniku pomiaru}
\end{enumerate}

\section{Do oceny typu B wykorzystać można}
\begin{enumerate}
\item{dane z pomiarów poprzednich}
\item{Doświadczenie i wiedzę na temat przyrządów i obiektów mierzonych}
\item{Informacja producenta przyrządów}
\item{Niepewność przypisane danym zaczerpniętymi z literatury}
\end{enumerate}

\section{Przykład}
Pomiar stałej grawitacji metodą wahadła matematycznego.

$$ T = 2 \pi \sqrt{\frac{l}{g}} $$
Mierzymy wielkości prost $ T, l $ - wartości , $ \Delta T = 1s, \Delta L = 1mm$ - maksymalna niepewność 
$$ \overline{g} = 9,78 \frac{m}{s^2} $$

$$ \Delta g = \frac{4\pi^2 \Delta L}{(\Delta T)^2} \quad \hbox{Do chrzanu!!!!}$$

$$ df = \frac{df}{dx}dx $$

$$ df = \sum_{i=1}^n \frac{df}{dx_i}dx_i $$

\section{Niepewność maksymalna}

$$ \Delta y = \left|\frac{\delta}{\delta x_1}\right| |\Delta x_1| $$

\section{niepewność standardowa}
$$ u_c(y) = \sqrt{ \sum_{i=0}^{n} \left[ \frac{dy}{dx_i} u(x_1) \right]^2} $$

\end{document}
