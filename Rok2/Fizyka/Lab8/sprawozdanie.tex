\documentclass[11pt]{article}
\usepackage[utf8]{inputenc}
%\usepackage[T1]{fontenc}
\usepackage{amssymb}
\usepackage{amsmath}
\usepackage{enumerate}
\usepackage{fullpage}
\usepackage{polski}  
\usepackage{indentfirst} 
\usepackage[pdftex]{graphicx}
\usepackage{multirow}
\usepackage{placeins}
\usepackage{multicol}
\usepackage{textcomp}

\author{Łukasz Dubiel}

\begin{document}

\section{Opracowywanie wyników}

L = 43 mH, C = 640 pF

\begin{tabular}{|r|r|r|r|r|r|r|r|r|r|}
\hline
	$R_d [\Omega$] & T[dz] & T[s] & $U_i$[V] & $U_{i+2}$[V] & $\beta$ & $R_p[\Omega]$ & $R_c[\Omega]$ & C[pF] \\  
\hline
	0 & 4,6 & $9,2 \cdot 10^{-4}$
	 & 2,2 & 1,4 & 489 & 42 & 42 & 502\\
	 
	14 & 4,6 & $9,2 \cdot 10^{-4}$
	 & 1,2 & 0,6 & 750 & 60,5 & 64,6 & 505\\
	 
	28 & 4,6 & $9,2 \cdot 10^{-4}$
	 & 2,6 & 1 & 1032 & 60,8 & 88,8 & 510\\
\hline
\end{tabular}

\begin{multicols}{2}

$$ \beta = \frac{\ln{\frac{U_i}{U_{i+2}}}}{T} $$

$$ R_d = 0 \implies  R_p = 2\beta L$$

$$ \omega = \frac{2\pi}{T} $$
$$ \omega = \sqrt{\omega_0^2 - \left( \frac{R}{2L} \right)^2} $$
\end{multicols}
$$ C = \frac{1}{L( \omega^2 - \beta^2)} $$
$$ \Delta C = \frac{1}{L}\left(\frac{1}{\omega}\Delta \omega + \frac{1}{\beta}\Delta \beta \right) = \frac{1}{0,043}(\frac{1}{6826}0,1 + \frac{1}{1032}0,1) = 0,0026 = 2,6*10^{-3} C $$
\bigskip
Przebieg aperiodyczny rozpoczynał się dla rezystancji o wartości (wyznaczane doświadczalnie)
$$ R_k = 480 \ \Omega $$
Teoretycznie rezystancja powinna wynieść
$$ R_k = 2 \sqrt{\frac{L}{C}} = 2 \sqrt{\frac{43 \cdot 10^{-3}}{640 \cdot 10^{-12}}} = 2\sqrt{6,7 \cdot 10^{6}} =  5,2 \cdot 10^{3} = 5,2\ k\Omega $$


\end{document}
