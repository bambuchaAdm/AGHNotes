\documentclass[11pt]{article}
\usepackage[utf8]{inputenc}
%\usepackage[T1]{fontenc}
\usepackage{amssymb}
\usepackage{amsmath}
\usepackage{enumerate}
\usepackage{fullpage}
\usepackage{polski}  
\usepackage{indentfirst} 
\usepackage[pdftex]{graphicx}
\usepackage{multirow}
\usepackage{placeins}
\usepackage{multicol}
\usepackage{textcomp}

\author{Łukasz Dubiel}

\begin{document}

\section{Opracowywanie wyników}

Dane zebrane poczas doświadczenia. Doświadczenie przeprowadzone dla L = 43mH, C = 640pF.
\begin{center}
\begin{tabular}{|c|c|c|c|c|c|}
\hline
	Lp. & $R_d [\Omega$] & T[dz] & T[s] & $U_i$[V] & $U_{i+2}$[V] \\
\hline
	1 & 0 & 4,6 & $9,2 \cdot 10^{-4}$
	 & 2,2 & 1,4 \\
	 
	2 & 14 & 4,6 & $9,2 \cdot 10^{-4}$
	 & 1,2 & 0,6 \\
	 
	3 & 28 & 4,6 & $9,2 \cdot 10^{-4}$
	 & 2,6 & 1 \\
\hline
\end{tabular}
\end{center}
Wyznaczamy współcznynik tłumienia $\beta$.
\begin{multicols}{2}
$$ \beta = \frac{ \ln{\frac{U_i}{U_{i+2}}}}{T} $$
\begin{center}
\begin{tabular}{|c|c|}
\hline
Lp. & $\beta$  \\  
\hline
1 & 489 \\
2 & 750 \\
3 & 1032 \\
\hline
\end{tabular}
\end{center}
\end{multicols}
Wyznaczaomy rezystancję pasożytniczą indukcyjności.
Do tego użyjemy pomiaru dla wyzerowanej rezystancji dekadowej $R_d = 0$.
$$ R_p = 2 \beta_1 L == 2 *489 * 43 * 10^{-3} = 42 \Omega $$
\begin{multicols}{2}
Teoretyczny wpółcznynnik tłumienia dla wartości.
$$ \beta_{teo} = \frac{R_c}{2L}  $$
\begin{center}
\begin{tabular}{|c|c|c|}
\hline
Lp. & $\beta$ & $\beta_{teo}$  \\  
\hline
1 & 489 & 489,13\\
2 & 750 & 651,1\\
3 & 1032 & 814 \\
\hline
\end{tabular}
\end{center}
\end{multicols}

\begin{multicols}{2}
Pulsacja obwodu rezonansoweog wynosi.
$$ \omega = \frac{2\pi}{T}  $$
\begin{center}
\begin{tabular}{|c|c|c|}
\hline
Lp. & $\omega$  \\  
\hline
1 & \multirow{3}{*}{6829}\\
2 & \\
3 & \\
\hline
\end{tabular}
\end{center}
\end{multicols}
Posiadając już pulsację oraz logarytm tłumiania możemy wyznaczyć z wzoru pojemność kondensatoar
$$ C = \frac{1}{L( \omega^2 - \beta^2)} $$

\begin{center}
\begin{tabular}{|c|c|}\hline
Lp. & C[pF] \\  
\hline
1 & 502\\
2 & 505\\
3 & 510\\
\hline
\end{tabular}
\end{center}

\newpage
Wyznaczenie niepewności i błędu pomiarowego.
$$ \Delta C = \frac{1}{L}\left(\frac{1}{\omega}\Delta \omega + \frac{1}{\beta}\Delta \beta \right) = \frac{1}{0,043}(\frac{1}{6826}0,1 + \frac{1}{1032}0,1) = 0,0026 = 2,6*10^{-3} C $$
\bigskip
Przebieg aperiodyczny rozpoczynał się dla rezystancji o wartości (wyznaczane doświadczalnie)
$$ R_k = 480 \ \Omega $$
Teoretycznie rezystancja powinna wynieść
$$ R_k = 2 \sqrt{\frac{L}{C}} = 2 \sqrt{\frac{43 \cdot 10^{-3}}{640 \cdot 10^{-12}}} = 2\sqrt{6,7 \cdot 10^{6}} =  5,2 \cdot 10^{3} = 5,2\ k\Omega $$


\end{document}
