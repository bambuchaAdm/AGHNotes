\documentclass[11pt]{article}
\usepackage[utf8]{inputenc}
%\usepackage[T1]{fontenc}
\usepackage{amssymb}
\usepackage{amsmath}
\usepackage{enumerate}
\usepackage{fullpage}
\usepackage{polski}  
\usepackage{indentfirst} 
\usepackage[pdftex]{graphicx}
\usepackage{multirow}
\usepackage{placeins}

\author{Łukasz Dubiel}
\title{EE \\ Złącze p-n}

\begin{document}

\maketitle

\section{Pojemność złączowa}
powstaje w obszarze zubożonym, dlatego dominuje przy polaryzacji zaporowej złącza

$$ C_j = \frac{\epsilon A}{l_d} $$

\section{Pojemności dyfuzyjna}
Dominuje przy polaryzacji złącza w kierynku przewodzenia, wynika z opóźnienia zmian napięcia względem zmian prądu.

$$ C_d = \tau_p \frac{I_d}{U_t} $$

\section{Pulsacja graniczna}
$$ \omega_{gr} = \frac{1}{r_s C_j} $$

\end{document}
