\documentclass[11pt]{article}
\usepackage[utf8]{inputenc}
%\usepackage[T1]{fontenc}
\usepackage{amssymb}
\usepackage{amsmath}
\usepackage{enumerate}
\usepackage{fullpage}
\usepackage{polski}  
\usepackage{indentfirst} 
\usepackage[pdftex]{graphicx}
\usepackage{multirow}
\usepackage{placeins}

\author{Łukasz Dubiel}
\title{Zasilanie elementów}

\begin{document}

\maketitle

\section{Źródło napięcia}


\section{Źródło prądu}

\section{Dzielnik napięcia}

\section{Dzielnik prądu}

\subsection{Masa (Węzeł odniesienia)}

\subsection{Problemy z obciążeniem}

\section{Dzielnik prądu}

\section{Zasilanie symetryczne}

\section{Kondensator}

\subsection{Definicja}
Element konserwatywny, przechowuje ładunek

\subsection{Idealny}

\subsection{Elektorlityczny}

\subsection{Statność kondensatora}
Rysunek

\newpage
\section{Układ całkujący}

$$ U_2(t) = U_1 \left( 1 - e^{-\frac{t}{\tau}} \right) $$

\subsection{Relacja stałej czasowej do okresu}


\section{Układ różniczkujący}

\subsection{Relacja stałej czasowej do okresu}

\section{Zastosowanie kondensatorów}
\subsection{Filtrowanie zasilania}

\subsection{Sprzęganie}

\subsection{Filtry}

\subsection{Obwód rezonansowy} 

\subsection{Kondensator w układach czasowych} 

\subsection{Przerzutnik astabilny}

\subsection{Przerzutnik monostabilny}

\subsection{Pojemności a maksymalna częstotliwość}

\subsection{Pojemności a czas propagacji}

\section{Cewka}

\subsection{Definicja}

$$ U = L \frac{dI_L}{dt} $$

\subsection{Samoindukcja}

\subsection{Cewka nieidealna}

Diagram wskazowy

$$ Q = \frac{X_L}{R} $$

\section{Układ całkujący}

\section{Układ różniczkujący}

\end{document}
