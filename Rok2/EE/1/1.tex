\documentclass[11pt]{article}
\usepackage[utf8]{inputenc}
%\usepackage[T1]{fontenc}
\usepackage{amssymb}
\usepackage{amsmath}
\usepackage{enumerate}
\usepackage{fullpage}
\usepackage{polski}  
\usepackage{indentfirst} 
\usepackage[pdftex]{graphicx}
\usepackage{multirow}
\usepackage{placeins}

\author{Łukasz Dubiel}
\title{Rezystory,Konensatory,Cewki}

\begin{document}

\maketitle

\section{Rezystor}

To element bierny. Zamienia energię elektryczną na cieplną.

\section{Prawo Ohma}
Natężenie prądu stałego $I$ jest proporcjonalne do całkowitej siły elektromotorycznej w obwodzie zamkniętym.

$$ I ~ U $$
$$ I = G U $$
$$ I = \frac{U}{R}$$

\subsection{Opis}
\begin{enumerate}
\item{Matematycznym - równienie $I = f(U)$ , $par = f(f)$}
\item{Graficzny}
\item{Katalogowy}
\end{enumerate}

\subsection{Charakterystyka prądowo-napięciowa}

\subsection{Częśtotliwościowe}

\subsection{Czasowe}
Wykres odpowiedzi czasowej na jakieś pobudzenie (napięcie lub prąd na wejściu). 

\subsection{Parametry}
\begin{enumerate}
\item{Rzystancja nominalna - wartość podawana przez producenta na obudowie}
\item{Tolerancja -dopuszczalna różnice między rzeczywistą wartością rezystancji a wartością nominalną}
\item{Moc znamionowa - wartość mocy która może się wydzielić w rezystorze w postaci ciepła (przy danej temperaturze) i nie ulegnie on zniszczeniu.}
\item{Napięcie znamionowe - największa wartość napięcia stałego (lub skuteczna napięcia przemiennego), która można doprowadzić do końcuwek rezustora nie powodując jego uszkodzeniach}
\item{Temperaturowa współczynnik rezystancji (TWR) - jak zmienia się rezystancja w zależności od temperatury} 
\end{enumerate}



\section{Oznaczenia}

\section{Normalizacja}
Szeregi E3,E6,E12,E96

\section{Rezystywność}
Wszystko (z wyjątkiem nadprzewodników).

\section{Podział rezystorów}
Ze względu na ustawianie
\begin{enumerate}
\item{Stałe}
\item{Nastawene}
\item{Półprzewodnik}
\end{enumerate}
Produkt
\begin{enumerate}
\item{Drutowe}
\item{masowe}
\item{warstwowe}
\end{enumerate}

\end{document}
