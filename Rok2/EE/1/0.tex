\documentclass[11pt]{article}
\usepackage[utf8]{inputenc}
%\usepackage[T1]{fontenc}
\usepackage{amssymb}
\usepackage{amsmath}
\usepackage{enumerate}
\usepackage{fullpage}
\usepackage{polski}  
\usepackage{indentfirst} 
\usepackage[pdftex]{graphicx}
\usepackage{multirow}
\usepackage{placeins}

\author{Łukasz Dubiel}
\title{Elementy Elektroniczne}

\begin{document}

\maketitle

\section{Co juz wiemy. Do Czego służą elementy.}

Teoria obwodów (elementy liniowe) $ \Rightarrow $ Elementy elektroniczne (jak zrobić z tego co jest po prawej to co jest po lewej) $ \Rightarrow $ Układy elektroniczne (elementy nieliniowe).


\section{Element a przyrząd}
\subsection{Element definicja}
Najprostsza fcześć układu elektorniczengo stanowiąca konstrukcję całości, ma pewnę własnośći i spełnia pewną określoną elementarną funkcję. 
\subsection{Przyrząd} 
Funkcjonalny składnik ukłądu elektrnicznego często skłądający się z kilku elementów spełniających pewne założonej.
\section{Układ elektniczny }
\begin{enumerate}
\item{Całość skłądających się z pewnych elementów}
\end{enumerate}

\section{System}
Zbiór odpowiednio połączonych i

współpracującyuch ze sobą układów elektronicznych.

\section{Urządzenie}
System lub ich zbiór stanowiący funkcjonalną cąłoość służący do określonych celów.

\section{Elektronikiem}
Dziedzina tachniki i nauki zajmującą się wytwarzaniem i przetwarzaniem sygnałów w postaci prądów i napięć elektrycznych lub pól elektromagnetycznych.

\section{Podział elementów}

\subsection{Bierne}
Takie które nie mają wpływu na przepływ energi.

\subsection{Aktywne}
Mogą sterować przenoszoną mocą.

\section{Omawiane elementy}
\begin{enumerate}
\item{Rezystor,Kondensator,Cewka}
\item{Fuzyka półprzewodników, złącze p-n}
\item{Dioda,aktywne}
\end{enumerate}

\section{Egzamin}
pisemne lub ustne
\section{Laboratorium}
Wykonywanie ćwieczeń ( będzie ich dziewięć ) i zaliczenie kolekwium.
\section{Ocena końcowa}
Średnia z ocen uzyskanych z egzaminu i zliczania laboratorium (z przewagą egzaminu).

\section{Książki}

Przyrządy półrzewdodnikowe i układy scalne



\end{document}
