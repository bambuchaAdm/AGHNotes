\documentclass[11pt]{article}
\usepackage[utf8]{inputenc}
%\usepackage[T1]{fontenc}
\usepackage{amssymb}
\usepackage{amsmath}
\usepackage{enumerate}
\usepackage{fullpage}
\usepackage{polski}  
\usepackage{indentfirst} 
\usepackage[pdftex]{graphicx}
\usepackage{multirow}
\usepackage{placeins}
\usepackage{wasysym}
\usepackage{xspace}
\usepackage{import}

\begin{document}
\begin{center}
\begin{tabular}{|c|c|c|c|c|}
\hline
Wydział & \underline{Łukasz Dubiel} & Rok & Grupa & Cw. nr \\
%\hline
EAIE & Daniel Iziorov, Piotr Mędrek & I & Grupa 6 & 1 \\
\hline
Data wykoania & \multicolumn{3}{|l|}{Temat} & Ocena \\
2.03.2012 &  \multicolumn{3}{|p{10.5cm}|}{Charakterystyki stffałoprądowe złącza p-n. diody prostownicze i specjalne} & \\
\hline
\end{tabular}
\end{center}

\section{Pomiary charakterysytk diod w kierunku przewodzenia}
\VZero
\subsection{Diody prostownicze}
Zebrane charakterystyki

%\includegraphics[scale=0.6]{out/prostownicza-linia.png}
\end{document}
