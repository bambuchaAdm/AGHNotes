\documentclass[11pt]{article}
\usepackage[utf8]{inputenc}
%\usepackage[T1]{fontenc}
\usepackage{amssymb}
\usepackage{amsmath}
\usepackage{enumerate}
\usepackage{fullpage}
\usepackage{polski}  
\usepackage{indentfirst} 
\usepackage[pdftex]{graphicx}
\usepackage{multirow}
\usepackage{placeins}
\usepackage{wasysym}
\usepackage{xspace}
\usepackage{import}
\usepackage{multicol}

\begin{document}
\begin{center}
\begin{tabular}{|c|c|c|c|c|}
\hline
Wydział & \underline{Łukasz Dubiel} & Rok & Grupa & Cw. nr \\
%\hline
EAIE & Daniel Iziorov, Piotr Mędrek & I & Grupa 6 & 1 \\
\hline
Data wykoania & \multicolumn{3}{|l|}{Temat} & Ocena \\
2.03.2012 &  \multicolumn{3}{|p{10.5cm}|}{Charakterystyki stffałoprądowe złącza p-n. diody prostownicze i specjalne} & \\
\hline
\end{tabular}
\end{center}

\section{Pomiary charakterysytk diod w kierunku przewodzenia}

\subsection{Diody prostownicze}
Zebrane charakterystyki
\begin{center}
\includegraphics[scale=0.45]{out/prostownicza-normal.png}
\includegraphics[scale=0.45]{out/prostownicza-log.png}
\end{center}
Ogólne równanie diody
$$ i_d = I_{gr0}(e^{\frac{u_d - r_di_d}{U_t}} - 1) + I_0(e^{\frac{u_d - r_di_d}{U_t}} -1) $$
Dla $ U_f > 100mV$ można przująć przybliżenie
$$ i_D = I_se^\frac{u_d}{nU_t} $$

Napięcie progowe wynosi
$$ U_k =  0.7144 $$

Fragmenty małych prądów i rekombinacyjny jest bardzo mały w charakterystyce, wynika z tego mocno dyfuzyjny charakter diody.
W uzyskanej charakterystyce można wyznaczyć trzy obszary. Gdy $\frac{U}{U_t} \in <16,20>$ dioda pracuje w obszarze dominacji prądu dyfuzyjnego. Gdy $\frac{U}{U_t} \in <20,27>$ obszar pracy przechodzi do obszaru prądów unoszenia. Powyżej znajduje się obszar omowy.

Używając metody regresji lniowej na obszarze dyfuzyjnym uzykujemy wynik
$$ a =  0.2378 \quad b = -8.1374 $$

Charakterystyka pół logartymiczn w obszarze dyfuzujnym odpowiada
$$ \ln{I_d} = \frac{u_d}{nU_t} + I_0$$
Co daje równość
$$ a = \frac{1}{n} \quad b = I_0 $$

Czyli odpowiednie wartośći są równe
$$ n = 4,20 \quad I_0 = 2924 \mu A $$

Gdzie $n$ to współczynnik niedoskonałości łącza, a $I_0$ to wartość prądu dyfuzyjnego

Wiadomym jest że $ I_0 < I_{GR0} $ tak więc, dla przyjętnego uproszczonego równia współczynik 
$$ I_s = I_0 $$
$$ I_s = 1812 \mu A $$

Rezystancja szeregowa wynosi
$$ r_s = \frac{\Delta U}{I_d} = \frac{U_d - U_k}{I_d} $$
Obliczenia wykonane dla prądu $20mA$
$$ r_s = \frac{0,756-0.7144}{0.02} = 2.08 \Omega $$

\section{Dioda germanowa}
Charakterystyki 
\begin{center}
\includegraphics[scale=0.48]{out/germanowa-normal.png}
\includegraphics[scale=0.48]{out/germanowa-log.png}
\end{center}
\begin{description}
\item[Napięcie progowe $U_k$] $=0,218\ V$
\item[Współczynnik niedoskońałośći złącza $n$]  $=3.899$
\item[Prąd dyfuzyjny $I_0$ ] $=15,05\ mA$
\item[Rezystancja szeregowa ($I=75mA$) $r_s$ ] $=0,934\ \Omega$
\end{description}
\section{Dioda impulsowa}
Charakterystyki 
\begin{center}
\includegraphics[scale=0.48]{out/germanowa-normal.png}
\includegraphics[scale=0.48]{out/germanowa-log.png}
\end{center}
\begin{description}
\item[Napięcie progowe $U_k$] $=0,7477\ V$
\item[Współczynnik niedoskońałośći złącza $n$]  $=4,2542$
\item[Prąd dyfuzyjny $I_0$ ] $=223\ \muA$
\item[Rezystancja szeregowa ($I=73mA$) $r_s$ ] $=1,924\ \Omega$
\end{description}

\section{Diody stabilizacyjne}
\includegraphics[scale=0.48]{out/stabilizacyjne.png}


\end{document}
