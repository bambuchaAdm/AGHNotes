\documentclass[11pt]{article}
\usepackage[utf8]{inputenc}
%\usepackage[T1]{fontenc}
\usepackage{amssymb}
\usepackage{amsmath}
\usepackage{enumerate}
\usepackage{fullpage}
\usepackage{polski}  
\usepackage{indentfirst} 
\usepackage[pdftex]{graphicx}
\usepackage{multirow}
\usepackage{placeins}
\usepackage{wasysym}
\usepackage{xspace}
\usepackage{import}
\usepackage{multicol}

\begin{document}
\begin{center}
\begin{tabular}{|c|c|c|c|c|}
\hline
Wydział & \underline{Łukasz Dubiel} & Rok & Grupa & Cw. nr \\
%\hline
EAIE & Daniel Iziorov, Piotr Mędrek & I & Grupa 6 & 9 \\
\hline
Data wykoania & \multicolumn{3}{|l|}{Temat} & Ocena \\
29.05.2012 &  \multicolumn{3}{|p{10.5cm}|}{Tranzystory polowe MOS} & \\
\hline
\end{tabular}
\end{center}

\section{Charakterystyki przejściowe tranzystora n-mos}

\begin{center}
\includegraphics[scale=0.45]{out/n-przejsciowa.png}
\end{center}

Charakterystyka półpierwiastkowa (dla $U_{ds} = 4V$)

\begin{center}
\includegraphics[scale=0.30]{out/n-pierwiastek.png}
\end{center}

$$ U_t = 1,63V $$


\subsection{Prądy nasycenia $I_{dd}$}
\begin{center}
  \begin{tabular}{|c|c|c|c|c|c|c|c|c|c|c|c|}
    \hline
    $U_{ds} [V]$ & 0 & 1 & 2 & 3 & 4 & 5 & 6 & 7 & 8 & 9 & 10 \\
    \hline
    $I_{dd} [mA]$ & 0 & 12 & 20 & 25 & 27 & 27 & 28 & 28 & 28 & 28 & 28 \\
    \hline
  \end{tabular}
\end{center}

\section{Charakterystykaw wyjściowa n-mos}

\begin{center}
\includegraphics[scale=0.45]{out/n-wyjsciowa.png}
\end{center}

\subsection{Współczynik $\lambda$}
Wyznaczamy z przeicięcia lini regresji z obszaru nasyconego z osią OX
\bigskip

\begin{tabular}{|c|c|c|c|c|c|c|c|c|c|c|c|}
\hline
$U_{gs}[V]$ & 0 & 1 & 2 & 3 & 4 & 5 & 6 & 7 & 8 & 9 & 10 \\
\hline
$\lambda$ & 0.03&-0.006&-0.034&-0.02&-0.017&-0.014&-0.011&-0.01&-0.009&-0.009&-0.011 \\
\hline
\end{tabular}

\bigskip
Rozrzut wynika z braku dokłąności przyrządów pomiarówych oraz przyjętej metody wyznaczania regresji. Zgodnie z rozkładem normanym najbardziej prawodopodobną wartośćią dla współczynika modulacji długośći wynosi.
$$ \lambda = -0.01$$

\section{Charakterystyka przejściowa p-mos}

\end{document}
