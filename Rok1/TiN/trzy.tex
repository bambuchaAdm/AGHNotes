\documentclass{article}
%\usepackage[utf8]{inputenc}
%\usepackage{polski}


\title{Technologia informatyczna}
\author{Łukasz Dubiel}

\begin{document}

\maketitle
\section{Zapis ułamków}
Zapis stałoprzecinkowy \\
Na podstawie tego systemu działają wszystkie procesory sygnalowe

\bigskip
Zapis zmiennoprzecinkowy \\
Używa zapisu znormalizowanego, IEEE-P754.
Definiuje kilka dodatkowych układków bitów nie będących znormalizowanymi liczbami np. NaN, 0 itp \\
Definiuje się kilka długości liczby \\
\begin{tabular}{|cccccccc|}
\hline
	Długość & Znak & Długość wykładnika & Zakres & Długość mantysy & Dokładność \\
	32 & 23 & 24 & 25 & 26 & 27 \\
	32 & 33 & 34 & 35 & 36 & 37 \\
	42 & 43 & 44 & 45 & 46 & 47 \\
\hline
\end{tabular}\\
Znaki specjalne\\
\begin{tabular}{|ccc|}
\hline
	11 & 12 & 13\\
	21 & 22 & 23\\
	31 & 32 & 33\\
\hline
\end{tabular}\\
Kosztowne obliczeniowo\\
Minusy
Nie można bezpośrednio porównywać ( należy prowadzić epsilon)

\section{LittleEndian vs BigEndian}
\subsection{Little-Endian}
Jako początek leci do pamięci najmniej znaczący bit ( najmłodszy bit ).

Intel, AMD
\subsection{Big-Endian}
Na początku dane lecą od najbardziej znaczącego bitu.

\newpage
\section{Dane wektorowe}
Ogrom danych, niskiej jakości ( np dziwięk wielokanalowy (16 bitów)).

Procesor obsługuje wiele danych na takakt poprzez umożliwiwanie operacji na dużych słów (128 bitów, x86, SSE)

Operacje na raz kilka paczek.
\section{Zapis danych logicznych}
Do jeden informacji - bit \\
Kompilator ma to w dupie ( jak chce to może sobie zapisać nie na jednym ale na dwóch )








\end{document}