\documentclass[11pt]{article}
\usepackage[utf8]{inputenc}
%\usepackage[T1]{fontenc}
\usepackage{amssymb}
\usepackage{amsmath}
\usepackage{enumerate}
\usepackage{fullpage}
\usepackage{polski}  
\usepackage{indentfirst} 
\usepackage[pdftex]{graphicx}
\usepackage{multirow}
\usepackage{placeins}


\author{Łukasz Dubiel}
\title{LIRK \\ Lisp AVR with assembly Kernel}

\begin{document}

\maketitle

\section{Wprowadzenie}

Na platformie AVR istnieje już kilka języków mających ugruntowaną pozycję oraz zestaw narzędzi. Kilka głównych z nich to C, Basecom czy Pascal. Większość z nich to języki niezależne od platformy. Jednak mając takie zabawki jakimi są mikrokontrolery z rodziny AVR operacje języki które operują na pełnych słowach czy wielokrotnościach stają się częściowo niewygodne. Dlatego postanowiłem spróbować innego podejścia do problemu.

\subsection{Lisp}

W roku 1958 John McCarty na MIT stworzył język programowania który został nazwany Lispem. Był to drugi, zaraz po FORTRANie język wysokiego poziomu. Przez ten czas Lisp ewoluował, jak również pojawiały się różne dialekty (oczywiście nie kompatybilne z sobą) aż stworzyły całą rodzinę języków. Obecnie gdy używa się słowa Lisp to mówi się o całej rodzinie języków lub o Common Lispie, dialekcie ustandaryzowanym przez ANSI w 1994 z późniejszymi modyfikacjami.

\subsection{Motywacje}

Z racji mojego dużego zainteresowania językami z rodziny Lispów oraz problemów jakie napotyka C w momencie kiedy trzeba operować na pojedynczych bitach a nie słowach zdecydowałem się na stworzenie "dialektu" Lispa na rodzinę AVR. 

Początkowo miało to być ćwiczenie w celu lepszego poznania Common Lispa, lecz potem zdecydowałem się, że to właśnie jego będę wykorzystywał w moich hobbystycznych projektach.

\section{Zamierzenia i cele}
Lirk nie ma (na razie) ambicji stania się nowym językiem programowania. Ma jedynie udostępniać funkcje które zminimalizuje ilość "nie potrzebnego" kodu w konstrukcjach stricte assemblerowych. Jak już wspomiałem na wstępie nazwa jest troszczekę myląca, gdyż Lirk w runtime korzysta tylko i wyłącznie z asemblera, jednak w momencie kompilacji (i związanej z nią generowaniem kodu) dostępne są wszystkie mechanizmy Common Lispa takie jak makra, funkcję wyższego rzędu, Common Lisp Object System (CLOS), Meta Object Protocol  MOP oraz wiele innych.  W dalszej części referatu przedstawię cele.


\subsection{Definicje oraz wołanie procedur}

Jednym z pierwszych problemów jakie nękały programistów była duplikacja kodu. Były badania wskazujące na to, że  średnia ilość błędów na jednostkę kodu (KLOC czyli tysiąc linii) jest stała i zależy do wielkości projektu [Potrzebne źródło]. Z racji tego, że assembler jest najmniej ekspresywnym językiem (ilość kodu potrzebna do wykonania zamierzonej akcji jest dość pokaźna) jest podatny na duplikacje. 

Na tak małej platformnie jaką jest AVR bardzo dobrze sprawuje się programowanie proceduralne. Na podobnych platformach działało C 30 lat temu. Lekko problematyczne w C jest to iż, każda procedura działa na zasadzie skoku i powrotu (standard C99 zaczerpnął z C++ słowo kluczowe inline, jednak sprawia on, że funkcja nie zawsze jest rozwijana w miejscu). 

Lirk z zamierzeniu ma oferować dla programisty obie te metody. Procedura może być wywołana przez skok z odłożeniem adresu powrotu na stos, jak i być rozwinięta w miejscu.

funkcję można wywołać bezpośrednio w celu rozwinięciu jej w miejscu.

[listing z rozwinięcie w miejscu]

albo wywołać jak wykonując skok

[listing z wykonaniem w programie]

Oczywiście w tym przypadku kompilator automatycznie zarządza etykietami oraz dodanie odpowiedni fragment kodu z istrukcjami powrotu.

Dodatkowo powinien w planach jest dodanie funkcji, że raz wołana funkcja jest rozwijana w miejscu, bez dodania niepotrzebnych instrukcji i etykiet. Lecz jeżeli programista wykona wołanie dwukrotnie funkcję, w assebly pojawiają się wszystkie potrzebne elementy by wykonać  

Oferuje to większą granulację kodu, oraz ułatwia testowanie jednostkowe. Nawet można się pokusić o użucie TDD [potrzebne zródło] jednak wymaga to napisanie odpowiednich narzędzi, które nie byłyby aż tak problematyczne.

\subsection{Tworzenie rozgałęzień sterowania}
Asembler oferuje kilkanaście instrukcji warunkowych. Są one w postaci predykatów. Język predykatowy jest naturalnym sposobem tworzenia warunków w językach funkcyjnych. 

[listing lirka oraz C, dwie kolumny ]

Utrzymywanie bardzo rozgałaziąnego kodu w assembly wymaga bardzo dużego skupienia oraz bardzo rozległego modelu mentalnego. W przypadku assembly dużo więcej jest w głowie programisty niż jest napisane.
Lirk ma na celu lekkie odzielenie oraz osłabienie wymagań od programisty oraz większe zgrupowania akcji powiązanych z sobą.

W asseblerach predykaty te pozwalają jedynie na przeskok o zadanych offset. Assemby umożliwia wyznaczenie tego offsetu przy pomocy etykiet. Dla skomplikowanych kodów nie jest wygodne, gdy trzeba użyć duże ilośći etykiet, W językach wysokiego poziomi zdecydowano się na bloki. Podbone rozwiazanie chcę umożliwić w Lirku. Konstrukt jezykowy ma zapewnić, by programista nie musiał wymyślać kolejnego identyfikatora w kodzie (co gdy musi robić cześciej niż powinien, prowadzi do tworzenia nie zrozumiałemych identyfikatorów). Jednoczensie Lirk zachowduje zgodność wsteczną, i gdy programista chce, może wykorzystać dalej stary moduł konstrukcyjny. 

Oczuwustym jest, że jeżeli wprowadza się do języja predykaty, czasami potrzebne jest stworzenie tworu barziej skomplikowanego, będącego złączeniem kilku mniejszych predykatów. Problem może być rozwiązany przy pomocy mechanizmu makr pozwalającym na generowanie kodu w momencie wykonania. 

Rozwiązania oparte na predykatach (które w większości stosuje programowanie funkcyjne) bardzo dobrze mapuje się na instrukcje rozgałęzienia sterowania w kodzie.  


\end{document}
