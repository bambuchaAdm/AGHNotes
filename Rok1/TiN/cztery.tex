\documentclass{article}
\usepackage[utf8]{inputenc}
\usepackage{polski}
\title{Czym są bity w świecie cyfrowym}
\author{Łukasz Dubiel}
\begin{document}
\maketitle

\section{Logika matematyczna}

\section{Algebra Bool'a}
\subsection{Postulat 1 - identyczność}
Zero jest elementem neutralnym dla dodatawania ( sumy ) \\
Jeden jest elementem neutralnym dla mnożenia (iloczynu bitowego);
\subsection{Element dopełniające}
\begin{displaymath}
\exists\ {x'} : x + x' = 1
\end{displaymath}
\subsection{Rozdzielności}
\subsection{Prawa DeMorgana}
\section{Dlaczego lubimy systemy binarny}
Poziomom napiecia reprezentuje dany symbol.
Np mechanizmy RS232.
\subsection{Pozimy logiczne}
CMOS,CMOS/TTL,TTL,GTL
\bigskip

Wszystkie współczesne systemy cyfrowe wymagają zebara.

Czas propagacji oraz stromość zbocza.

Definicja od połowy zmiany stanu

Przeplatanka (paróweczki) wymaga trznsmisji danych.
Ma szerokośc i ile trwa jeden bit.

Zbicia parówek w wielką kiełbase jeśli zaczniemy konstrukcję na równoległą.
\section{bramki}
\subsection{bufor}
\subsection{and}
\subsection{or}
\section{Negancja bramek}
By zanegować funkcję logiczną wystarczy dorysować kółko na wyjściu.

\section{Moc wydzielana na układach CMOS - straty mocy}
\begin{displaymath}
P_{dynamiczne} = ( C_{zew} + C_{wew} ) f V^2_{DD}
\end{displaymath}
$P_{q-s}$-prąd zwarcia \\
$P_{static}$-prądy upływu ( tak 

\section{TDP}
Totla Desctiption Power

\section{Jak się robi scalaki}

\end{document}