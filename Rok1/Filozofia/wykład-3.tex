\documentclass[11pt]{article}
\usepackage[utf8]{inputenc}
%\usepackage[T1]{fontenc}
\usepackage{amssymb}
\usepackage{amsmath}
\usepackage{enumerate}
\usepackage{fullpage}
\usepackage{polski}  
\usepackage{indentfirst} 
\usepackage[pdftex]{graphicx}
\usepackage{multirow}
\usepackage{placeins}

\author{Łukasz Dubiel}
\title{Filozofia - pierdolenia ciąg dalszy}

\begin{document}

\maketitle

Wspominał o kontynentalne rozumowanie ( Kartezjusz ), cała żeczywistość dzieli się na dwie cześci, rzeczy rozciągłej i rzeczy myślącej ( nie pamiętam )

Oświeceniowej materalizmowi , usunięcie rzeczy myślącej i zostawienie rozciągłej. Istnieje tylko materia, nie istnieje rzecz myśląca jako soobny byt, jest to szczególna materia. Informacje dostajemy przez narządy zmysłów. 
Myśl stara ( Demokryt ) i jazda dalej.

Klasyczna filozofia niemiecka i pozytywizm francuski  ( rozpowszechnienie w kulturze anglo-saskiej )

\section{Kantyzm}
1780r. - sporo po empiryźmie subiektywnym, oraz materialiźmie. Jest to reakcja na dwa kierunki. Łaczy się z klimatem politycznym w krajach niemieckich, domienne od rewolucyjnej francji. Drugi miejsce po Artystotelesie jeśli chodzi o ilość cytowań. Nie łatwa dokrytna, ale coś omówimy,

\subsection{Teoria Poznania}
"Niebo gwaździste nademną, prawo moralne we mnie" - etyka"

Dla Bererey i Humea ( brystyjscy idealiści subiektywni ) nie mamy podstaw do tego żeby na zewnątrz jest coś co o nas informuje. 

Cant zaczyna problem tam gdzie Hume kończy i nie ma przebacz... Dla Hume wiedza o zewnętrznych bytach używamy asocjacją i dlatego możemy wiedzieć co się dzieje w świecie. NA podstawie bodźców możemy dowiedzieć się o idei badanego przedmiotu.

Dla Hume wzorce wynikają z przyzwyczajenia ( więcej nie śmiga ). ( długa dygresja o tym że nie wie o co chodzi )

Cant stwierdza że takie podejście jest naiwne. Uwzględniał opóźnienia w przyjmowania impresji. Jak datagramy IP. Teraz pytanie jak defragmentujemy konstrukcje. Cant używa apriorycznej ( ma specyficzne znaczenie ) struktury zbierania eventów. Aprioryczny umożliwia takich konstrukcji. 

W ten sposób nasza zmysłowość, funduje sobie taką możliwość, żeby postrzegać coś, gdzieś i kiedyś. Subiektywne umieszczenie kierunku innych rzeczy.
Z tego wynika że nasze wnioski nie są skutkiem doświadczenia, lecz je umożliwiają. Czyli to my jesteśmy factorką która generuje event handlery. To my mamy potem łączyć te datagramy ( evnety, które są pofragmentowane ) tylko na postawie chwilowo.
Rozum to struktura złożona. Na niższym poziomie wykazuje postrzeganie większych struktur z naszy eventów ( anliza przyczynowo skutkowa ). 

Ingarden dawał kawałek który jest dobrą polemiką ( Krytyka czystego intelektu ). (niewyraźne) Intelekt pozwala na skłądanie pofragmentowanych datagramów.

Podział intelektu na części. Najbardzej znane są dwie ( aprioryczne dwie, ( nie pamiętam jakie) ) ale nie zgadzał się, że to nie będzie czyste doświadczenie. Ani aprioryczne formy zmysłowośći ani doświadczenia nie mają racji bytu ( należą do zbioru pustego ) w filozofi hume

Idziemy indukcją (uogulnienie, mam mambę więc wszyscy mają ją ), nasdtąpiło krytytka natego narzędzia ( wkońću nie wszyscy mają mambe ), żadna naszej wiedzy nie może być pewna ani absolutna. Indukcja nie śmiga w takich konstrukcji.

Cant uważa, że nie wszystko co wiemy na tym świecie opiera się, lub może się opierać na indukcji.
W czystym przyrodoznastwie czy geometria, to prawa postawowe ( prawa newtona czy aksjomaty ekulidesa ) będące skutkiem indukcji mogą być, a nawet są.
Czysty rozum poprzedza wszystkie możliwe doświadczenia.

Na trzech poziomach rozumu mamy aprioryczne formy zmysłowości, aprioryczne formy zrozumienia i formy syntetyczne. Najwazniejsza to matematyka, a dokładniej geometria będąca ( nie pamiętam czym)

Pewne nasze odpowiedz nie mogą być weryfikowane przez nasze doświadzczenie. Przez co nasza wiedza nie opiera się tylko na doświadczeniu. Do tego trzeba dodać wiedzę, na temat relacji miedzy doświadzeniach ( rachunek predykatów I rzędu ). Formy zdaniowe są skutkiem doświadczenia, przez co ( od sceptyków ) nie są pewne.

Pierwsza kategoria sądów jest oczywista.Ale nic nie mówi o rzeczywistości. Dodawania czegoś nowego.

To co jest zawarte w pojęciu przelewa się do sądów analitycznych. Wg hume i empirystół są tylko sądy analityczne ( z aprioryczne ) i sądy systnetyczne ( postrori )

[Narysował macież, na pionowej osi ma a,s ( apiroryczne i systetyczne ) oraz na poziomej ap, i apost ). Hume i inne dają że istniej tylko (1,1), (1,2) , (2,2) od lewego górnego rogu. Cant dodaje (2,1) ]

Metafizyka jest wiedzą słaba z definicji. Metafizyka jest zwalczana przez filozofi. Metafizyka to bardzo ogólne sądy na to co może istnieć.

Nie może istnieć metafizyka, która jest kompatybilna rozumowym zbieraniem informacji. Przez co teologia tym bardziej odpada ( jest jeszcze gorsze ).  To było na czasie (nie wiem czemu, ale było )

Spory o to czy może istnieć teologia naturalna ( taka w której działa tylko rozum ). W oświceniu została poddana religia również. Dlatego pojawili się Deiści, ludzie wierzący że coś jest, ale nie ma to realnego wpływu na ich życie. Kant pisał przed rewolucją, ale się ładnie wpisywał w lutry rewolucjonistów.

Cant doszedł do tego iż nie może istnieć religia naturalna. Cant nie negował Boga, ale zmieniał jego niszę ekologiczną ( z metafizyki, czy filozofii ale etyki ). Cant przeszedł przez 5 faz. Ostatni ( krytyczny ) wchodzi dużo i wiele.

Tego potrzebne jest apriori nie tylko doświadczinee by opisywać sensownie materiie. (Angdota)  Wszystko co wiemy bierze się z nas. Dane empiryczne wchodzą tylko przez apriori. 

Przedmiot poznania . Musimy trakować jako wrażenie przez gdzieś i kiedyś. My jesteśmy zmuszeni zakładać, że na zewnątrz jest coś. Ale to założenie nie wychodz poza nas.
Uważamy że na zewnątrz nas są rzeczy same w sobie. Ale możęmy je poznawać tylko prez zjawiska ( fenomeny ). 

Kim jest kant ? Jest idealistą subiektywny transcendentalnym. Wysyła nas na zewnątrz. Co jest inne niż u huma. Świat zewnętrzny jest niepoznawalny na gruncie teorii poznania, lecz pozwala na to rozum praktyczny ( z racji tego że my jesteśmy rzeczami ).

Jesłi bylibyśmy zdeterminowanym musileibyśmy być na tem konstrukcji. Będzie chciał przejći doc zegoś, nie wiem do czego, ale chyba do etyki.

Teraz będzie o "Krytyce czystego rozmu", i nie wiem co...
Kantyści dalej istnieją i nawet się mają dobrze. Nie traktujmy tego co ten cżłowiek powiedział jako $\alpha$ i $\Omega$

Jak to jest z matematyką.. Matematyka zinterpretowana empiryczna nie jest empiryczna. Matematyka czyta może być w zbudowana tylko analitycznie. 

[Ale dlaczego ja go nie rozumiem ?] 

[Przerwa]
[20s później]
[Po przerwie]

\subsection{Etyka}
Co wolno a czego nie wolno. W czasach Kanta było pewnikiem, że w systemie filozoficznym, musi być etyka.

Co naj dajnomion mówi. Kant wiedział co i jak ma być. Widział, że czarnuchy wpierdalają ludzi na ofiarę. Musi być bezwarunkowe i konieczne dla wszystkich.

Co mówi prawo moralne ? Że dobre jest postępowanie takie, że ten kto to mówi chciałby by było powszechne. Ale Kant mówi to że jest kwestia społeczeńśtwa, czasów i kultury. I to jest aprioryczne.

Kant wypowiada się na poziomie Metaetyki a nie etyki ( prawie jak w Pythonie ). Czyli omawia o możliwościach poglądów a nie o konkretynch poglądach. Etyka w takim pojęciu jest aprioryczna. I jest smaczne...

Co jak wpływa na inne jaja. I jak można kończyć mechanizmy i inne cuda. 

Skąd my to wiemy ? Z apriori bo jesteśmy ukształtowani. Ake czy musimy stosować ten konstrukcji. Tutaj trzeba przywołać istotę najwyższą ( ale nie prawodawcy czy sędziego ). Przestawia go jako odpowiedz na pytanie "Dlaczego my mamy postępować etycznie a nie inaczej?".

"Czy naprawdę Bóg istnieje ?" Faktycznie nie doi końca. Faktycznie nie musi być problemu. Trzeba myśleć o sprawach etycznych jakby gonie było. Kant nie mówi że boga nie ma, ale twierdzi, że nie sposób przestawić mechanizmów że Bóg jest. Trzeba użyć metafizyki a metafizyka jest zła.

Równolegle do Teorii poznania i Etyki ( Estetyki )

Krytyka konstrukcji :P

Następca Kanta jest Hegl a potem będzie Marks 

\section{Hegiel}
(tutaj cześć na której mnie nie było)

Sfera ducha subiektywnego. Potem neguje wszystko, wciela się w przyrodę, gdzie wszystko jest uporządkowane, deterministyczne. I tam jak człowiek wynika z przyrowdy, tak z negacji negacji ( prawo podówjnego przeczenia nie działa ). W terminologi ducha hegla, gdy duch przechodzi przez zaprzeczenia zaprzeczenia dostają się stan obiektywny. Co wiecej nie zależy od przyrody. Temu czemuś przysługuje obiektwyizm, najwyższa prawda.

Coś obiektywnego w terminologi Hegla, posiada status czystego pojęcia. ( analogia do platona ). 

Rozumność poszukiwana przez Oświeceniowców w utopiach i sytemach społecznych, jest czymś oczywistym. I tak działa. Duch wszechświatowy w pewnym momencie nakazuje, że właśnie status odnoszący się do tego, to sie wciela w różne systemy filozoficzne, religijne przez twórców religii. 

Duch wszechświatowy z wszechpojęcia towrzy subiektowność ludzką potem przyrodę, a na samym końcu dochodzi do tego co Hegel nazywa sferą ducha absolutonego. Gdzie jest wszystko.
Analogia do pięknej niewisty. Duch wszechświatowy w pewnych realizacjach wchodzi gdzieś, w systemach politycznych i innych. 

Głowny zainteresowanie Hegla to rózne systemy polityczne. Które rónie realizują rozumność, ale  na różny sposób

Ustroje oligarchyncze - grupa ludzi rządi innymi, potem przechodzi w społeczeństwie obywatelskie. Zgodnie z systemami wartości, etycznych, moralnych czy prawnych. Hegiel staje okoniem względem teoretyków umowy społecznej. Zdaniem Hegla państwo nie jest czymś zależnym od jednostem. Jest czymś koniecznym ( w sensie historycznym ) oraz jest nadrzędna względem jednostem ( jest ważnego ). 

Hegiel trakował monarchię pruską, jako wcielenie umysłu i tam powinno być, jednak obecnie uważa się że bliżej mu było do do monarchi konstytucyjnej. Wyłamują się konstruktu, romumność się zmienia, nie ma charakteru statycznego ( a fuj statyczność ). Zdarzali się persony zwane przez Hegla kamerdynerami, którzy pchali historię na odpowiednie tory ( Aleksander Wielki, Napoleon )

Jak już zrozumieliśmy prawa historii możemy je świadomie wykorzystać. Tak by było dobrze.

Dwa aforyzmy z okresu Hegla

\bigskip
Co jest rzeczywiste , jest rozumne. - na zasadzie granicy, że dąży to co inaczej\\


Co jest rozumne jest(będzie) rzeczywiste. - jeżeli duch wszechświatowy ma jakieś plany to na pewno je zrealizuje. Jakkolwiek by to dziwne nie było. Najpierw sukces, potem porażka czyli zaprzeczenie, a potem znowu sukces czyli zaprzeczenie zaprzeczenia ( Na przykładzie stanu wojennego ). 

Jest o inne podejście niż filozfii francuskiej, że myślenie jest statyczne, że raz zaprowadzony rozumny porządek tak pozostanie. Hegel stoi w opozycji i nie ma przebacz. Duch wszechświatowy, przez swoich ambasadorów ( kamerdynerów historii )  którzy kierują tak by się zbliżyć do optymalnego porządku.

Nauka jest kulturowa, to nie jest własność pojedynczej istoty ( odkrycie ) ale należy dóbr absolutnych.

Historyczne znacznie wydarzeń w których uczestniczmymy ich oddźwięk jest poza horyzontem uczestników ( nawet przywódców ). Coś widzą, wielcy kamerdynerzy ( a co to gradacja karerdynerów ? ), mniej mali, a bierni uczestnicy gówno widzą.

Odpowiednio interpretuje w charakterze sensu. Pewnej logiki która nią rządzi. 

Czy tą logikę można głębić, poruszamy się w dziedzinie ducha subiektywnego. Ten duch może dojść do sensu tylko jako cząstka ducha obiektwynego ( myślenie zorganizowane).

\section{Marks i Engels}

Marks postawił system Hegla z głowy na nogi.
Hasło "Proletariusze wszystkich kontynentów łączcie się" 

Cytat pochodzi z manifestu komunistycznego, który ta para tylko redagowała, za zamówienie I między narodów ki. 

To jest myślenie młodego Marksa ( która jest różna od Marksa dojrzałego ). Engels był starym kawalerem. Marks miał żonę ( arystokratka ). Wyprodukowali dużo dzieł razem. Co jest dość dziwne gdyż filozofowie byli ( są ? ) indywidualistami. 

"Nie świadomość ludzi określa byt, ale byt określa świadomość ludzi". 

Jeżeli nie są wolni to siedzą zniewoleni, jeżeli są woli ( z natury rzeczy ) to walczą.

Ludzie czegoś tam chcą, ale efekt dążeń, nie jest tym czego oczekiwali ( i tutaj taki wielki trollface historii ).

Mamy dwa przypadki. Robotnicy w Rosji II XXw chcieli poprawić socjalizm, to socjalizm obalili.

Otóż zadaniem Hegla oraz Marksa i Engelsa historia jest logiem naszych obiektywnych dążeń, ale one są zupełnie inne niż nasz subiektywne.

Cytat " Przedmowie do przyczynku krytyki " - 15 zdań będących esencją Marksa i Engelsa.

Czy w życiu społecznym ludzie dążą do by siły wytrówrcze się rzwijały ? Spór rozwiązany.

Są warunki które sprzyjają rozwojowi siły ( wytwórczej ) inne mniej pomagają siły. W krajach byłego związku radzieckiego próbowano wprowadzać komputeryzację. Ale gówno z tego wyszło, bo system nie wytrzymała swobodnego przepływu informacji.

"Jak maszyna parowa rozwaliła feudalizm, tak mikroprocesor rozwali realny socjalizm" 

Procesy pomagająca rozwojowi siły, np etyka protestancka. Jednak w innych okolicznościach. Przykład korei północnej w latach 60. Albo Królestwo polskie w czasie uwłaszczenie chłopów. 

Dzielimy na dwie części, materialna, zobiektywizowana, oraz to co nie jest (dopełnienie).

Teraz schemat - dół siła wytwórcza - góra nadbudowa
Nadbudowa dzieli się na to co jest obiektywne oraz to co jest subiektywne. 

Standardowa interpretacja materializmu historycznemu mówi o rewolucji, gdy nadbudowa musi zmienić konstrukcji.

Nasze oddziaływanie może trafiać na coś takiego jak bariery ekologiczne. Co powoduje, że pod siłami musimy dorysować jeszcze siłę. Siła, wcinaj suple.

Epoka burżuazyjna jest koniec prehistorii. Ukąszenie Heglowskie.
\end{document}
