\documentclass[11pt]{article}
\usepackage[utf8]{inputenc}
%\usepackage[T1]{fontenc}
\usepackage{amssymb}
\usepackage{amsmath}
\usepackage{enumerate}
\usepackage{fullpage}
\usepackage{polski}  
\usepackage{indentfirst} 
\usepackage[pdftex]{graphicx}
\usepackage{multirow}
\usepackage{placeins}

\author{Łukasz Dubiel}
\title{Filozofia - po raz n-ty }

\begin{document}

\maketitle

Na dole przyroda, potem ludzkość. To ma być głownie siły "wytwórcze". W obrembie sił wytwórczej. Cześć hardwareowa ( materia nieożywiona, komputery czy lokomotywy) i softwareowe ( ludzie ). One odziałują na siebie maszyny są takie jak zaprojektują ludzie, ludzi determinują maszyny z których korzystają.

Teraz najwyższy konstrukcja, znowu podzileone przezna dwie cześci. Wżej są wszelako rozumiane systemi ideei ( politycznych i globalnych ) odziaływania w górę są mocniejsze. Zależności w górę są mocniejsze (przyroda na technologia a ona na ludzi) Nie jest potrzebne jak w oświeceniu klasycznym, że trzeba mieć baterue acc 

{Materializm historyczny }
Reformafja zrodziła kapitalizm .

\end{document}
