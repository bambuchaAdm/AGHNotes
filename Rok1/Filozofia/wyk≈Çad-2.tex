\documentclass[11pt]{article}
\usepackage[utf8]{inputenc}
%\usepackage[T1]{fontenc}
\usepackage{amssymb}
\usepackage{amsmath}
\usepackage{enumerate}
\usepackage{fullpage}
\usepackage{polski}  
\usepackage{indentfirst} 
\usepackage[pdftex]{graphicx}
\usepackage{multirow}
\usepackage{placeins}

\author{Łukasz Dubiel}
\title{Filozofia - nowożytność \\ rozpoczęcie pierdolenia}

\begin{document}
\maketitle

\section{Tytułem wstępu - czyli jak można przez pół godziny robić dygresję do poprzedniego wykładu}
Tomizm -nauczany w KULu wszystkiego.
Około XX wieku rzucono hasło powrotu do tomasza.

Obecnie mamy ( XXI w ) neotomizm.
Kraków - wybitny neotomista ( nie padło nazwisko ).

KUL - zakuci neotomiści. 

Finał dygresji ( o ja zajebiście ).
W średniowieczu tomizm był uprzywilejowany. Na spółę z scholastykę. Ramy w poruszaniu się ( łącznie z krytyką ).
\section{Zasługi Tomasza}
Wprowadzenie dorobku Arytstotelesa i programu arystotelików do filozofi europerjskiej.

\section{Spół o uniwersja ( powszechniki )}
Odpowiednik formy u arystotelików klasycznych.

Storna konceptualistyczna. Engels w jednym swoim tesktu, konceptualizm jest pierwszym przejawem materializmu.

Czołowym komentator - aberores - materialistyczna interpretacja arystotelizm. Interpretacja formy jako części materii. Nie przypisywał składnikowi formalnego żadnych dodatkowych rzeczy.

Lata 1620-1630.
Panowie Francieszek Beakown( Bejkon nie wiem jak się pisze ) oraz Kartezjusz ( wiadome o kogo chodzi z matematyki, dokładnie geometria analityczna) klucz 1632 r. rozprawa o Metodzie ( Kartezjusz )

Dla większości europy poczatek nowożytności to odkrycie Ameryki, a w krajach protestanckich jest to rozpoczęcie reformacji ( wywieszenie tez Lutra na drzwiach ). W Krakowie proklamowano państwo które miało pierwsa wiarę urzędową, protestantyzm.

Reformacja powoduje upadek autorytetu papiestwa i innych dziwnych rzczach ( usunięcie filozofi scholatycznej z życia polskiego ). 

Aluzja o rokładzie funcji pracowitości od wiary ( protestańci największe , prawosłami najmniejsze )

\section{Zasada solafide ( nie wiem o co chodzi )}
W liście Pawła do rzymian, jest wspomnienie o niskiej wadze uczynków. Liczy się tylko wiara. Tak jest w protestantyźmie.

Zupełnie inny mechanizmy niż w katolicyźmie. W protestantyźmie trzeba się wystrzegać grzechu. tylko wiara prowadzi do zbawienia. 

Wiara jest najwięcej warta niż wartość poznawcza. Podobnie jak u Agustyna, wiara jest darem od boga i nie używał mózgu ( jako cześć ludzkiego ), opozycja do Tomasza, który uważał, że mózg pozwala dostać się do przedsiąków wiary.

Toria predyscynajcji ( deterministyczne określenie dziejów człowieka ) - wybór przez Boga celu istnienia człowieka.

Człowiek może sobie zapracować na zgawiawienie ( to jest myśl Tomasza). Rozum może do Boga, może nie doprowadzić, ale przynajmniej podprowadzić.

Prostestantyzm przesuwał całość na stronę Boga, zrzucając z człowieka odpowiedzialność ( Bóg robi git pull)

Przechodząc na metodologię marksistowską, kościół w średniwiecznej europie opierał się na fedualiźmie ). Potem gdy dochodziło do pojawiania się burżuazji, potrzebującej wyzwolenia do fedualizmu.

Postulat ubogiego kościoła.
Prawdziwe poznanie boga i nie wiem czego tam nie wiem.

Bardziej zwierzyć boga i typ poznania do boga.

W wielkiej mierze interesowała się odkryciami fizyki czy biologi, o ile dokładały się do tworzenia przedsiąka wiary ( prowadzące do boga ).

Sola scriptum - ścisłe trzymania się litery pisma świętego. Nie przyjmowania pism Tomasza i Augustyna ( uznanie tradycji za szkodliwą ). Autorytety religijne, zawdzięczającego autorytet czemuś innemu niż pismo święte jest nie warty funta kłaków. Znacznie filozofi się zmienił. Reformacja i humanizm, spoowdował laicyzaję filozofię, jako że kościół protestancki nie potrzebował już filozofi.

Filozofia się rozwija poza strukturami oficjalnymi ( kościół, uniwersytety ). To już nie byli ludzie frontu doktyrnowego. Doraźnie oznacza to zmniejszenie wpływy na masy, gdyż filozofię uprawiania tylko Ci co chcieli.

Czasy nowożytne to okres zerwania z starymi autorytetami. Tomasz i spółka tracą na znaczeniu. Filozofowie muszą walczyć o swoje. Nawet o im się udaje, w decydującej wielkości w krajach protestanckich.

Największy okres rozkwitu filozofii w nowożytnej europie to Wielka brytania ( anglia ) , Francja, Holadndia, protestancka cześć Niemiec.

sytuacja nie dotyczy czyści prawosławnej.

Francieszek Baykon - filozofia poznania - podjerzany o łapownictwo. Wraz z kartezjuszem ( empiryzm, to samo). Bykon był epigonem, towórca metody indukcyjno-poznawczej.

Odejście na rzecz, hipotetyczno-dedukcyjnej.

Doświadczenie ludzi myli...


Badania empiryczne muszą być poddane kontroli ( krytyka poznania ). Dzieło Nonum organum, mówi że doświadczenie i poznanie nie może być zaburzone przez idoli ( uwarunkowania jednostkowe czy jakieś inne )

Idole jaskini - osobiste doświadczenia człowieka.

Idole teatru - ciśnienie w naukach społecznych

Oczyszczenie poznania od mechanizmów zaburzenia...

Noż ja pierdole ile można pierdolić... Głody jestem zjadłbym coś dobrego... Jakieś mięso albo cos podobnego.. W ojczyźnie bejkona, no może być bekon, został ktoś tam ścięty. Jakcyś pozytywiści nie wiem o co chodzi ( Wokulski ? ) no nie wiem. `Jak już poiwedziałem` nie wiem co powiedziałeś, bo pierdolsz już tak dłuższą chwilę. `to nie ma być indukcja` jak nie indukcja to nie wiem. Wiecie co wyraża sobą całka ? Nie jestem taka łatwa na jaką wyglądam. `Teraz powiem klika słów o dalszych losach empiryzmu angielskego` to jak to to nie umarło ? Jakić Locke ? Jebazni materialistczni empiryści. Jkaiś hobbes ? A teraz ich młodsi koledzy Berkley, to chyba jak ta baza. Oraz Hume to też było chyba ale naporawdę nie wiem materialisci impremestyczni. `Podzielił materiał na doświadczenie człowieka, na idee i impresje` no jakie to impresje są po alkocholu to nie wiem. `Podział na elemetarne bodźce poznawszcze`, które pozwstają przez odpowiednią ilość pomergowanych konstrukcji... Jakości zmysłowe, no nie wiem... 
`Lock powiedział, że pewne wartości mają sowoje odpowiedniki na zewnątrz, a inne są tylko nasz, np. smak` czyli kazdy jest inny. `Jak rozróżniamy wartości pierwotne od twórnyuch`, ale po jaką cholerę go pisać. O nagios się zaaktualizował... Nie wie o co chodzi... Nie nie wiem do czego zmierzeasz... 
Dobra teraz już wiem. Podział na idealizm subiektywne. Najbardziej paradoksalnym efektem rozumowania było znane twierdzenia Berklaya ( gdyby nie Bóg, mielibyśmy podstawy że istniejemy tylko my ). No to fajnie  ? A gdzie małe drobne skrzydlate świnki ? Ale zajebisty rysunek... Słoność. Morał lema a berkleya.
Myśl można znaleźć ją u sceptyków starożytnych. inny dostęp do pana boga. Solipsyzm, Każdy mógłby stwierdzić że jest jedyny na świecie a reszta to nasza wyobraźnia.
My nie doświadczamy przyczynowości.

Wielokrotny powrót, jako neopozytywizm. 
 

\end{document}
