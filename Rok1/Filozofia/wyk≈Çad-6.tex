\documentclass[11pt]{article}
\usepackage[utf8]{inputenc}
%\usepackage[T1]{fontenc}
\usepackage{amssymb}
\usepackage{amsmath}
\usepackage{enumerate}
\usepackage{fullpage}
\usepackage{polski}  
\usepackage{indentfirst} 
\usepackage[pdftex]{graphicx}
\usepackage{multirow}
\usepackage{placeins}

\author{Łukasz Dubiel}
\title{Filozofia \\ Dziwnego wykładu ciąg dalszy}

\begin{document}

\maketitle

\section{Marksizm}

"Świat jest zbiorem faktów, nie rzeczy." L. Wittgenstein
\\

Wzmocnienie przykładu pokrewnym nowym, z okresu 19 i 20 wieku ( może nawet 20/21 ). Liberalizm. Liberałownie mówią, dużo o wolności oraz o sprawiedliwości. Wolność to ma być przedewszystkim wolność gospodarcza. Na wolnym rynku panują prawa dżungli. Karol Darwin zrobił na kapitaliźmie reverse engenering i wykryte prawnie.
Bogaty jest słusznie bogaty, biedny jest słusznie biedny. Pomoc społeczna jest nie dość że spacza to jest niesłuszna, bo powoduje niezowlanie.

Kiedy musi się wyłonić klasa kapitalistów, przedsiębiorców. I teraz oni muszą być przekonani, że chcą się dorobić. Muszą uważać, to za słuszne i inne takie. Gdzie jest wygrany tam musi być przegrany. Oni, przegrani, muszą wiedzieć, że chcą i to jest słuszne, bo są słabi. Nie jest to skutkiem niesprawiedliwości społecznej czy skrzywdzenie przez historii. 

Punkt widzenia, zależy do punktu siedzenia. I teraz gdy przywódcy solidarności (okres '89) teraz są biznesmanami czy siedzą u władzy. Teraz dla nich zwiększanie świadczeń socjalnych, jest sprzeczne z ich celami. Jest to normalne, gdyż oni awansowali z klasy robotniczej, do burżuazji czyli rządzącej.

Taki układ (możliwość awansu) umożliwia konstrukcję stabilnego społeczeństwa opartego na hierarchie, gdyż problematyczne jednostki w warstwie niższej są promowane (lub mają taką możliwość) do klasy wyższej, gdzie nie darzą do buntu. Dzięki temu masa, nie ma przywódców więc nic nie zrobi.

\section{Pozytywizm}

W okresie gdy nauka dostarczała ważnych danych dla filozofie, zdarzyli się ludzie którzy bardziej wierzyli w fakty. Odpowiednio podczas omawiania poszczególnych szkół filozoficznych, zaobserwowaliśmy, że filozofowie uważają różne rzeczy jako źródło poznania. Jedni opierali się o empirii, inni na kontemplacji (Pitagorejczycy i spółka).

Obecnie mamy dwa fronty. Filozofie opierającą się na twardej nauce (matematyka,fizyka) i używającego twardej logiki oraz wyrażeń. Druga strona używa konstruktów bardziej miękich. Nie lubią używać faktów, ale zazwyczaj nie wiem czego. Do pierwszego grona wchodzi Marks z Engelsem, którzy mówili, że podnieśli socjalizm z utopi do nauki. Engels przy końcu życia stwierdził, że nie będzie potrzebna filozofia, lecz nauki społeczne. 

Z punku człowieka lubiącego brać wzór z nauki, będą się wzbraniać, przed stwierdzenie, że w naukach społecznych używane są fakty. Bardzie będą iść w stronę stwierdzania, że jest to metafizyka. 

To jest coś nie zowiazanego z konstrukcją i innymi cudami. Nie wiem co tera pisze, bo urawał mi się wątek i czekam na synchronizację. 

Nawet w starożytności a nawet średniowieczeu, można było się doszukąć elementy pozytizmu. Czyli ograniczenie zainteresowania filozofów do obszarów twardej nauki. Hume na spółę z d'Alamberte stwierdzili, że filozofia powinna się zbierać tylko do tego co wyprodukują twardą naukę.

Teraz A.Comte uczeń Sensimone'a (uważającego że społeczeństwem powinni rządzić posiadających.) rozszerzył kontrukcję podzielił budowanie ludzkości na czy częśći. 3 Pozytywistyczna. tutaj poszła psychoanaliza. Nauka jest na tyle skuteczna, że może  popanować umysły, że ludziom, nie jest potrzebne ani mity ani metafizyka. 

Skoro metafizyka, to klasyczne podejście do filozofi jest do bani (chrzanu). Comte w starszym wieku (zdaniem niektórych znaków) uważał, że da sie stworzyć, pozytywną religię (bazującą na naukach ścisłych). Autor "Kursu filozofii pozytywnej". Wiekszość ludzi uważa, że naukę powinno się brać w pozytywny sposób. 

Metoda indukcyjna ( Beaykon ) jest przez Comta reanimowana. Pozytywizm formułowany przez Comta jest bardzo podobna do scientyzmu. Czyli przeświadczenia, że nauka jest czymś co rozwiąże wszystkie problemy, oraz pozwoli odesłać na śmietnik wszystko co jest nienaukowe, czyli metafizyczne, spekulatywne rozważania. 

Dziej nie ma wielu ludzi, którzy nazwali by się pozytywistami, ale wielu jest. Opierają się na faktach, i gównych. Należy korzystać tylko z tego i odsiać spekulatywne śmiecie (metafizyka) jak śmiecie. Metafizyka jest źródłem eksterminizmu, terroryzmu i innych dziwnych rzeczach. Nawet nie warto na nie marnować farby drukarskiej.

Pzytywizm w Polsce odnosi się do pracy u podstaw. Narodził się w po upadku postania stycznowego. W Polsce obie znaczenia się splotły ( takie działanie matematyczne ). Przed '89 stwierdzenie o Pozytywiśmie czhodziło o nie buntowanie się i pracę u postaw.

Leszek Kołakowski w jednej w swoich czesnych prac, przekonuje, że również w swojej XX w. wydaniu (neopozytiwizm) jest to nurt  czysto technokratyczny powodujący odejście (zmanipululowane) od jakiekolwiek ideologi (aideologiczne) podejście do świata.

Comte ukuł stwierdzenie socjologia, chociaż o żadnej socjologi nie można było mówić. On ukuł piramidę nauk, gdzie fizyka leży na dole ( jako najważniejsza ) przez biologię czy chemię, i kończy się na socjologii. Według Comte socjologia ma być nauką bardzo twardą, opierającą się na fatach.

Behawioryzm zajmuje się nie wewnętrznym stanie człowieka, tylko ma działać na zasadzie badania reakcji na bodźce. Również psychologia chciała tak zrobić, ale nie da się tego zrobić. Była to pozytywan próba (od pozytywizmu ) w tych dziedzinach.

Drugi pozytywizm to był empirokrytyzm. Chodziło by nie oddzielać przedmiotu od podmiotu. Jest to podobieństow do Berkley i Hume'a. Lecz empirokrtycy uważali, że podział na wewnętrzno/zewnętrzne jest nie na miejscu

Neopozytiwizm, koło wiedeńskie, podjęło próbę zrekonstrułowania wiedzy w oparciu tylko na faktach. Kiedy uzna się za fakty to samo co Berklay i Hume, czyli stane atomowe, elementarne naszego doświadczenia, i przypatrzymy jaką strukturę mam nasza wiedza budowana na nich oraz używając narzędzi logiki, a tym bardziej używając praw nauki, możemy oddzielić te plewy ( metafizykę ) od całości. Dlaczego ? Bo to co nie można wyprowadzić z podstawowych predykatów i działa, to jest to metafizyka i jest do chrzanu (czy bani) 

Kryterium sensowności to sprowadzenie wyrażenia do konstrukcji postawowych proedykatół konstrułujących dane wyrażenie. 

Trzeba formułować jak największą ilość hipotez i próbować je falsyfikować (próbować udowodnić jest nieprawdziwość), lub jeśli wiemy jak moglibyśmy ją sfalsyfikować to przyjmujemy ją na naszą wiedzę do czasy wykazania jej fałszu. 

Dzięki temu możemy stować indukcję na wyższych rzędach konstrukcji. Ale ono nie mogę dość do pewności i nie ma zmiłuj, dlatego idąc w wprost nie jest możliwe i szkoła wiedeńska się myliła.

Rola doświadczenia z perspektywy Pottera ( nie jestem pewny nazwiska) ma na celu wykazania fałszu doświadcznia.

Jeżeli jakieś teazy okazują się niefalsyfikowalne to cześć metafizyki. Metafizuka opisuje że działa. Tego rodzaju twierdzenia (metafizyczne) nie mając być źle traktowane, nie z zasady, lecz z faktu, że są niefalsyfikowalne ( obalalne ).

Teorie bydowane na takich faktach
Pseudoteroie (budowane na metafizycznych teazach) nie rozwijają gdyż nie są falsyfikowalne.

\section{Kolos}
Pytanie z o czeterach epokach, starożytnośći, średniowiecza, nowożytności oraz czasów współczesnych.

\end{document}
