\documentclass[11pt]{article}
\usepackage[utf8]{inputenc}
%\usepackage[T1]{fontenc}
\usepackage{amssymb}
\usepackage{amsmath}
\usepackage{enumerate}
\usepackage{fullpage}
\usepackage{polski}  
\usepackage{indentfirst} 
\usepackage[pdftex]{graphicx}
\usepackage{multirow}
\usepackage{placeins}

\author{Łukasz Dubiel}

\begin{document}
\title{Systemy i podstawy filozofii}

\maketitle

Pełen dystansu stosunek do filozofii. 

Dodanie faktu atmosfery higestycznej ( oczekiwania popwtórnego przyjścia jezusa, już jutro ).
Do tego filozofia i inne cuda nie jest potrzebne.

W pierwszych wiekach widzimy wpływy filozofi na ewangelię.
Przykłąd : "Na początku było słowo" ( łać. logos ) 
Dzieło nie wygląda na autorstwa Jana ( inna mętalność ). Środowiko około jezusowe było konserwatywne, lecz tekst ma wpływy hellenistyczne ( obce w środowisku apostoslkim)

Bliski związek z Artystotelowkimi widzenia istoty. Trójca święta. Widać tu naturę w sensie Arystotelowskim.

Przez to bylibyśmy politeistami gdyby nie sobór w Nicei ( plus Konstantyn ).

Dylemat chrześcijaństawa . Jak się ma Jezus do Boga starotestamentowego. Roztrzugnięcie na rzecz trynarny.

Dwa przykładu.
poczatke III i koniec IV wieku. Wyrażanie konstrukcji relacji między istotami boskimi.

W starożytności były punktowe wydarzenie, to pomoc filozofii dla chrześcijańska.

Od średniowiecza ( scholeastyki ), filozofia jest "służką" ( jest zależnością ) teologii. 

Chrześcijanie nie wiedzieli (IV w) nie wiedzieli, że filozofia im pomoże. 

Chrystanizowano na początku na obecny świat "prawosławny". Układ bardziej uduchowiony. Czerpali od neo-platoników ( plotyn i spółka ). Chrześcijanie czerpali od nich konstukcję. To co wynika z medytacji, kontemplacji i innych dziwnych wypadów w świadomość.

Chrześcijaństwo łacińskie ( czasy Teodozjujsza, koniec IV w.), gdy urzędowo wprowadzano chrześcijaństow, onie nie mogło być mistyczne, mocno uduchowione itp. 

Po przejęciu głównej roli w państwie ( szkolnictwo, tworzenie kadr), trzeba było użyć innych konstrukcji.

Trzeba zchrystanizować, to co było pogańskie. I trzeba było obrobić wszystko.

Nie ostatnie miejsce podczas tego przerabiania miała filozofia.

\section{Augustyn}
Wczesny chrześcijanin. Początkowo poganin, nawrócił się w dojrzałym wieku. Był to akt nagły. 
Chciał zerwać z wszytkim co dotychczas zajmował ( kultura Rzymska ). 
Odcięcie się do ojca.

Fakt, nawrócenia poprzedzało podsłuchanie kazania biskupa Mediolanu. Spłynęło na niego wiedza, że to co chciało było dobre.


Retor ( mówca do wynajęcia ).
Schrystanizował pisma Platona.

Prawda wybiórcza, nie metodyczna ( nie poprzez pomiary i innych ) lecz poprzez obcowanie duszy z światem ideii, platońska Anamneza ( trzeba sprawdzić pisownię ).

Dla wierzącego Augustyna poznanie przez duże P było to poznanie Boga. Dokładne i całkowite. Nazwał to iluminację ( wświetlenie ). Nazwa nawiazuje na oswietlenie duszy, gdzie wsyzstko odnośnie boga staje się jasne.

Wiedza o Bogu nie może być poznana przez badania, czy medytacja to akt wiary musi być aktem Boskim. Bóg zsyła na człowieka poznania. W sprzeczności w Platonie, w ograniczeniu poznania do jednej najwyższej istoty.

Pod słowem człowiek rozumie się tylko duszę. Ciało jest tylko dodatkiem na ten okres. Podobnie do platona jak i gnostyków wczesnochrześcijańskich. 

Co my poznajemy poznając Boga. Poznajemy idee, które powołał do istnienie Bóg. Z punktu widzenia Boga są ich plany stworzenia ( zamysł Boży ich stworzenia ). 

Poznawanie przyrody ma znaczenie i jest dobrą rzeczą, o ile wzmacnia nas w wierze.

Rozum pozostawiony sam sobie błądzi. Nie może dojść do rozwiązania. Akt wiary -> Iluminacja -> Boże plany. Skoro znamy plany znamy to co odpowiada im odpowiada.
Upraszczająć całość poznania pochodzi od Boga.

Jest jedna nieścisłość. Odegrała niemałą rolę tysiąc lat później podczas reformacji. Czy Bóg związku z tym że jest wszystkowiedzący, nie pogrywa sobie z nimi, ponieważ pewnych zbawia, a innych zsyła na potępienia.

Odłam Braci Polskich wierzy, że ludzie nie mogą zmienić faktu potępienia pewnych ludzi. 

Wybity doktor kościoła ( Anielski )

Augustyn musiał tak to rozdzielić. Wszystko zalerzy od aktu wiary ( uzyskanego od Boga ). Wyrwanie : Łaskę można odrzucić. Ale Bóg sprawia że i tak ją zaakceptujemy.

Czy człowiek jest usprawiedliwoiony ? Czy człowiek jest ustawiony na całość. Augustyn odpowiedział, kto ma wiarę ma dobre uczynki. Kto nie, nie może.

Stan inluminacji jest ciągle połaczenie z wiarą. Dobro jako etyczne, i w wszystkich innych formach. Co nie pochodzi od boga, dobre nie jest.

Apologeta - obrońca wiary. Wg nich akt wiary nie 
Biskup hopy i tam coś robił...,


Ale mis ie spać chce, nooż ja pierdole
Bardzo jest przegięty aktem wiary...
i LUMINAJCJI 
Zbinął przez obleżenie genrmanów w hipponie

\section{Pośmiertny dorobek Augustynów}
Wpływał na późniejsze działa aż do dzisiaj.
Do Tomasza jest głównym autorytetem w Chrześcijaństwie. 

\section{Trochę Łaciny}
\begin{enumerate}
\item{credo quia absurdum} \\
Tertiusz
\item{credo ut uitelligam} \\
Augustyn
\item{philosophia aneilla theologiaea} \\
Tomasz z Akwinu
\end{enumerate}

\section{Tomasz z Akwinu}
Dominikanin pochodzenia arystotelesowego . Bogacz w zakonie żebraczym ( skrajna decyzja ).

Dominikanie byli pryncypialny. Fachowcy od doktryny. 
"Czuje się jak Inkwizytor kilka wieków za późno."

To co tomasz wiąże z wiarą jest oczywiste.

brak sprzeczności między wiarą a rozumem. Podobnie jak u arystotelesa. Nie dotyczy Pryncypiów. 

nie wychodznie poza perspektywy starożytyczne. Tomasz miał wszelkie dane, by dać rozumowi wielkie uprawnienia.

Rozwijanie wiedzy rozumowej, i dochodzi do nauk szegłowej. Rozum i doświadczenie jest całością i jest. Tak więc...

Dobra filozofia, jakaś pogańska nie istnieje. Tylko teologia.

Teolog zna prawdę i mieli ją przy pomocy filozofii by było to zrozumiałe.

Tajemnica trójcy świętej wg filozofa i teologa.
Rozum nie jest nie omylny. Rozum mimo omylności może się przydać sprawach ważnych ( telogicznych ).

[Dwa zbiory preciete, fizofia i tajemince wairy]
Naturalne teologia = przecięcie filozofii oraz tajemnic wiary.

\subsection{Pięć dowodów na istnienie boga}
Cześć filozfii naturalnej. Preambuły wiary.

Mają wierzącemu pomoc w życiu.
Mają uprawdopodobnić istnienie Boga, nie udowodnić ich istnienie.

Skrajne naciąganie.

Wiara nie jest sprzeczna z rozumem. Mogą działa razem. Nie musi być odrzucony.
\end{document}