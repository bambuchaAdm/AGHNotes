\documentclass{article}
\usepackage[utf8]{inputenc}
%\usepackage[T1]{fontenc}
\usepackage{amssymb}
\usepackage{amsmath}
\usepackage{enumerate}
\usepackage{fullpage}
\usepackage{polski}  
\usepackage{indentfirst} 
\usepackage[pdftex]{graphicx}
\usepackage{multirow}
\usepackage{placeins} 

\author{Łukasz Dubiel}
\title{Matematyka - Ćwiczenia \\ część 5}

\begin{document}
\maketitle

\section{Zadanie 2}
\subsection{j}
$$a_n = \sqrt[n]{\frac{2}{3}^n + \frac{1}{2}^n + \frac{1}{5}^n}$$
Używamy Twierdzenia o trzech ciągach gdzie, $ b_n = \sqrt[n]{(\frac{2}{3})^n}$ oraz $c_n = \sqrt{3(\frac{2}{3})^n}$ . 
Dowód trywialny...
\subsection{k}
Wykorzystamy kryterium d'Alemberta
$$\lim_{n \to \infty}{\frac{2^{n+1}}{(n+1)!} \frac{n!}{2^n}} = \lim{n \to \infty}{\frac{2}{n+1}} = 0 $$
\subsection{n}
Z twierdzenia o tzech ciągach, gdzie $$ b_n = \sum_{i=1}^{n}{\frac{1}{\sqrt{n^2}}} $$
oraz $$ c_n = \sum_{i=1}^{n}{\frac{1}{\sqrt{n^2 + n}}} $$
Tak wieć $$b_n \leq a_n \leq c_n$$
\subsection{p}
Stosujemy wzory na sumę ciągu geometrycznego
%$$ a_n = \frac{\sun_{i=1}^{n}{(\frac{1}{2})^i}}{\sun_{i=1}^{n}{(\frac{1}{3})^i}}$$

\subsection{u}
$$ a_n = \frac{\sum_{i=1}^{n}{2i-1}}{n+1}$$
Po przekształceniach
$$ a_n = \frac{-n}{n+1} $$
$$ \lim_{n \to \infty}{a_n} = -1 $$
\subsection{w}
Z kryterium d'Alamberta
$$a_n = \frac{(2n)!}{n^{2n}} $$
Kryterium zbieżności
$$\lim_{n \to \infty}{\frac{a_{n+1}}{a_n}} = \lim_{n \to \infty}{\frac{2(2n +1)}{n+1} \left( \frac{n}{n+1} \right) ^{2n}} = \lim_{n \to \infty}{\frac{4 + \frac{1}{n}}{1 + \frac{1}{n}} \left[ \left( 1+\frac{1}{n} \right) ^n \right] ^{-2}} = 4 * e^{-2} = \left(\frac{2}{e}\right)^2 < 1$$
Przez to 
$$\lim_{n \to \infty}{a_n} = 0 $$
\section{4}
TW ( o zbieżności ciągów ograniczonych i monotonicznych )
\bigskip

Z: $ (a_n) \subset R $ jest niemaljeący i ograniczony z góry to
$$ \lim_{a_n \to \infty} = \sup \{a_n : n \in N \} $$

\subsection{a}
$$ a_{n+1} > a_n $$
Stosujemy podstawienie $ x = a_n  $ i rozważamy
$$ \sqrt{2 + x} > x $$
Co sprowadza się do
$$ x^2 - x - 2 > 0 $$
$$ x \in {0,2} $$
$$ x+2 \in { 2, 4 } $$
$$ \sqrt{x+2} \in ( \sqrt{2}, 2  ) $$

$$ a_{n+1}^2 = 2 + a_n $$
$$ \sup { a_n } = 2 $$
$$ \lim_{ n \to \infty }{a_n} = 2 $$

\newpage
\section{Przydatne granice funkcji}
$$ \lim_{x \to 0}{\frac{\sin{x}}{x}} = 1 $$
$$ \lim_{x \to 0}{\frac{\exp{x} - 1}{x}} = 1 $$
$$ \lim_{x \to 0}{\frac{\tan{x}}{x}} = 1 $$
$$ \lim_{x \to 0}{\frac{ \ln{ 1+x }}{ x }} = 1 $$
$$ \lim_{x \to 0}{\frac{ \cos{x} - 1}{x}} = 1 $$
$$ \lim_{x \to 0}{\cos {\frac{1}{x}}} \in \o $$
$$ \lim_{x \to 0}{ \left( 1 + x \right) ^\frac{1}{n}} = e $$
$$\lim_{x \to 0}{\frac{\arctan{x}}{x}} = 1 $$
$$\lim_{x \to 0}{\frac{\arcsin{x}}{x}} = 1 $$
$$\lim_{x \to 0}{\frac{a^x-1}{x}} = \ln{a} $$
$$\lim_{x \to 0}{\frac{\log_{a}(1+x)}{x}} = \log_{a}{e} \ \ \  \forall{a>0} $$

\section{Zadanie 1 - zestaw 3}
\subsection{b}
$$\lim_{x \to 0}{\frac{\sqrt[3]{1+x} - \sqrt[3]{1-x}}{x}} $$
Rozszerzająć do sześcianów w liczniku odejmujemy i powinno śmignąc. Czyli
$$\lim_{x \to 0}{\frac{1+x - 1 + x}{x (\sqrt[3]{(1-x)^2} +
\sqrt[3]{1-x^2} + \sqrt[3]{(1-x)^2})}}$$

$$\lim_{x \to 0}{\frac{2}{\sqrt[3]{(1-x)^2} +
\sqrt[3]{1-x^2} + \sqrt[3]{(1-x)^2}}} = \frac{2}{3}$$

\subsection{c}
$$\lim_{x \to 0}{\frac{\sin{3x}}{\sin{5x}}} = \lim_{x \to 0}{\frac{3x \frac{\sin{3x}}{3x}}{5x \frac{\sin{5x}}{5x}}} = \frac{3}{5} $$
\subsection{d}
$$\lim_{x \to \infty}{ \left( \frac{x-1}{x+3} \right) ^{x+2} }$$ 
$$\lim_{x \to \infty}{ \left( \frac{x+3-4}{x+3} \right) ^{x+2} } = 
\lim_{x \to \infty}{ \left( 1 + \frac{-4}{x+3} \right) ^{x+2} }$$
$$\lim_{x \to \infty}{\left[ \left( 1 + \frac{-4}{x+3} \right)^{\frac{x+3}{-4}} \right] ^{\frac{4(x+2)}{x+3}} } $$ 
$$\lim_{x \to \infty}{e^{-4\frac{x+3}{x+2}}} = e^{-4}$$
\subsection{e}
$$\lim_{x \to 0}{\frac{1-e^x}{\tan{x}}} = \lim_{x \to 0}{\frac{-2x\frac{1-e^{2x}}{2x}}{x\frac{\tan{x}}{x}}} = -2 $$
\subsection{f}
$$\lim_{x \to 5}{\frac{x^3 - 125}{x^2 - 25}} = \lim_{x \to 5}{\frac{(x-5)(x^2 + 5x + 25)}{(x-5)(x+5)}} = 7,5  $$

\subsection{g}
$$\lim_{x \to 2}{\frac{x^3 - 6x^2 + 12x - 8}{x^2 - 4}} $$
$$\lim_{x \to 2}{\frac{(x^3 - 2^3) - 6x(x - 2) }{(x - 2)(x+2)}} $$
$$\lim_{x \to 2}{\frac{(x-2)^2 }{(x+2)}} = 0 $$

\subsection{h}
$$\lim_{x \to 0}{\frac{\tan{x}}{4x}} = \frac{1}{4}\lim_{x \to 0}{\frac{\tan{x}}{x}} = \frac{1}{4} $$

\subsection{i}
$$\lim_{x \to 0}{\frac{\sqrt{x^2+1} - 1}{\sqrt{x^2 + 25} -5}} $$
$$\lim_{x \to 0}{\frac{\sqrt{x^2+1} - 1}{\sqrt{x^2 + 25} -5}}\frac{\sqrt{x^2+1} + 1}{\sqrt{x^2+1} + 1} \frac{\sqrt{x^2+25} + 5}{\sqrt{x^2+25} + 5} $$
$$\lim_{x \to 0}{\frac{\sqrt{x^2+25} + 5}{\sqrt{x^2+1} + 1}} $$

\subsection{j}
$$\lim_{x \to \frac{\pi}{2}}
{\frac{\cos^5{5x}\cos^{17}{17x}}{\cos^9{9x}\cos^{13}{13x}}} $$
Podstawiając bezpośrednio dostajemy symbol nieoznaczony, dlatego trzeba zastosować sztuczkę. Podstawmy takie coś, $ x = \frac{\pi}{2} -t $ oraz zmieńmy zmienną przy której chcemy robić granicę.
$$\lim_{t \to 0}
{\frac{\cos^5{5(\frac{\pi}{2} -t)}\cos^{17}{17(\frac{\pi}{2} -t)}}{\cos^9{9(\frac{\pi}{2} -t)}\cos^{13}{13(\frac{\pi}{2} -t)}}} $$
Wymnażając wnętrza cosównów, stosując wzory redukcyjne uzyskamy
$$\lim_{t \to 0}
{\frac{\sin^5{5t}\sin^{17}{17t}}{\sin^9{9t}\sin^{13}{13t}}} $$
Korzystamy z elementarnej granicy $\lim_{x \to 0}{\frac{\sin{x}}{x}} $
I dostajemy, że
$$\lim_{x \to \frac{\pi}{2}}
{\frac{\cos^5{5x}\cos^{17}{17x}}{\cos^9{9x}\cos^{13}{13x}}} = \frac{5^5 * 17^{17}}{9^9 * 13^{13}} $$

\subsection{k}
$$\lim_{x \to 0}{\sqrt[x]{1-3x}} $$
Napiszmy to równoważnie
$$\lim_{x \to 0}{{(1-3x)}^{\frac{1}{x}}} $$ 
$$\lim_{x \to 0}{\left({(1-3x)}^{\frac{1}{3x}}\right)^3}  = e^3$$ 

\subsection{l}
$$\lim_{x \to 0}{\sqrt[x]{1 + \sin{x}}} $$
Sytuacja pdobna ja w poprzednim
$$\lim_{x \to 0}{\left( 1 + \sin{x}\right)^\frac{1}{x}} $$
$$\lim_{x \to 0}{\left( 1 + \frac{\sin{x}}{x}x\right)^\frac{1}{x}} $$
$$\lim_{x \to 0}{\left( 1 +x\right)^\frac{1}{x}} = e $$

\subsection{m}
$$\lim_{x \to 0}{\frac{1}{x}\ln{\sqrt{\frac{1+x}{1-x}}}}$$
Używając twierdzeń o logarytmie ilorazu
$$\lim_{x \to 0}{\left[\frac{1}{2}\frac{\ln{(1+x)}}{x} + \frac{1}{2}\frac{\ln{(1-x)}}{-x}\right]}$$
$$\frac{1}{2}\lim_{x \to 0}{\frac{\ln{(1+x)}}{x}} + \frac{1}{2}\lim_{x \to 0}{\frac{\ln{(1-x)}}{-x}} = \frac{1}{2} + \frac{1}{2} = 1$$

\subsection{n}
$$\lim_{n \to 0}{\frac{e^{2x-1} - e^{x}}{x}} $$
Napiszmy to w trochę zmienionej formie
$$\lim_{n \to 0}{\frac{e^{2x} - 1 - (e^{x}-1)}{x}} $$
To drugie minus jeden bierze się z dodania $1$ i wyciągnięcia $-1$ przed nawias, następnie podzielmy sobie to na dwie granice
$$\lim_{n \to 0}{\left[\frac{e^{2x}-1}{x} - \frac{e^{x}-1}{x}\right]} $$
$$\lim_{n \to 0}{2\frac{e^{2x}-1}{2x}} - \lim_{n \to 0}{\frac{e^{x}-1}{x}} = 2 - 1 = 1$$

\subsection{o}
$$\lim_{x \to -1}{e^{\frac{1}{x+1}}} $$
Z racji że funkcja wykładnicza jest elementarna wystarczy znaleźć granicę wyrażenia $ \frac{1}{x+1} $
$$\lim_{x \to -1}{\frac{1}{x+1}} $$
Jak nie widać od razu to sobie podzielimy na strony.
$$\lim_{x \to -1^-}{\frac{1}{x+1}} = \left[\frac{1}{0^-}\right] = -\infty$$
$$\lim_{x \to -1^+}{\frac{1}{x+1}} = \left[\frac{1}{0^+}\right] = \infty$$
Z racji, że granica lewostronna jest różna od prawostronnej tak więc granica w $-1$ nie istnieje, przez co nie istnieje granica pierwotnego wyrażenia

\newpage
\subsection{p}
$$\lim_{x \to \infty}{x \sin{\frac{1}{x}}} $$
$$\lim_{x \to \infty}{ \frac{\sin{\frac{1}{x}}}{\frac{1}{x}}} $$
To teraz sztuczka. Powiedzmy że $ p = \frac{1}{x} $ i teraz chcielibyśmy przerobić nasze wyrażenie z użyciem $p$.

Tak więc żeby to zrobić musimy dowiedzieć się do czego dąży $p$, gdy $x \to \infty$ a potem zamienić. Czyli :
$$ \lim_{x \to \infty}{p} = \lim_{x \to \infty}{\frac{1}{x}} = 0$$

Teraz po podstawieniu nasze wyrażenie wygląda tak:
$$\lim_{p \to 0}{\frac{\sin{p}}{p}} = 1$$
Czyli ostatecznie 
$$\lim_{x \to \infty}{x \sin{\frac{1}{x}}} = 1 $$

\subsection{r}
$$\lim_{x \to 0}{\frac{\arctan{x}}{x}} $$
Rozwiązemy to nie stosując wzorku, bo mu nie ufamy ( nikt nam nie pokazał dowodu, a na słowo tak nie można ).

Z racji że nie jest to aż tak fajne użyjemy sobie podstawienia. Niech $ x = \tan{p}\ $, oraz liczymy granice dla dla $ p \to 0 $, żeby było dobrze i wyrażenia były równe.

Niczego to nie zmienia, ponieważ
$$ \lim_{p \to 0}{\tan{p}} = 0 $$
Czyli nasze wyrażenie wygląda następująco.
$$\lim_{p \to 0}{\frac{\arctan{(\tan{p})}}{\tan{p}}} $$
Z racji tego iż nasze $x$ dąży do zera, możemy opuścić $\arctan$
Czyli
$$\lim_{p \to 0}{\frac{p}{\tan{p}}} $$
$$\lim_{p \to 0}{\frac{p\cos{p}}{\sin{p}}} $$
$$\lim_{p \to 0}{\cos{p}\frac{1}{\frac{\sin{p}}{p}}} $$
$$\lim_{p \to 0}{\cos{p}} = 1 $$
Udowodniliśmy wzorek, czyli możemy z niego korzystać.

 

\end{document}