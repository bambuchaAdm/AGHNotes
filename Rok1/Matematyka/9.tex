\documentclass[11pt]{article}
\usepackage[utf8]{inputenc}
%\usepackage[T1]{fontenc}
\usepackage{amssymb}
\usepackage{amsmath}
\usepackage{enumerate}
\usepackage{fullpage}
\usepackage{polski}  
\usepackage{indentfirst} 
\usepackage[pdftex]{graphicx}
\usepackage{multirow}
\usepackage{placeins}

\author{Łukasz Dubiel}
\title{Matematyka Ćwiczenia}

\begin{document}

\maketitle

$$ \int \frac{dx}{x^3-8} $$
Faktoryzujemy wielomian
$$ \int \frac{dx}{(x-2)^2(x^2 + 2x + 4)^2} $$
I teraz rozbijamy na sumę ułamków prostych
$$ \int \frac{dx}{(x-2)^2(x^2 + 2x + 4)^2} = \int \left( \frac{A}{x-2} + \frac{B}{(x-2)^2} + \frac{Cx + D}{x^2 + 2x + 4} + \frac{Ex + F}{x^2+2x+4} \right)dx $$

$$ 1  = A(x-2)(x^2+2x+4)^2 + B(x^2 + 2x + 4)^2 + (Cx+D)(x-2)^2(x^2+2x+4) + (Ex+F)(x-2)^2 $$
$$ 1 = A(x^5 + 4x^2 + 12x^3 + 16x^2 + 16x -2x^2 -8x^3 -24x^2 -32x -32) + B(x^4 + 4x + 12x^2 + 16x + 16) +\\
=Cx^5 - 2Cx^4 =8Cx^2 + 16Cx + D^4 -2Dx^3 +8x^2 + 8Dx + 16D + Ex^3 - 4EX^2 + 4ex +Fx^2 - 4Fx + 4f$$
Daje nam to układ 
$$\begin{cases}
A+C =0 \\
2A + B -2C +D = 0 \\
4A + 4B +2D + E = 0 \\
-8A + 12B -8C - 4E + F = 0 \\
-16A + 16B + 16C +8 D + 4E -4F = 0 \\
-32A + 16A + 16D + 4F = 1
\end{cases}$$
Z pierwszego równania wiemy że $$ C = -A $$. Postawiamy do następnego i wyprowadzamay z drugiego $$D = -4A - B$$.
Uzyskane wyniki postawiamy do trzeciego i wyprowadzamy $$ E = 4A - 2B $$
Idąc dalej tym tropem dostajemy $$ F = 16A - 20B $$
I teraz do ostatnich dwóch równań uzyskane wartości. Uzyskamy układ dwóch równań z dwiema niewiadomymi 
$$\begin{cases}
-16A + 16B - 16A + 8(-4A - B) + 4(4A -2B) - 4(16A -20B) = 0 \\
-32A + 16B - 64A - 16B + 64A + 80B = 1
\end{cases}$$
Układ upraszcza się do
$$\begin{cases}
-112A + 80B = 0 \\
-32A - 80B = 1
\end{cases}$$
Po zsumowaniu 
$$ -144A = 1 $$
$$ A = \frac{-1}{144} $$
$$ B = \frac{-1}{1440} $$

I dalej zabawa

\section{Następna}
$$\int \frac{dx}{(x+1)\sqrt{x^2-1}} $$
Używamy postawianie $ t = x + 1$ 
$$ \int \frac{dx}{t\sqrt{t ( t-2)}} $$
$$ \int t^{-\frac{3}{2}}(t-2)^{-\frac{1}{2}} dt $$

\section{Dygresja}
Dla całek w postaci
$$ \int x^r(a+bx^s)^p \quad r,s,q \in \mathbb{Q} $$
I teraz przypadki
\begin{enumerate}
\item{ $p \in \mathbb{Z}$} \\
Używamy postawienia $x = u^{k}$, gdzie $ k $ to wspólny mianownik $r$ i $s$.
\item{ $\frac{r+1}{s} \in \mathbb{Z} $} \\
Postawiamy $ a + bx^s = u^n $, gdzie $n$ to mianownik p
\item{ $\frac{r+1}{s} + p \in \mathbb{Z}$ } \\
Używamy postawienia $ax^-s + b = u^n$, gdzie $n$ to mianownik $p$.
\end{enumerate}
Koniec dygresji...

\section{Dalsza cześć zadania}
Mamy trzeci przypadek z dygresji
$$ -2t-1 + u = u^2 $$
$$ t = \frac{-2}{u^2-1} $$
$$ t-2 = \frac{-2u^2}{u^2-1}$$
$$ dt = \frac{4u}{(u^2-1)^2} $$
I teraz podstawiając dostajemy 
$$ \int \left( \frac{-2}{u^2-1} \right)^{-\frac{3}{2}} \left( \frac{-2u^2}{u^2-1} \right)^{-\frac{1}{2}} \frac{4u}{(u^2-1)^2}du $$
Po usunięciu ujemnych wykładników skrócą nam sie wyrażenia
$$ \int \frac{1}{2^2}\left(\frac{1}{u^2} \right)^{\frac{1}{2}} 4u du = \int du = u + C $$
$$\sqrt{-2t^{-1} + 1} + C = \sqrt{\frac{-2}{x+1} + 1} + C $$

\end{document}
