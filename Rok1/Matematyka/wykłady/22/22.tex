\documentclass[11pt]{article}
\usepackage[utf8]{inputenc}
%\usepackage[T1]{fontenc}
\usepackage{amssymb}
\usepackage{amsmath}
\usepackage{enumerate}
\usepackage{fullpage}
\usepackage{polski}  
\usepackage{indentfirst} 
\usepackage[pdftex]{graphicx}
\usepackage{multirow}
\usepackage{placeins}

\author{Łukasz Dubiel}
\title{Układy równań}

\begin{document}

\maketitle

\section{Definicja} Układem $m$ równań z $n$ niewiadomymi nazywamy układ równań,
\begin{equation}
\label{rowanie}
\begin{cases}
a_{11} x_1 + \ldots + a_{1n} x_n = b_1 \\
a_{21} x_1 + \ldots + a_{2n} x_n = b_2 \\
\quad \vdots \\
a_{m1} x_1 + \ldots + a_{mn} x_n = b_n \\
\end{cases}
\end{equation}
gdzie $a_{ij} \in \mathbb{R}$ to współczynniki układu $\eqref{rowanie}$. $b_i \in \mathbb{R}$ to wyrazy wolne, $x_j$ - to niewiadome, $\forall i \in \{1, \ldots , m\} \quad \forall j \in \{1,\ldots, n\}$

Rozwiązaniem układu $\eqref{rowanie}$ nazywamy każdą $n$-ke liczb $(x_1,\ldots,x_n) \in \mathbb{R}^n$, która spełnia wszystkie równanie układu $\eqref{rownanie}$

Macierze związane z układem $\eqref{rownanie}$

Macierz $ A = \begin{vmatrix}
	11 & 12 & 13\\
	21 & 22 & 23\\
	31 & 32 & 33\\
\end{vmatrix}$ nazywamy macierzą współczynników układu $\eqref{rownanie}$ .

Macierz $\begin{bmatrix}
	b_1 \\
	b_2 \\
	\vdots \\
	b_n
\end{bmatrix}$ nazywamy macierza ( kolumna ) wyrazów wolnych.

Macierz $ X = \begin{bmatrix}
	11\\
	21\\
	31\\
\end{bmatrix}$ nazywamy macierzą (kolumną) niewiadomych.

Macierz $ [A|B] = \begin{bmatrix}
	11 & 12 & 13 & 14\\
	21 & 22 & 23 & 24\\
	31 & 32 & 33 & 34\\
\end{bmatrix}$ nazywamy macierzą uzupełnie układu $\eqref{rownanie}$

\subsection{Uwaga}
Układ $\eqref{rownanie}$ można zapisać w sposób równoważny w postaci macierzowej.
$$ A \cdot X = B $$

\section{Definicja}
Układ $\eqref{rownanie}$ nazywamy jednorodną wtedy i tylko wtedy, gdy $\forall i \in  \{ 1 , \ldots , m \} \quad b_i = 0$, czyli B jest macierzą zerową. 
W przeciwnym wypadku mówimy, że układ $\eqref{rownanie}$ jest niejednorodny.
\section{Definicja}
Układ $\eqref{rownanie}$ nazywamy oznaczonym (nieoznaczonym) [sprzecznym] wtedy i tylko wtedy, gdy posiada dokładnie jedno rozwiązanie (nieskończonie wiele rozwiązań)[zero rozwiązań]. 

\section{Układy kwadratowe}
\subsection{Definicja}
Układ $\eqref{rownanie}$ nazywamy układem kwadratowym wtedy i tylko wtedy, gdy $m = n$ ( czyli macierz A jest kwadratowa).
\subsection{Definicja układu Cramera}
Układ kwadratowy $\eqref{rownanie}$ nazywamy układem Cramera, wtedy i tylko wtedy, gdy $ \det{A} \not = 0$

\section{Twierdzenie Cramera}
Jeśli układ $\eqref{rownanie}$ jest układem Cramera, to
\begin{enumerate}
\item{ układ $\eqref{rownanie}$ jest zonaczony}
\item{ rozwiązaniem układu jest dane przez tak zwane wzory Cramera.
$$ \forall j \in \{1, \ldots , n \} \quad x_j = \frac{\det{A_{x_j}}}{\det{A}}$$ gdzie $A_{x_j}$ powstaje przez $A$ przez zastąpienie $j$-tej kolumny kolumną wyrazów wolnych.}
\end{enumerate}
\section{Przykłady}
$$ \begin{cases} 2x - 3y - z = 1 \\ x - y = 0 \\ 2x + z = - 1\end{cases}$$
Sposób I ( z wzorów Cramera )
$ \det{A} = \begin{bmatrix}
	2 & 3 & -1\\
	1 & -1 & 0\\
	2 & 0 & 1\\
\end{bmatrix}= -1 \Rightarrow \hbox{Układ jest układem cramera} \\ \det{A_x} = \begin{bmatrix}
	1 & 3 & -1\\
	0 & -1 & 0\\
	1 & 0 & 1\\
\end{bmatrix} = 0 \quad \det{A_y} = \begin{bmatrix}
	2 & 1 & -1\\
	1 & 0 & 0\\
	2 & 1 & 1\\
\end{bmatrix} = 0 \quad \det{A_z} =\begin{bmatrix}
	2 & 3 & 1\\
	1 & -1 & 0\\
	2 & 0 & -1\\
\end{bmatrix} = 1$
Czyli układ jest układem oznaczonym i rozwiazania są dane wzorami Cramera $ X = (0 , 0 , -1) $

Drugi sposób
$$ A \cdot X = B \quad/ A^{-1} \cdot $$
$$ A^{-1} \cdot A \cdot X = A^{-1} \cdot B $$
$$ X = A^{-1}\cdot B$$

$$ A^{-1}=\begin{bmatrix}
	1 & -3 & 1\\
	1 & -4 & 1\\
	-1 & 6 & -1\\
\end{bmatrix}$$
$$ X = \begin{bmatrix}
	x\\
	y\\
	z\\
\end{bmatrix} = \begin{bmatrix}
	1 & -3 & 1\\
	1 & -4 & 1\\
	-1 & 6 & -1\\
\end{bmatrix}\cdot \begin{bmatrix}
	1\\
	0\\
	-1\\
\end{bmatrix} = \begin{bmatrix}
	0\\
	0\\
	1\\
\end{bmatrix} $$

\section{Układy dowolne}
\section{Twierdzenie Kroneckera - Capellego}
Układ $\eqref{rownanie}$ posiada co najmniej 1 rozwiązanie $\iff r(A) = r([A|B])$ 
$$ A \cdot X = B \iff \begin{bmatrix}
	11 & 12 & 13 \\
	21 & 22 & 23 \\
	31 & 32 & 33 \\
\end{bmatrix} \cdot \begin{bmatrix}
	x_1\\
	x_2\\
	\vdots\\
	x_n
\end{bmatrix} = \begin{bmatrix}
	11\\
	21\\
	31\\
	41\\
\end{bmatrix} \iff x_1 \begin{bmatrix}
	11\\
	21\\
	31\\
	41\\
\end{bmatrix} + x_2 \begin{bmatrix}
	11\\
	21\\
	31\\
	41\\
\end{bmatrix} + \ldots + x_n \begin{bmatrix}
	11\\
	21\\
	31\\
	41\\
\end{bmatrix} = \begin{bmatrix}
	11\\
	21\\
	31\\
	41\\
\end{bmatrix}
$$
Czyli kolumna B jest liniowo niezależna z A, jeśli tylko układ $\eqref{rownanie}$ ma rozwiązanie, a to zonacza, że $r([A|B]) = r(A)$

\subsection{Wniosek}
Układ jest sprzeczny $\iff r(a) < r([A|B])$

\section{Twierdzenie o układach niesprzecznych}
\begin{enumerate}
\item{Jeśli $r(A) = r([A|B]) = n$ czyli rząd jest równy ilości zmiennych to układ $\eqref{rownanie}$ jest oznaczony}
\item{Jeśli $r(A) = r([A|B]) = r < n$ czyli rząd jest mniejszy niż ilość zmiennych to układ $\eqref{rownanie}$ jest nieoznaczony, a rozwiązania zależą od $(n-r)$ parametrów.}
\end{enumerate}
\subsection{Uwaga}
nie ma innych układów równań liniowych niż sprzeczne, nieoznaczone i oznaczone.

\subsection{Wniosek o układach jednorodnych}
Układy jednorodne są niesprzeczne. Jeśli są oznaczone to jednym rozwiązaniem jest wektor zerowy. 

\subsection{Wnioski o układach kwadratowych}
\begin{enumerate}
\item{Twierdzenie Cramera jest wnioskiem z twierdzenie o układach niesprzecznych, bo \\$\det{A} \not = 0 \Rightarrow r(A) = n$}
\item{Jeśli dla układzie kwadratowego mamy, $\det{A} = 0$ oraz $x_i \quad \det{A_{x_i}} \not = 0$ to układ jest sprzczy gdyż $r(A) < n$ , a $r([A|B]) = n$ }
\item{Jeśli dal układu lniowego mamy $ \det{A} = \det{A_{x_1}} = \det{A_{x_2}} = \cdot = \det{A_{x_n}} = 0$, to jedyne co wiemy, że ukłąd nie jest oznaczony. Gdyż
$r(A) < n$ oraz $r([A|B]) < n$ i nic nie wiemy.}
\end{enumerate}
\section{Algorytm Gaussa ( rozwiązywania dowolnych układów równań)}
Wystarczy sprowadzić macierz $[A|B]$ do postaci schodkowej i wszystko widać.  ( Operacje elementarne na macierzy $[A|B]$ odpowiadają równoważnemu przekształcaniu układu $\eqref{rownanie}$ - zbiór rozwiązań się nie zmieni !).
\subsection{Przykład 1}
$$ \begin{cases}
x- y = 3 \\
2x+y = 0 \\
-x + 2y = -5 
\end{cases} $$
\subsection{Przykład 2}
$$ \begin{cases}
2x -2y - 6z = 0 \\
x - y + 2z = 1 \\
3x - 2y + 11z = 1 \\
-x -3z = 1 \\
\end{cases}$$
\subsection{Przykład 3}
$$ \begin{cases}
x_1 + 2x_2 + x_3 - x_4 + x_5 = 0 \\
x_1 + 2x_2  +2 x_3 + x_4 + 2x_5 = 1 \\
-2x_1 - 4x_2 x_3 + 2x_4 - x_5 - 2 \\
\end{cases}
$$

Za parametry przyjmujemy kolumny, gdzie nie ma schodków. 

\section{Twierdzenie o układach jednorodnych}
Dla układu jednorodnego, który możemy zapisać $$ A \cdot X = 0$$ z $n$ niewiadomymi, określamy zbiór rozwiązań $$W_{0} := \{ (x_1,\ldots,x_n) \in \mathbb{R}^n : A \cdot \begin{bmatrix}
	11\\
	21\\
	31\\
\end{bmatrix} = \begin{bmatrix}
	11\\
	21\\
	31\\
\end{bmatrix} \}$$
Wtedy 
\begin{enumerate}
\item{$W_0$ jest podprzestrzenią wektorową w $\mathbb{R}^n$}
\item{$\dim{W_0} = n - r(A)$}
\end{enumerate}

\section{Twierdzenie o układach niejednorodnych}
Niech \begin{equation}A \cdot Y = B \label{nie} \end{equation} - układ niejednorodny, oraz \begin{equation}A \cdot X = 0 \label{jed} \end{equation} - to jest układ jednorodny i jest stworzony z układem \eqref{nie}

Jeśli $ Y = \begin{bmatrix}
	y_1\\
	\vdots\\
	y_n\\
\end{bmatrix}$ - jedno z rozwiązań układu $\eqref{nie}$, to
\begin{enumerate}
\item{Wtedy rozwiązanie układu niejednorodnego $\eqref{nie}$ ma postać $Y = X + Y_{0}$, gdzie X jest jednym z równań układu $\eqref{jed}$}
\item{Dla dowolnego rozwiązania $X$ układu $\eqref{jed}$ $X+Y_{0}$ jest rozwiązaniem układu $\eqref{nie}$.}
\end{enumerate}

\subsection{Wniosek}
Każde rozwiązanie układu niejednorodnego $\eqref{nie}$ jest sumą rozwiązania szczególnego układu $\eqref{nie}$ i jakiegoś dowolnego rozwiązania układu $\eqref{jed}$.
$$ W = \{ (x_1, \cdot , x_n) \in \mathbb{R}^n :  A \cdot \begin{bmatrix}
	x_1\\
	\vdots\\
	x_n\\
\end{bmatrix} = B$$
$$W = Y_0 + W_0$$



\end{document}
