\documentclass[11pt]{article}
\usepackage[utf8]{inputenc}
%\usepackage[T1]{fontenc}
\usepackage{amssymb}
\usepackage{amsmath}
\usepackage{enumerate}
\usepackage{fullpage}
\usepackage{polski}  
\usepackage{indentfirst} 
\usepackage[pdftex]{graphicx}
\usepackage{multirow}
\usepackage{placeins}

\author{Łukasz Dubiel}
\title{Macierze odwrotna}

\begin{document}

\maketitle

\section{Definicja}
Mamy macierz kwadratową $A \in M_{n \times n}$. Jeżeli istnieje macierz $ B \in M_{ n \times n}$ taka, żę
$$ A \cdot B = B \cdot A = I_{n \times n} $$, to mówimy że  $A$ jest odwaracalna , a macierz $B$ nazywamy macierzą odwrotną do $A$. \\ Zapis $B = A^{-1}$ .

\subsection{Uwaga}
Jeśli dla $A \in M_{n \times n}$ istnieje $ A^{-1}$, to jest jedna JEDYNA!.

\section{Twierdzenie o postaci macierzy odwrotnej}
Jeśli $A \in M_{n \times n}$ jest odwracalna, to \\ \begin{enumerate}
\item{ $$\det{A}\not = 0$$}
\item{ $$A^{-1} = \frac{1}{\det{A}} \cdot \begin{bmatrix}
	D_{11} & \ldots & D_{1n}\\
	\vdots &  & \vdots\\
	D_{n1} & \ldots & D_{nn}\\
\end{bmatrix}$$ gdzie $D_{ij} = (-1)^{i+j} \cdot M_{ij} = (-1)^{i+j} \det{A_{ij}}$ , $M_{ij}$- to minorant, a $A_{ij}$ - po wycięciu wiersza $i$-tego oraz $j$-tej kolumny wycięciu}
\end{enumerate}
\subsection{Wniosek}
$$ A \hbox{ jest odwracalna} \iff \det{A} \not = 0 $$
\subsubsection{Dowód}
$\\(\Leftarrow)$ z poprzedniego twierdzenia punkt 1. \\
$(\Rightarrow)$ z poprzedniego twierdzenia punkt 2.

\section{Definicja macierzy osobliwej i nie osobliwej}
Macierz $A \in M_{n \times n}$ nazywamy nieosobliwą wtedy i tylko wtedy, gdy $\det{A} \not = 0$. W przeciwnym wypadku mówimy, że a jest osobliwa.
\subsection{Wniosek}
$$ A \hbox{ jest odwaracalna} \iff A \hbox{ jest nieosobliwa} $$

\section{Własności macierzy odwrotnej}
Miech $A,B \in M_{n \times n}$ i nieosobliwe. Wtedy \\
\begin{enumerate}
\item{ $$ (A^{-1})^{-1} = A $$}
\item{ $$ (A^{T})^{-1} = (A^{-1})^{T}$$}
\item{ $$ (\alpha A)^{-1} = \frac{1}{\alpha} A^{-1} \forall\ \alpha \in \mathbb{R} \backslash \{0\}$$}
\item{ $$ (A \cdot B)^{-1} = B^{-1} \cdot A^{-1} $$}
\item{ $$ (A^n)^{-1} = (A^{-1})^n $$}
\end{enumerate}

\subsection{ Uwaga do punktu 4 }
$$ (A \cdot B)^{-1} = A^{-1} \cdot B^{-1} $$ Nie tak
$$ (A \cdot B) \cdot A^{-1} \cdot B^{-1} = A \cdot B \cdot A^{-1} \cdot B^{-1} $$
Ale dlaczego .
$$(A \cdot B)^{-1} = B^{-1} \cdot A^{-1} $$
$$(A \cdot B) \cdot B^{-1} \cdot A^{-1} = A \cdot I \cdot A^{-1} = I$$
\section{Metody wyznaczania macierzy odwrotnej}
\begin{enumerate}
\item{przez macierz dopełnień algebraicznych - do bani dla macierzy stopnia $\geq 5$}
\item{prez przekształcenie układu równań}
\item{algorym Gaussa}
\end{enumerate}
\subsection{Przekształcenie do układu równań}
Przykład $ A = \begin{bmatrix}
	2 & 3 & -1\\
	1 & -1 & 0\\
2 & 0 & 1\\
\end{bmatrix} \quad
\det{A} \not = 0 \quad A \cdot \begin{bmatrix}
	x_1\\
	x_2\\
	x_3\\
\end{bmatrix} = \begin{bmatrix}
	y_1\\
	y_2\\
	y_3\\
\end{bmatrix} \iff \begin{cases} 2x_1 - 3x_2 - x_3 = y_1 \\ x_1 - x_2 = y_2 \\ 2x_1 + x_3 = y^3\\ \end{cases}$
Teraz 
$$ A \cdot X = Y / A^{-1} \cdot $$ 
$$ A^{-1} \cdot A \cdot X = A^{-1} Y $$
$$ X = A^{-1} Y \iff \begin{bmatrix}
	x_1\\
	x_2\\
	x_3\\
\end{bmatrix} = \begin{bmatrix}
	 &  & \\
	 & A^{-1} & \\
	 &  & \\
\end{bmatrix} \cdot \begin{bmatrix}
	y_1\\
	y_2\\
	y_3\\
\end{bmatrix}$$

$$ \begin{cases} 2x_1 - 3x_2 - x_3 = y_1 \\ x_1 - x_2 = y_2 \\ 2x_1 + x_3 = y_3 \end{cases} \Rightarrow \begin{cases} x_1 = y_1 - 3y_2 + y_3 \\ x_2 = y_1 - 4y_2 + y_3 \\ x_3  = 2y_1 - 6y_2 + y_3 \end{cases} $$
$$ A^{-1} = \begin{bmatrix}
	1 & -3 & 1\\
	1 & -4 & 1\\
	2 & -6 & 1\\
\end{bmatrix}$$

Jako ćwiczenie wykonać $A \cdot A^{-1}$ oraz $A^{-1} \cdot A$

\subsection{Algorytm Gaussa}
Mamy macierz $$ A = \begin{bmatrix}
	2 & 3 & -1\\
	1 & -1 & 0\\
2 & 0 & 1\\
\end{bmatrix}$$

To teraz tworzymy macierz \\
$$\begin{bmatrix}
	A & | & I_{n \times n} 
\end{bmatrix}_{n\times 2n} \xrightarrow{\hbox{Operacje elementarne na wierszach}}\begin{bmatrix}
	I_{n \times n} & | & B\\
\end{bmatrix} \Longrightarrow B = A^{-1}
$$
$$ A = \begin{bmatrix}
	2 & 3 & -1 & 1 & 0 & 0\\
	1 & -1 & 0 & 0 & 1 & 0\\
2 & 0 & 1 & 0 & 0 & 1\\
\end{bmatrix} \xrightarrow{w_1 \iff w_2} \begin{bmatrix}
	1 & -1 & 0 & 0 & 1 & 0\\
	2 & 3 & -1 & 1 & 0 & 0\\
2 & 0 & 1 & 0 & 0 & 1\\
\end{bmatrix} \xrightarrow{w_1 - 2w_1 \\ w_3 - 2w_1} 
\begin{bmatrix}
	1 & -1 & 0 & 0 & 1 & 0\\
	0 & -1 & -1 & 1 & 0 & 0\\
2 & 0 & 1 & 0 & 0 & 1\\
\end{bmatrix}$$

\section{Rząd macierzy}
\subsection{Definicja}
Rzędem macierzy $A \in M_{m \times n}$ nazywamy maksymalną ilość jej liniowo niezależnych wierszy traktowanych jako wektory z $\mathbb{R}^n$ lub kolumn traktowanych jako wektory z $\mathbb{R}^m$. \\ Oznaczenie $ r{(A)}$.
\subsection{Uwaga}
Definicja jest poprawna. Maksymalna liczba liniowo niezależnych wierszy $=$ maksymalna liczba liniowo niezależnych kolumn.

\section{Macierz schodkowa}
\subsection{Definicja}
Mówimy, że macierz $A \in M_{m \times n}$ ma postać schodkową jeśli wszystkie jej niezerowe wiesze występują kolejno jeden po drugim, począwszy od pierwszego oraz w każdym takim wierszu pierwszy niezerowy element znajduje się w kolumnie o wskaźniku większym niż wskaźnik kolumny w którym znajduje się pierwszy niezerowy element wiersza poprzedniego.
\subsection{Przykład}
$$\begin{bmatrix}
	0 & -1 & 3 & -1 & 1 & 0\\
	0 & 0 & 1 & 1 & 0 & 1\\
	0 & 0 & 0 & 0 & 2 & 1 \\
	0 & 0 & 0 & 0 & 0 & 0\\
\end{bmatrix}$$

\subsection{Uwaga}
Każdą macierz można doprowadzić do macierzy schodkowej stosując operacje elementarne na wierszach tej macierzy (np. algorym Gaussa). Takie operacje elementarne nie wpływają na rząd macierzy.

\section{Twierdzenie o rzędzie macierzy schodkowej}
Rząd macierzy schodkowej jest równy ilości schodków tej macierzy.
\subsection{Przykład}

$\\ \begin{bmatrix}
	0 & 0 & 0 & 0 & 2 & 1\\
	0 & -1 & 3 & -1 & 1 & 0\\
	0 & -2 & 7 & -1 & 2 & -1\\
	0 &  1 & 0 & 4 & -5 & -5\\
\end{bmatrix} \xrightarrow{w_1 \iff w_2} \begin{bmatrix}
	
	0 & -1 & 3 & -1 & 1 & 0\\
	0 & 0 & 0 & 0 & 2 & 1\\
	0 & -2 & 7 & -1 & 2 & -1\\
	0 &  1 & 0 & 4 & -5 & -5\\
\end{bmatrix}  \xrightarrow{w_3 - 2w_1 \\ w_4 + w_1} 
\begin{bmatrix}
	
	0 & -1 & 3 & -1 & 1 & 0\\
	0 & 0 & 0 & 0 & 2 & 1\\
	0 & 0 & 1 & -1 & 0 & -1\\
	0 &  0 & 3 & 3 & -4 & -5\\
\end{bmatrix} \xrightarrow{w_2 \iff w_3} 
\begin{bmatrix}
	
	0 & -1 & 3 & -1 & 1 & 0\\
	0 & 0 & 1 & -1 & 0 & -1\\
	0 & 0 & 0 & 0 & 2 & 1\\
	0 &  0 & 3 & 3 & -4 & -5\\
\end{bmatrix} \xrightarrow{w_4 - 3w_2} 
\begin{bmatrix}
	
	0 & -1 & 3 & -1 & 1 & 0\\
	0 & 0 & 1 & -1 & 0 & -1\\
	0 & 0 & 0 & 0 & 2 & 1\\
	0 &  0 & 0 & 0 & -4 & -5\\
\end{bmatrix} \xrightarrow{w_3 + 2w_3}
\begin{bmatrix}
	
	0 & -1 & 3 & -1 & 1 & 0\\
	0 & 0 & 1 & -1 & 0 & -1\\
	0 & 0 & 0 & 0 & 2 & 1\\
	0 &  0 & 0 & 0 & 0 & -3\\
\end{bmatrix}
$
\section{Własności rzędu macierzy}
\begin{enumerate}
\item{$$ A \in M_{m \times n} \quad  \Rightarrow r(A) \leq \min{(m,n)}$$}
\item{$$ \forall\ A \in M_{m \times n} \quad r(A) = r(A^{T})$$}
\item{$$ \forall\ A,B \in M_{m \times n} \quad r(A+B) \leq r(A) + r(B)$$}
\item{$$ \forall\ A \in M_{m \times n}\quad \forall B \in M_{n \times k} r(A \cdot B) \leq \min\{r(A),r(B)\} $$}\end{enumerate}

\section{Twierdzenie o rzędzie}
Rząd macierzy $A_{n \times n}$ jest równy największemu ze stopni minorów niezerowych w A.
\subsection{Przykład}
[Zrobiona wcześniej macierz schodkowa]
Wszytkie minory stopnia większego są zerami [To widać]
Wszystkie minory stopnia 4 są równe zero i istnieje co najmniej jeden minor nie zerowy stopnia trzy.
\subsection{Przykład}
$$ A = \begin{bmatrix}
	1 & 3 & 0\\
	0 & -1 & 5\\
	1 & 2 & 5\\
\end{bmatrix} \quad \det{A} = 0 \quad \begin{vmatrix}
	11 & 3\\
	0 & -1\\
\end{vmatrix} = -1
$$
Więc rząd macierzy jest równe do.
\subsection{Przykład}
Sprawdz czy wketory $(1,-1,0),(2,0,-1),(1,1,1)$ tworzą bazę w $\mathbb{R}^n$
To teraz bierzemy sobie macierz
$$\begin{vmatrix}
	1 & -1 & 0\\
	2 & 0 & -1\\
	1 & 1 & 1\\
\end{vmatrix} = 4$$
Więc wektory są liniowo niezależne.

\end{document}
