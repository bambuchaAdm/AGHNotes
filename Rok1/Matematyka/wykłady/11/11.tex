\
documentclass[11pt]{article}
\usepackage[utf8]{inputenc}
%\usepackage[T1]{fontenc}
\usepackage{amssymb}
\usepackage{amsmath}
\usepackage{enumerate}
\usepackage{fullpage}
\usepackage{polski}  
\usepackage{indentfirst} 
\usepackage[pdftex]{graphicx}
\usepackage{multirow}
\usepackage{placeins}

\author{Łukasz Dubiel}
\title{Matematyka - wykład 11}

\begin{document}
\maketitle

\section{Całki nieelementarne}
Dla każdej funkcji elementarnej jej pochodna jest jakąś (inną) funkcją elementarną.

Z całkami, sorry, nie jest tak prosto.

\bigskip
Przykład całki nie elementarnej
$\int\sqrt{1+x^3}dx$ - całka eliptyczna

$$\int e^{-x^2}dx$$
$$\int \cos{x^2}dx$$ \\
$$\int \sin{x^2}dx$$ \\
$$\int \frac{\sin{x}}{x}dx$$ sinus całkowy \\
$$\int \frac{\cos{x}}{x}dx$$  \\
$$\int \frac{e^x}{x}dx$$ \\
$$\int \frac{dx}{\ln{x}}$$ \\

\section{Najprostsze metody całkowania}

\subsection{Twierdzenie o liniowości całki}
Niech $\alpha$ - liczba rzeczywista, $f$,$g \in C(I), \ \ I \subset \mathbb{R}$

Wtedy
\begin{enumerate}
\item{$\int (\alpha f(x))dx = \alpha \int f(x) dx$}
\item{$\int(f(x) + g(x))dx = \int f(x) dx + \int g(x)dx$}
\end{enumerate}
Dowód 

Ad.1) $$\frac{d}{dx}(\alpha f(x)) = \alpha f(x)$$

Ad.2) $$\frac{d}{dx}(\int f(x)dx + \int g(x)dx) = \frac{d}{dx}(\int f(x)dx) + \frac{d}{dx}(\int g(x)dx) = f(x) + g(x)$$

\subsection{Całkowanie przez cześci}
Jeśli funkcje $u,v \in C^1(I)\quad I \subset \mathbb{R}$ -przedział otwarty 
to
$$\int u(x)v'(x)dx = u(x)v(x) - \int u'(x)v(x)dx \quad \forall x \in I $$
Dowód.
Robimy pochodną prawej strony i jak otrzymamy lewą, to jest ok.
$$\frac{d}{dx}\left(u(x)v(x) - \int u'(x)v(x)dx\right)$$
I wychodzi od razu 

\subsubsection{Przykład - rozwalamy logarytm}
$$\int \ln{x} dx $$
Możemy równoważnie zapisać
$$\int 1 \ln{x} dx $$
I teraz przyjmując $u(x) = \ln{x}$ , $v'(x) = 1$ dostajemy
$$\int 1 \ln{x} dx = x\ln{x} - \int x \frac{1}{x} dx = x\ln{x} - x + c$$

Przyjmując $u(x) = 1$ i $v'(x) = \ln{x}$ dochodzi do wielkiej bani.
\subsection{Przykład - rozwalamy $\int x^2\sin{x}dx$}
$$\int x^2 \sin{x} dx$$
przyjmujemy $u(x)=x^2$ i $v'(x) = \sin{x}$ przez co $u'(x) = 2x$ i $v(x) = -\cos{x}$
Czyli mamy
$$\int x^2 \sin{x} dx = x^2\sin{x} + 2\int x\cos{x}$$
Stosujemy jeszcze raz... Przyjmijmy $u(x) = x$ oraz $v'(x) = -\cos{x}$ i teraz dostajemy 
$$\int x \cos{x}dx = x\cos{x} + \int \sin{x}$$
$$\int x^2 \sin{x} dx = x^2\sin{x} + 2x\cos{x} + \cos{x} + c$$

\subsection{Przykład $\int e^x\sin{x}dx$}
$$\int e^x\sin{x}dx$$
Stosując przez części $u(x) = e^x$, a $v'(x) = \sin{x}$
$$\int e^x\sin{x}dx = -e^{x}\cos{x} + \int e^x \cos{x} dx$$
Wygląda na to iż nie za bardzo, to może jeszcze raz
$$\int e^x \cos{x} dx$$
I teraz $u(x) = e^x$, oraz $v'(x) = \cos{x}$ i dostajemy
$$\int e^x \cos{x} dx = e^x\sin{x} - \int e^x\sin{x} dx$$
Okazuje się że ostatni wyraz jest naszą szukaną całką tak więc na drugą stronę.
$$\int e^x\sin{x} dx = \frac{e^x\cos{x} + e^x\sin{x}}{2}$$

\subsection{Całkowanie przez podstawianie}
TW:
Niech $I,Y  \subset \mathbb{R}$ - przedziały otwarte, $ \phi : J \to I$ jest bijekcją  
Jeśli $\phi \in C^1(J)$ to

$$\int f(\phi(x))\phi'(x)dt = \int f(t)dt \quad t = \phi{x}$$
\newpage
Dowód:
Niech $F(t)$ - pierwotna $f(t)$
$$\int f(t)dt = F(t) + C$$
$$\frac{d}{dt}\left( F(\phi(x)) \right)$$

\subsection{Przykład}
$$\int x e^{x^2} dx$$
tak więc $f(x) = \frac{1}{2}e^x$ oraz $\phi(x) = x^2 = t$ i uzyskujemy
$$\int x e^{x^2} dx = \int \frac{1}{2}e^{x^2}2xdx = \int \frac{1}{2}e^t dt = \frac{1}{2}e^{x^2} + C$$ 

\subsection{Przykład}
$$\int \sin{2x}dx $$
Dwa podstawienia $ t = 2x$ oraz $ t = \sin{x}$.
Wychodzą dwa różne wyniki różne od stałej.

\subsection{Szczególne przypadki}
\subsubsection{Logarytm laturalny}
$$\int \frac{g'(x)}{g(x)}dx$$
Przez podstawienie $ g(x) = t $ dostaniemy, że
$$\int \frac{g'(x)}{g(x)}dx = \ln{|g(x)|} + C$$
\subsubsection{$g(x)^{\alpha}$}
$$\int g'(x)(g(x))^{\alpha}dx$$
podstawienie $ t = g(x) $ uzyskamy
$$ \int t^\alpha dt == \frac{t^{\alpha+1}}{\alpha +1} = \frac{(g(x))^{\alpha+1}}{\alpha+1}$$

\subsubsection{Przykłady}
$$\int \tan{x} dx$$
ćwiczenie 
$$\int \ctg{x} dx$$
$$\int (2x+3)^6 dx$$
$$\int x^3\sqrt{4+x^2}dx$$

$$\int \arctan{x}dx$$
Całkujemy przez części $u(x) = \arctan{x}$ , $v'(x) = 1$
Czyli
$$\int \arctan{x}dx = x\arctan{x} - \int \frac{x}{1+x^2}dx$$
Tym razem bierzemy podstawienie $t = x^2 + 1$, więc 
$$\frac{1}{2}\int \frac{dt}{t} = \frac{1}{2}\ln{1+x^2} + C$$ czyli 
$$\int \arctan{x}dx = x \arctan{x} - \frac{1}{2}\ln{1+x^2} + C$$

$$\int \arcsin{x}dx$$
Przez części

\end{document}
