\documentclass[11pt]{article}
\usepackage[utf8]{inputenc}
%\usepackage[T1]{fontenc}
\usepackage{amssymb}
\usepackage{amsmath}
\usepackage{enumerate}
\usepackage{fullpage}
\usepackage{polski}  
\usepackage{indentfirst} 
\usepackage[pdftex]{graphicx}
\usepackage{multirow}
\usepackage{placeins}

\author{Łukasz Dubiel}
\title{Matematyka \\ Odwzorowania liniowe }

\begin{document}

\maketitle

\section{Definicja}
Odwzorowania $f: \mathbb{R}^n \to \mathbb{R}^m$ nazywamy liniowym wtedy i tylko wtedy, gdy \begin{enumerate}
\item{$$\forall\ u,v \in \mathbb{R}^n \quad f(u + v) = f(u) + f(v)$$}
\item{$$\forall u \in \mathbb{R}^n \forall \alpha \in \mathbb{R} \quad f(\alpha u) = \alpha f(u)$$}
\end{enumerate}
$\mathbb{R}^n$ nazywamy dziedziną $f$, $\mathbb{R}^m$ przeciwdziedziną funkcji $f$.
\subsection{Wniosek}
Odwzorowanie liniowe $f: \mathbb{R}^n \to \mathbb{R}^m$ ma następujące własności
\begin{enumerate}
\item{$$f(\overline{0_{\mathbb{R}^n}}) = \overline{0}_{\mathbb{R}^m}$$}
\item{$$\forall u \in \mathbb{R}^n \quad f(-u) = -f(u)$$}
\end{enumerate}
\subsection{Dowód}
\begin{enumerate}
\item{$$ f(\overline{0}_{\mathbb{R}^n}) = f( 0 \cdot u ) = 0 f(u) = \overline{0}_{\mathbb{R}^m}$$}
\item{$$f(-u) + f(u) = f( -u + u ) = f(\overline{0}_{\mathbb{R}^n}) = \overline{0}_{\mathbb{R}^m}$$}
\end{enumerate}

\subsection{Przykład, funkcja liniowa z szkoły}
$$f : \mathbb{R} \to \mathbb{R} $$
$$ f(x) = ax + b$$
$$ f(0) = a \cdot 0 + b = b $$
jeśli f, odwzorowanie liniowe to $b = 0$
$$ f(x) = ax$$
\begin{enumerate}
\item{$$f(x_1 + x_2) = a(x_1 + x_2) = ax_1 + ax_2 = f(x_1) + f(x_2)$$}
\item{$$f(\alpha x) = a \alpha x = \alpha a x = \alpha f(x)$$}
\end{enumerate}
\subsection{Przykłady}
\begin{enumerate}
\item{obrót płaszczyzny wokół punkut $(0,0)$ jest odwzorowaniem liniowym $$ \mathbb{R}^2 \to \mathbb{R}^2 $$}
\item{symetrie płaszczyzny względem prostych przechodzących przez $(0,0)$ są liniowymi przekształceniami $$ \mathbb{R}^2 \to \mathbb{R}^2 $$}
\end{enumerate}

\subsection{Przykład}
$$f:  \mathbb{R}^3 \ni (x,y,z) \to (2x + y - z, x - 3z) \in \mathbb{R}^2$$

\begin{enumerate}
\item{$$ u = (x_1,y_1,z_1), v = ( x_2,y_2,z_3)$$ $$ f(u + v) = f(x_1 + x_2 , y_1 + y_2 , z_1 + z_2 ) = (2(x_1 + x_2) - y + y+2 - (z_1 + z_2), x_1 + x_2 - 3(z_1 + z_2)) = (2x_1 + y_1 - z_1 , x_1 - 3z_1) + (2x_2 + y_2 -z+2, x_2 + 3z_2) = f(u) + f(v)$$}
\item{$$ u = (x , y, z) , \quad \alpha \in \mathbb{R} $$ $$f(\alpha u) = f(\alpha x , \alpha y, \alpha z) = (2\alpha x+ \alpha y - \alpha z, \alpha x - 3 \alpha z) = \alpha f(u)$$}
\end{enumerate}

\subsection{Przykład}

$$ f: \mathbb{R}^3 \ni (x,y,z) \to (2x +  y - z , x - 3z + 1) \in \mathbb{R}^2$$
Liniowe nie jest

\subsection{Przykład}
$$f:\mathbb{R}^2 \ni (x,y) \to ( x^2 , xy , -x - y) \in \mathbb{R}^3$$
Też nie jest liniowe

\section{Twierdzenie, warunek konieczny i wystarczający na liniowe odwozrowania}
$$f : \mathbb{R}^n \to \mathbb{R}^m \hbox{ jest liniowe } \iff \forall\ u,v \in \mathbb{R}^n\ \forall\ \alpha, \beta \in \mathbb{R} \quad f(\alpha u + \beta v) = \alpha f(u) + \beta f(v)$$

\section{Uogulniowny WKW na liniowe przekształcenia}
$$f : \mathbb{R}^n \to \mathbb{R}^m \hbox{ jest liniowe } \iff \forall\ u,v \in \mathbb{R}^n\ \forall\ \alpha, \beta \in \mathbb{R} \quad f(\alpha_1 u_1 + \ldots + \alpha_n u_n) = \alpha_1 f(u_1)+ \ldots + \alpha_n f(u_n)$$

\subsection{Wniosek}
Aby zbadać, odwzorowanie linowe $f : \mathbb{R}^n \to \mathbb{R}^m$ w sposób jednoznaczny możemy podać, jego współrzędne dla n wektó¶w z jakieś bazy $\mathbb{R}^n$
\subsubsection{Uzasadnienie}
Niech $B = (u_1,\ldots,u_n)$ - reper bazowy w $\mathbb{R}^n$ 
$$v_1 := f(u_1)\quad v_2 := f(u_2) , \ldots , v_n := f(u_n) $$
Niech $$ u \in \mathbb{R}^n $$ - dowolny 
$$ \exists \alpha_1, \alpha_2 ,\ldots , \alpha_n  \in \mathbb{R} : u = \alpha_1 u_1 + \alpha_2 u_2 + \ldots + \alpha_n u_n \Longrightarrow [\alpha_1,\alpha_2, \ldots , \alpha_n]_B $$
$$ f(u) = f(\alpha_1 u_1 + \alpha_2 + u_2 + \ldots + \alpha_n u_n) = \alpha_1 f(u_n)$$

\section{Interpretacje geometryczne}
\subsection{$\mathbb{R} \to \mathbb{R}$}
Przejście z prostej na prostą.
\subsection{$\mathbb{R}^2 \to \mathbb{R}$}
$(x,y) \to z =  \alpha a + \beta b $ to obrazem jest płaszczyzna przechodzącą przez $(0,0)$.

\section{Jądro i obraz odwzorowania liniowego}
\subsection{Definicja}
Jądrem odwzorowania linioweg $f : \mathbb{R}^n \to \mathbb{R}^m$ nazywamy zbiór $$ \ker f :=  \{ u \in \mathbb{R}^n : f(u) = 0_{\mathbb{R}^n}\} = f^{-1}(\overline{0}_{\mathbb{R}^m})$$
Obrazem odwzorowania liniowego $f : \mathbb{R}^n \to \mathbb{R}^m$ nazywamy zbiór $$ im(f) := \{ v \in \mathbb{R}^m : \exists\ u \in \mathbb{R}^n \quad f(u) = v \} = f(\mathbb{R}^n)$$
czyli zbiór wartości.

\section{Twierdzenie o jądrze i obrazie}
Jeśli $f : \mathbb{R}^n \to \mathbb{R}^m$ - liniowe, to:
\begin{enumerate}
\item{$\ker{f}$ jest podprzestrzenią wektorową w $\mathbb{R}^n = \mathbb{D}_{f}$}
\item{$im$ jest podprzestrzenią w  $\mathbb{R}^m = \mathbb{D}_{f}$}
\end{enumerate}
\subsection{Wniosek}
Jeśli $f:\mathbb{R}^n \to \mathbb{R}^m$ - linowe to
$$\dim{\ker{f}} \leq n$$ oraz $$ \dim{im\ f} \leq m$$
\section{Twierdzenie o generatorach obrazu}
Nich $f: \mathbb{R}^n \to \mathbb{R}^m$- linowe $B = (u_1,\ldots, u_2)$- baza w $\mathbb{R}^n$ to wtedy
$$ im\ f = \lim{\{f(u_1),f(u_2),\ldots,f(u_n)}$$
\subsection{Dowód} 
%$\\(\subset)$ oczywiste
%$\\(\subset)$ to $v \in im\ f \Righarrow u \in \mathbb{R}^m$
\subsection{Wniosek}
Jeśli $f:\mathbb{R}^n \to \mathbb{R}^m$ - liniowe, to $$\dim{ \Im{f}} \leq n$$
\section{Definicja rzędu odwzorowania}
Rzędem nazywamy odwzorowania liniwego $f:\mathbb{R}^n \to \mathbb{R}^m$ nazywamy liczbę $r(f) : \dim{ \Im{f}}$

\subsection{Przykład}
$f: \mathbb{R}^3 \to \mathbb{R}^2, \quad f(0,0,1) = (-1,3) \quad f(0,1,1) = (0,-3) \quad f(1,1,1) = (2,-2)$

Znajdź $\ker{f}, \Im{f}$ ich bazy i wymiary.
To zaczynamy po kolei.
$f(x,y,z) = (0,0)$ to faktycznie chodzi nam $f(x,y,z)$
Jeżeli znamy dla jakieś bazy powiedzmy znamy takie wartości $f(1,0,0)\ f(0,1,0)\ f(0,0,1)$ -są znane $\Rightarrow f(x,y,z) = f(x(1,0,0) + y(0,1,0) + z(0,0,1)) = xf(1,0,0) + yf(0,1,0) + zf(0,0,1)$

$f(0,1,0) = f((0,1,1) - (0,0,1)) = f(0,1,1) - f(0,0,1) = (0,-3) - (-1,-3) = (1,0)$

$f(1,0,0) = f((1,1,1) -(0,1,1)) = f(1,1,1) - f(0,1,1) = (-1,3) - (0,3) = (-1,6)$
Czyli
$$f(x,y,z) = x(2,1) + y(1,0) + z(-1,-3) = (2x+y-z,x-3z)$$
$$\ker{f} = \{(x,y,z) \in \mathbb{R}^3 : f(x,y,z) = (2x+y-z,x-3z) = (0,0)$$
Co jest równoznaczne z rozwiązaniem układu równań.
$$\begin{cases}
2x + y - z = 0 \\
x -3z = 0
\end{cases}$$
$$\ker{f} = \hbox{lin}{(3,-5,1)}, \quad \dim{\ker{f}} = 1$$

Teraz obraz
$im{f} = \{(2x + y -z, x - 3z) : x,y,z \in \mathbb{R}\} = \{x(2,1) + y(1,0) + z(-1,-3) : x,y,z \in \mathbb{R}\}= \hbox{lin}{(2,1),(1,0),(-1,-3)}$
Ale one są liniowo zależne bo mamy $\mathbb{R}^n$ i $n=2$ więc nie mogą być trzy, więc obrazem jest 
$\hbox{lim}\{(2,1),(1,0)\}$

\section{Twierdzenie o związku wymiarów jądra i obrazu}
Niech $f : \mathbb{R}^n \to \mathbb{R}^m$ - liniowe
Wtedy $$\dim{\ker} + \dim{\hbox{im}{f}} = m \dim{\mathbb{D}_f}$$
\section{Przykład reaktywacja}
$\dim{\ker{f}}= 1$ z twierdzenia wynika $\dim{\hbox{im}{f}} = 3 -1 = 2$

\section{Definicje}
Odwzorowanie $f: \mathbb{R}^n \to \mathbb{R}^m$ nazywamy
\subsection{Monomorfizmem}
wtedy i tylko wtedy gdy $f$ jest liniowe i iniektywne.
\subsection{Epimorficzne}
wtedy i tylko wtedy, gdy $f$ jest liniowe i suriektywne.
\subsection{Izomorfizmem}
wtedy i tylko wtedy, gdy $f$ jest liniowe i bijektywne.
\subsection{Endomorfizm}
wtedy i tylko wtedy, gdy $f$ jest liniowe i $m = n $.
\subsection{Automorfizm}
wtedy i tylko wtedy, gdy $f$ jest endomorfizmem i bijektywne.

\subsection{WkW na epimorfizm}
$f: \mathbb{R}^n \to \mathbb{R}^m$ - epimorfizm $\iff r(f) = m$

\subsection{WkW na monomorfizm}
$f: \mathbb{R}^n \to \mathbb{R}^m$ - monomorfizm $\iff \ker{f} = \{\overline{0}_{\mathbb{R}^n}\} \iff r(f) = n = \dim{\mathbb{D}_f}$
\subsubsection{Uzasadnienie}
$$f(u_1) = f(u_2) \Longrightarrow u_1 = u_2$$
$$f(u_1) -f(u_2) = \overline{0}_{\mathbb{R}}^m$$
$$f(u_1 - u_2) = \overline{0}_{\mathbb{R}^m}$$
\subsection{Wnioski}
$f: \mathbb{R}^n \to \mathbb{R}^m$ - izomorfizm $\iff n = m $, z zatem izomorfizm to to samo co automorfizm.
\subsection{Uwaga}
Ogólnie odwzorowania liniowe definiuje się między dowolnymi przestrzeniami wektorowymi, u nas tylko między $\mathbb{R}^n$ i $\mathbb{R}^m$ .

\section{Działania na odwzorowaniach liniowych}
\subsection{Twierdzenie}
Niech $f,g : \mathbb{R}^n \to \mathbb{R}^m,\quad \alpha \in \mathbb{R}$ . Wtedy
\begin{enumerate}
\item{$f+g : \mathbb{R}^n \to \mathbb{R}^m$}
\item{$\alpha f : \mathbb{R}^n \to \mathbb{R}^m$}
\end{enumerate}
\subsection{Twierdzenie o liniowości złożenia}
Jesłi $f: \mathbb{R}^n \to \mathbb{R}^m$ jest liniowe oraz $ g: \mathbb{R}^m \to \mathbb{R}^k$ jest liniowe, to $ g \circ f : \mathbb{R}^n \to \mathbb{R}^k$ też jest liniowe.
\subsection{Twierdzenie o liniowości odwzorowania odwrotnego}
Niech $f: \mathbb{R}^n \to \mathbb{R}^m$ jest automorfizmem.
Wtedy $f^{-1} : \mathbb{R}^n \to \mathbb{R^n}$ - automorfizm
wtedy 
$$ f\circ f^{-1} = f^{-1} \circ f = \hbox{id}_{\mathbb{R}^n} \quad \hbox{id}_{\mathbb{R}^n} : \mathbb{R}^n \ni u \to u \in \mathbb{R}^n$$

\subsection{Definicja funkcjonału}
Odwzorowanie $f:\mathbb{R}^n \to \mathbb{R}$ które jest liniowe nazywa jest formą liniową lub funkcjonałem liniowym.

\section{Macierz odwzorowania liniowego}
Niech $f:\mathbb{R}^n \to \mathbb{R}^m$ - liniowe $B_1 = (u_1,\ldots,u_n)$ -baza w $\mathbb{R}^n, \quad B_{2} = (v_1,\ldots,v_m)$-baza $\mathbb{R}^m$. Wtedy macierzą odwzorowania $f$ w bazach $B_1,B_2$ nazywamy macierz $M_f(B_1,B_2) \in M_{m \times n}$ Wtedy $j$-ta kolumna to kolumna wzpółrzędnych wektora $f(u_j)$ względem bazy $B_{2}$
$$ M_f (B_1,B_2) = \begin{matrix}
	a_1 & a_2 & 13\\
	21 & 22 & 23\\
	31 & 32 & 33\\
	a_n1 & 42 & 43\\
\end{matrix}$$
$$ f(u_1) = [a_11,a_21,\ldots,a_n1]_{B_{2}}$$
$$ f(u_2) = [a_12,a_22,\ldots,a_n2]_{B_{2}}$$

\section{Twierdzenie}
Niech $f: \mathbb{R}^n \to \mathbb{R}^m$ - liniowe , $B_1 = (u_1,u_2,\ldots,u_n)$ - baza w $\mathbb R_n, B_2 = (v_1,v_2,\ldots,v_n)$ -naza w $\mathbb{R}^n$ Niech $u \in \mathbb{R}^n \quad u = [x_1,x_2,\ldots,x_n]_{B_1} \, v \in \mathbb{R}^m, v = [y_1,y_2,\ldots,y_m]_{B_2}$
%Wtedy istnieje dokładnie jedna macierz $A \in M_{m \times n}$ (co więcej $A = M_j(B_1,B_2)$) taka, że $$ f(u) = v  \iff A \cdot \begin{matrix
%	11\\
%	21\\
%	31\\
%\end{matrix}
% = \begin{matrix}
%	11\\
%	21\\
%	31\\
%\end{matrix}$$

\subsection{Uwaga:}

\end{document}
