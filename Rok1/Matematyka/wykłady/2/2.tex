\documentclass[11pt]{article}
\usepackage[utf8]{inputenc}
%\usepackage[T1]{fontenc}
\usepackage{amssymb}
\usepackage{amsmath}
\usepackage{enumerate}
\usepackage{fullpage}
\usepackage{polski}  
\usepackage{indentfirst} 
\usepackage[pdftex]{graphicx}
\usepackage{multirow}
\usepackage{placeins}

\author{Łukasz Dubiel}
\title{Rachunek zbiorów}

\begin{document}
\maketitle

\section{Oznaczenia}
$$ A, B, C \hbox{ - zbiory }$$
$$ a, b, c \hbox{ - elementy zbioru}$$

\section{Relacje}
\subsection{Należenia}
$$ x \in A $$
Czytaj x należy do A.
\subsection{Inkluzja (zawieranie się)}
$$ A \subset B $$
czyli $$ A \subset B \iff \forall_x\ \ x \in A \implies x \in B$$

\subsection{Równość}
\section{Działania}
$$ A = B \iff A \subset B \wedge B \subset A $$
\subsection{Suma}
$$ A \cup B = \{ x : x \in A \vee x \in B\}$$
\subsection{Iloczyn}
$$ A \cap B = \{ x : x \in A \wedge x \in B\}$$
Zwany w niektórych kręgach również przecięciem.
\subsection{Różnica}
$$ A \backslash B = \{ x : x \in A \wedge x \not \in B \}$$
\subsection{Iloczyn kartezjański}
$$ A \times B = \{ (x,y) : x \in A \wedge x \in B \} $$
\subsubsection{Uwaga 1}
W iloczynie kartezjański para jest parą uporządkowaną. Dlatego nie jest przemienny czyli
$$ A \times B  \not = B \times A $$
\subsubsection{Uwaga 2}
Jeżeli $A = \o$ lub $B = \o$ to $$ A \times B = \o$$
\subsubsection{Uogólnienie}
Niech $A_1,\ldots,A_n \not = \o$ wtedy
$$ A_1 \times A_2 \times \ldots \times A_n = \left\{ (x_1,x_2,\ldots,x_n): x_i \in A_i \quad \forall i \right\} $$ 

\section{Zbiory liczbowe}
\subsection{Przedziały liczbowe w $\mathbb{R}$}
$$ (a,b) = \{ x : a < x < b \} = \{ x : x > a \wedge x < b \} $$
$$ <a,b> = \{ x : a \leq x \leq b \} $$

\section{Supermum i infinum}
\subsection{Definicja}
Zbiór $A \subset B, A \not = \o$ nazywamy ograniczonym z góry wtedy i tylko wtedy, gdy $$ \exists\ M \in \mathbb{R} \forall\ x \in A \quad x \leq M$$, każde takie $M$ nazywamy majorantem zbioru $A$.

Zbiór $A \subset B, A \not = \o$ nazywamy ograniczonym z dołu wtedy i tylko wtedy, gdy $$ \exists\ m \in \mathbb{R}\ \forall x \in A \quad x \geq m$$

Mówimy, że zbiór $A \subset \mathbb{R}, A \not = \o$ jest ograniczony jeśli jest ograniczony z góry i z dołu. 

\subsection{Supremum}
Supremum zbioru $A$ ( kres górny ) to najmniejsza z majorant zbioru i oznaczamy 
$$ \sup A$$
\subsection{Infinum}
Infinum zbioru $A \subset \mathbb{R}$ (kres dolny) to największa z minorant, zbioru $A$ oznaczane $$ \inf A$$

\end{document}