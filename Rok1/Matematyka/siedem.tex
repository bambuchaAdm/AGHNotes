\documentclass{article}
\usepackage[utf8]{inputenc}
%\usepackage[T1]{fontenc}
\usepackage{amssymb}
\usepackage{amsmath}
\usepackage{enumerate}
\usepackage{fullpage}
\usepackage{polski}  
\usepackage{indentfirst} 
\usepackage[pdftex]{graphicx}
\usepackage{multirow}
\usepackage{placeins} 

%\usepackage{rotating} 
%\addtolength{\hoffset}{0cm} \addtolength{\textwidth}{0cm}
%\addtolength{\voffset}{-0,2cm} \addtolength{\textheight}{1,2cm}
\title{Matematyka - ćwiczenia \\ siedem}
\author{Łukasz Dubiel}
\begin{document}
\maketitle
\section{Kolokwium}
\subsection{Funkcje cyklometryczne}
\subsubsection{arcsin}
\subsubsection{arccos}
\subsubsection{arctg}
\subsubsection{arcctg}
\subsection{Ddziedzina funckji}
\subsection{Nierówności}
\subsection{Indukcja}

\subsection{Granice ciągów}
\subsection{Granice funkcji}
Nie ma reguły a'Hospitala
\subsection{Ciągłość funkcji}

\section{Zestaw czwarty}
\subsection{1h}
$$f(x) = \arccos{\sqrt{1-x^2}} $$
$$f'(x) = \frac{x}{|x|\sqrt{1-x^2 }}$$

\subsection{1i}
$$f(x) = \sqrt{\ln{x}} + \arctan{\frac{1-x}{1+x}} $$
$$f'(x) = \frac{1}{2\sqrt{\ln{x}}} \frac{1}{x}  + \frac{1}{1+(\frac{1-x}{1+x})^2} * \frac{-1(x+1) - (1-x)}{(1+x)^2} $$
$$f(x) = \frac{1}{2\sqrt{\ln{x}}} + \frac{-2}{(1+x)^2 + ( 1 - x )^2} $$
$$f(x) = \frac{1}{2\sqrt{\ln{x}}} - \frac{1}{1+x^2} $$

\subsection{1j}
$$f(x) = \sqrt[n]{x} $$
Wykorzystujemy własność że $$x^x = e^{\ln{x^x}} = e^{x ln x} $$
Czyli ogólnie $$ \frac{d\ f(x)^{g(x)}}{dx} = \frac{d}{dx}\left( e^{g(x) \ln{f(x)}} \right) 
= e^{g(x) \ln{f(x)}} \frac{d\left( g(x) \ln{f(x)} \right)}{dx}$$
 
\subsection{1k}
$$f(x) = x^{\sin{x}} $$
$$f'(x) = e^{\sin{x} \ln{x}} * \left[ \cos{x}\ln{x} + \frac{\sin{x}}{x} \right] $$

\subsection{1l}
$$f(x) = {\sin{x}}^x $$
$$f'(x) = {\sin{x}}^{x} * 
\left[ \ln{\sin{x}} + x \cot{x} \right] $$

\subsection{1m}
$$f(x) = 2^{({\ln{2}})^x} $$
$$f'(x) = 2^{({\ln{2}})^x} * \ln{2} * g'(x)$$
$$g'(x) = (\ln{x})^x * \frac{d}{dx}\left[ x * \ln{\ln{x}} \right] $$
$$g'(x) = (\ln{x})^x * '\left[ \ln{\ln{x}} + x\frac{1}{\ln{x}} \frac{1}{x} \right] $$

\subsection{1n}
$$f(x) = 6^{(\tan{x})^x} $$
$$f'(x) = 6^{(\tan{x})^x} * \ln{6} * g'(x)$$
$$g'(x) = \left[(\tan{x})^x\right]' =  (\tan{x})^x *\left[ \ln{\tan{x}} + \frac{x}{\cos{x} \sin{x}}\right]$$

\newpage
\section{Zbadaj pochodną dla funkcji z definicji}
\subsection{2a}
$$f(x) = |x-1|^3 |x-2| $$
Rzobijamy na przypadki dla $ x \in (-\infty,1) $ , $x \in (1,2) $ oraz $ x\in (2,\infty) $.
\begin{displaymath}
\begin{cases}
f(x) = (1-x)^3 (2-x) \ \ \ x \in (-\infty,1) \\
f(x) = (x-1)^3 (2-x) \ \ \ x \in (1,2)	\\
f(x) = (x-1)^3 (x-2) \ \ \ x\in (2,\infty)
\end{cases}
\end{displaymath}
Liczymy pochodne lewo- i prawostronne w punktach $1,2$ 
( Dopiszcie resztę tutaj )
$$
\lim_{x \to 1^-}{f(x)}
$$
$$
\lim_{x \to 1^+}{f(x)}
$$
Oraz
$$
\lim_{x \to 2^-}{f(x)}
$$
$$
\lim_{x \to 2^+}{f(x)}
$$
\end{document}