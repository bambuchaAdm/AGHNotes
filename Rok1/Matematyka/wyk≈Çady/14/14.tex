\documentclass[11pt]{article}
\usepackage[utf8]{inputenc}
%\usepackage[T1]{fontenc}
\usepackage{amssymb}
\usepackage{amsmath}
\usepackage{enumerate}
\usepackage{fullpage}
\usepackage{polski}  
\usepackage{indentfirst} 
\usepackage[pdftex]{graphicx}
\usepackage{multirow}
\usepackage{placeins}

\author{Łukasz Dubiel}
\title{Matematyka - całkowanie funkcji pierwiastkowych }

\begin{document}

\maketitle
\section{Wielomiany wielu zmiennych}
$$ R(u,v) = \frac{P(u,v)}{W(u,v)} $$
$R(u,v)$ - funkcja wymierna zmiennych $u$ i $v$ \\
$P(u,v)$ i $W(u,v)$ - wielomiany zmiennych $u$ o $v$.

\subsection{Przykład}
$$ P(u,v) = uv^3 + u^2v^2 - uv + u$$
$$ W(u,v) = v - v^2 + uv $$
$$ R(u,v) = \frac{uv^3 + u^2v^2 - uv + u}{v - v^2 + uv} $$

\section{Całki zawierająca pierwiastki funkcji homograficznej}
Postać
$$ \int R\left(x,\sqrt[n]{\frac{\alpha x + \beta}{\gamma x + \delta}}\right) $$
Do rozwiązania tej całki używamy postawienia
$$ \sqrt[n]{\frac{\alpha x + \beta}{\gamma x + \delta}} = t $$
$$ \frac{\alpha x + \beta}{\gamma x + \delta} = t^n $$
$$ \alpha x + \beta = t^n(\gamma x + \delta )$$
$$ x(\alpha +t^n \gamma) = t^n \delta - \beta $$
$$ x = \frac{\delta t^n - \beta}{\gamma t^n  + \alpha} = \varphi(t)$$
$$ dx = \varphi'(t)dt$$
Funkcja $\varphi(t)$ jest wymierna więc można ją na pewno całkować. Używając zamiany zmiennych dostajemy

$$ \int R\left(x,\sqrt[n]{\frac{\alpha x + \beta}{\gamma x + \delta}}\right) = \int R \left( \varphi(t),t\right)\varphi'(t)dt $$
Gdzie całka po prawej stronie jest na pewno wymierna wiec możemy ją scałkować.
\subsection{Przykład 1}
$$ \int \frac{\sqrt[3]{x-1}}{x}dx$$
Tak więc stosujemy postawienie $$ \sqrt[3]{x-1} = t$$
$$ x-1 = t^3 $$
$$ x = t^3 + 1$$
$$ dx = 3t^2dt$$
Używając go na naszej całce dostajemy
$$ \int \frac{t}{t^3-1} 3t^2dt = 3 \int \frac{t^3}{t^3+1}dt = 3\int \frac{t^3 + 1 -1}{t^3+1} = 3\int dt - 3\int  \frac{dt}{t^3+1}$$
Piesza całka jest trywialna, zajmijmy się drugą. Mianownik po faktoryzacji ma postać \\ $(t+1)(t^2-t+1)$ tak więc
$$ \int \frac{dt}{t^3+1} = A \int \frac{dt}{t+1} + \alpha \int \frac{2t -1}{t^2 -t +1} + \beta \int \frac{dt}{t^2-t+1} $$
Teraz pasowałoby obliczyć współczynniki (ale możemy użyć leniwego wartościowania, a nie są nam teraz potrzebne ). 
$$ \frac{A(t^2 -t +1) + \alpha (2t -1) (t+1) + \beta(t+1)}{t^3+1} =\frac{1}{t^3+1} \xrightarrow{\hbox{ćw.}} \begin{cases} A = \frac{1}{3} \\ \alpha = \frac{1}{6} \\ \beta = \frac{1}{2} \end{cases}$$
Mając już współczynniki ( a nie musimy ich mieć, jeszcze, bo są potrzebne do finalnego rozwiązania, a całki są ważniejsze), możemy wyznaczyć resztę całek. Pierwsza z nich.
$$ \int \frac{dt}{t+1} = \ln{|t+1|} + C = \ln{|\sqrt[3]{x-1}|} + C $$
Druga
$$ \int \frac{2t -1}{t^2 -t +1} dt = \ln{ |t^2 -t +1|} + C = \ln{| (x-1)^{\frac{2}{3}} + \sqrt[3]{x-1} + 1|} + C$$
\newpage
Trzecia
$$ \int \frac{dt}{t^2 - t + 1} = \int \frac{dt}{\left(t - \frac{1}{2}\right)^2 + \frac{3}{4}} = \frac{4}{3} \int \frac{dt}{\left(\frac{t-\frac{1}{2}}{\sqrt{\frac{3}{4}}}\right)^2 + 1}$$
Stosujemy postawienie $n = \frac{t-\frac{1}{2}}{\sqrt{\frac{3}{4}}} \Rightarrow dn = \frac{1}{\sqrt{\frac{3}{4}}} dt$.
$$ \frac{4}{3} \int \frac{\sqrt{\frac{3}{4}}dn}{n^2+1} = \sqrt{\frac{4}{3}} \arctan{n} + C = \sqrt{\frac{4}{3}} \arctan{\left(\frac{\sqrt[3]{x-1}-\frac{1}{2}}{\sqrt{\frac{4}{3}}}\right)} $$
I teraz uzyskane całki postawiamy do wyniku z uwzględnieniem współczynników.
\section{Całki zawierające pierwiastki z trójmianów kwadratowych}
Postać
$$ \int R(x,\sqrt{ax^2 + bx + c})dx  \quad a \not=0$$
\subsection{Podejście naiwne}
Używamy postawienia $$ \sqrt{ax^2 + bx + c} = t $$
I tutaj trafiamy na problem bo $$ x = \frac{-b \pm \sqrt{b^2 - 4a(c-t^2)}}{2a} = \varphi(t)$$ nie jest wymierne
jak również $\varphi'(t)$ nie jest wymierne. Do chrzanu.
Ale ratuje nas pan Euler dając nam do ręki potężną broń do walki z całkami jaką są jego trzy postawienia.
\subsection{$ a > 0$ czyli pierwsze podstawienie Eulera}
Postawienie brzmi
$$ \sqrt{ax^2 + bx + c} = t - \sqrt{a}x $$
Podnosimy do kwadratu
$$ ax^2 + bx + c = t^2 -\sqrt{a}xt + ax^2 $$
Redukujemy wyrazy podobne oraz porządkujemy i wyznaczamy $x$ z równania.
$$ x = \frac{t^2 -c}{\sqrt{a}t + b} = \varphi(t)$$
I świetnie bo $\varphi(t)$ oraz $\varphi'(t)$ są wymierne. Czyli.
$$\int R(x,\sqrt{ax^2 + bx + c})dx = \int R(\varphi(t), t - \sqrt{a}\varphi(t))\varphi'(t)dt$$
\newpage
\subsubsection{Przykład - ważny}
$$ \int \frac{dx}{\sqrt{x^2-k}} , \quad k \in \mathbb{R} $$
$a = 1$, wiec możemy zastosować pierwsze podstawienie Eulera
$$ \sqrt{x^2+k} = t - x$$
$$ x^2 + k = t^2 -2tx + x^2 $$
$$ x = \frac{t^2 - k}{2t}$$
$$ dx = \frac{4+t^2}{2t^2}$$
To teraz aplikujemy do naszej całeczki.
$$\int \frac{dx}{\sqrt{x^2-k}} = \int \frac{\frac{4+t^2}{2t^2}}{t-\frac{t^2-k}{2t}} dt = \int \frac{dt}{t} = \ln{|t|} + C = \ln{|x + \sqrt{x^2+k} |} + C $$
\subsection{ $ c > 0 $  czyli drugie postawienie Eulera}
$$ \sqrt{ax^2 + bx + c} = xt + \sqrt{c} $$
Tak jak poprzednio podnosimy do kwadratu.
$$ ax^2 + bx + c = x^2t^ + 2\sqrt{c}xt + c$$
$$ (a - t^2)x^2 = (2t\sqrt{t} - b)x $$
$$ x = \frac{2t\sqrt{c} -b}{a -t^2}  = \varphi(t)$$
Tak jak poprzednio $\varphi(t)$ i $\varphi'(t)$ jest wymierne więc możemy całkować do woli.
\subsection{$a<0$ czyli trzecie postawienie Eulera}
$a < 0 \Longrightarrow \Delta \geq 0$ by umożliwić wykonywalność działania, tak więc.
$$ ax^2 + bx + c = a(x - x_1)(x - x_2)$$
Tak więc postawiamy wyrażanie
$$ \sqrt{ax^2 + bx + c} = t(x - x_1)$$
Podnosimy do kwadratu ( w wyrażeniu po prawej stronie równie dobrze mogłoby być $x_2$, zależy co jest lepsze)
$$ a(x - x_1)(x - x_2) = t^2(x-x_1)^2 $$
$$ a(x -x_2) = t^2(x-x_1)$$
$$ x = \frac{ax_2 - t^2x_1}{a - t^2} = \varphi(t)$$
Tak jak poprzednio $ \varphi(t)$ jak i $\varphi'(t)$ jest wymierne więc wolna droga do całkowania.

\section{Całki ilorazów wielomianu i pierwiastku trójmianu kwadratowego}
$$ \int \frac{W_n(x)}{\sqrt{ax^2 + bx + c}} dx \quad a\not=0$$
gdzie $ \deg{W_n} = n $ , oraz $W_n$ jest wielomianem
\subsection{Uwaga - o naszych obecnych możliwościach}
Umiemy już całkować funkcje w postaci $$\int \frac{dx}{\sqrt{ax^2 + bx + c}} $$

Algorytm
\begin{enumerate}
\item{Sprowadzić trójmian do postaci kanonicznej}
\item{W zależności od znaku $a$ przez podpowiednie postawienie sprowadzić do całki
$$ \int \frac{dt}{\sqrt{t^2+k}} \quad a > 0$$
lub
$$ \int \frac{dt}{\sqrt{1-t^2}} \quad a < 0$$}
\end{enumerate} 

\subsubsection{Przykład 1}
$$ \int \frac{dx}{\sqrt{x^2 -x + 2}} = \int \frac{dx}{\sqrt{\left(x-\frac{1}{2}\right)^2 + \frac{7}{4}}} $$
Stosując podstawienie $t = x - \frac{1}{2} \Rightarrow dt = dx$ dostajemy
$$ \int \frac{dt}{\sqrt{t^2+\frac{7}{4}}} = \ln{|t + \sqrt{t^2 + \frac{7}{4}}|} + C= \ln{|x - \frac{1}{2} + \sqrt{x^2 - x + 2}|} + C $$
Wykorzystaliśmy tutaj ważny przykład z części o postawieniach Eulera. Można to zrobić bez niego używając rzeczonego postawienia.
\subsubsection{Przykład 2}
$$ \int \frac{dx}{\sqrt{-x^2 + x + 2}} = \int \frac{dx}{\sqrt{-\left(x-\frac{1}{2}\right)^2 + \frac{9}{4}}} = \int  \frac{dx}{\frac{3}{2} \sqrt{1 -\left(\frac{x-\frac{1}{2}}{\frac{3}{2}}\right)^2}}  = \int  \frac{dx}{\frac{3}{2} \sqrt{1 -\left(\frac{2x - 1}{3}\right)^2}}$$
Używamy postawienia $ t = \frac{2x-1}{3} \Rightarrow dt = \frac{2}{3}dx$, czyli nasza całka przyjmuje
$$ \int \frac{dt}{\sqrt{1-t^2}} = \arcsin{t} + C = \arcsin{\frac{2x-1}{3}} + C$$

\subsection{Twierdzenie}
Dla dwolnego wielomianu $W_n(x)$ stopnia $n$ istnieje wielomian $P_{n-1}(x)$ stopnia $n-1$ oraz liczba $\lambda \in \mathbb{R}$ taka, że
\begin{equation} \int \frac{W_n(x)}{\sqrt{ax^2 + bx + c}}dx = P_{n-1}(x)\sqrt{ax^2 +bx +c} + \lambda \int \frac{dx}{\sqrt{ax^2+bx+c}} \label{eq:calka_2} \end{equation}
\subsection{Metoda wspólczyników nieoznaczonych}
Skoro mamy już poprzednie twierdzenie to możemy sie pokusić o wyznaczenie współczynników wielomianu $P_{n-1}$ oraz $\lambda$, aby to zrobić należy
\begin{enumerate}
\item{Różniczkujemy obustronnie równość \eqref{eq:calka_2}}
\item{Mnożymy obustronnie uzyskaną równość przez $ \sqrt{ax^2 + bx + c}$}
\item{Porównujemy współczynniki przy odpowiednich potęgach x z obu stron równości}
\end{enumerate}

To działa :)
Wykonując pierwszy krok na równości \eqref{eq:calka_2} dostaniemy.
$$ \frac{W_n(x)}{\sqrt{ax^2 + bx + c}} = \left(P_{n-1}(x)\sqrt{ax^2+bx+c}\right)' + \frac{\lambda}{\sqrt{ax^2+bx+c}} $$
Po pomnożeni przez  czynnik wspomniany w punkcie drugim mamy
$$W_n(x) = P_{n-1}'(x)(ax^2 + bx + c) + \frac{1}{2}P_{n-1}(x)(2ax+b) + \lambda $$

Podobnie jak w przypadku całkowania funkcji wymiernych mam do rozwiązania jakiś układ równań.

\subsection{Przykład 1}
$$ \frac{8x + 6}{\sqrt{2x^2 + 8x}}dx = A \sqrt{2x^2 + 8x} + \lambda \int \frac{dx}{\sqrt{2x^2 + 8x}}dx $$
\subsection{Przykład 2}
$$ \int \sqrt{9 - x^2}dx = \int \frac{9 - x^2}{\sqrt{9-x^2}} dx = (Ax + B)\sqrt{9-x^2} + \lambda \int \frac{dx}{\sqrt{9-x^2}}$$

\end{document}
