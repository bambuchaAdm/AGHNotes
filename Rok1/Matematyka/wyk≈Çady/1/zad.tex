\documentclass[11pt]{article}
\usepackage[utf8]{inputenc}
%\usepackage[T1]{fontenc}
\usepackage{amssymb}
\usepackage{amsmath}
\usepackage{enumerate}
\usepackage{fullpage}
\usepackage{polski}  
\usepackage{indentfirst} 
\usepackage[pdftex]{graphicx}
\usepackage{multirow}
\usepackage{placeins}

\author{Łukasz Dubiel}
\title{Elementy Logiki}

\begin{document}

\maketitle

\section{Funktory logiczne}
\subsection{Zdania logiczne}
$$p,q$$
\subsection{Zaprzeczenie}
$$ \neg p $$
\subsection{Koniunkcja}
$$ p \vee q $$
\subsection{Alternatywa}
$$ p \wedge q$$
\subsection{Implikacja}
$$ p \implies q $$
gdzie $p$ to założenia, $q$ to teza. Oraz $q$ jest warunkiem koniecznym dla $p$ oraz $p$ jest warunkiem wystarczającym dla $q$.
$$ \neg(p \implies q) \equiv p \wedge \neg q$$
\subsection{Zasada Kontrapozycji}
$$ p \implies q \iff (\neg q \implies \neg p )$$
\subsection{Rownoważność}
$$ p \iff q$$
Czytamy jako $p$ wtedy i tylko wtedy (wtw), gdy $g$

$$ p \iff q \equiv (p \implies q) \wedge (q \implies p)$$
\section{Prawa de'Morgana}
$$\neg(p \wedge q) \equiv \neg p \vee \neg q$$
$$\neg(p \vee q) \equiv \neg p \wedge \neg q$$
\section{Kwantyfikatory}
$$ \forall \quad \hbox{Kwantyfikator duży "dla każdego"} $$
$$ \exists \quad \hbox{Kwantyfikator mały "istnieje taki"}$$
%$$ \forall  \quad \hobx{Kwantyfikator "dla prawie wszystkich"} $$
%$$ \exists \quad \hobx{Kwantyfikator "istnieje dokładnie jeden taki"}$$
\subsection{Przeczenie kwantyfikatorów}
$$ \neg( \forall_x \quad \Phi(x)) \iff \exists \neg \Phi(x) $$
$$ \neg( \exists_x \quad \Phi(x)) \iff \forall \neg \Phi(x) $$
\subsection{Kolejność kwantyfikatorów}
Kolejność kwantyfikatorów ma znaczenie, i nie można zamieniać jej w dowolny sposób
$$ \exists_y \forall_x\ \Phi(x,y) \implies \forall_x \exists_y\ \Phi(x,y) $$
Układ po lewej oznacza, że istnieje jeden taki "uniwersalny" $y$ dla wszystkich $x$-ów. Drugi oznacza, że dla każdego $x$'a możemy znaleźć takiego $y$ z którym będzie mu dobrze. Ale tych $y$-ków może być dużo i różnych. Dlatego zapis po lewej stronie jest mocniejszy.

\end{document}
