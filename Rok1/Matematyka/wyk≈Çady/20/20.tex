\documentclass[11pt]{article}
\usepackage[utf8]{inputenc}
%\usepackage[T1]{fontenc}
\usepackage{amssymb}
\usepackage{amsmath}
\usepackage{enumerate}
\usepackage{fullpage}
\usepackage{polski}  
\usepackage{indentfirst} 
\usepackage[pdftex]{graphicx}
\usepackage{multirow}
\usepackage{placeins}

\author{Łukasz Dubiel}
\title{Macierze}
\date{10 stycznia}

\begin{document}

\maketitle
\section{Definicja}
Macieżą o wymiarach $m$ x $n$ nazywamy każde odwzorowanie 
$$ A:\{1,\ldots,n\} \times \{1,\ldots,n\} \to \mathbb{R}$$
Wartość tego odwzorowania dla elementów $(i,j)$ oznaczamy $a_{i,j}$ i nazywamy elementami macierzy $A$. Każde takie odwzorowanie utożsamiamy z zapisem :
$$\begin{bmatrix} 
a_{1,1} & a_{1,2} & \ldots & a_{1,n} \\
a_{2,1} & a_{2,2} & \ldots & a_{2,n} \\
\vdots & \vdots & a_{i,j} & \vdots\\
a_{m,1} & a_{m,2} & \ldots & a_{m,n} 
\end{bmatrix}$$

O elementach $a_{i,1},\ldots,a_{i,n}$ mówimy, że tworzą wiersze.\\
O elementach $a_{1,j},\ldots,a_{m,j}$ mówimy, że tworzą $j$-tą kolumnę \\
Piszemy $A \in M_{m \times n}$ - zbiór macierzy $m \times n$
\section{Rodzaje macierzy}
\begin{enumerate}
\item{Macierz zerowa Niech $A = [a_{ij}]_{m \times n}$ $$ \forall\ i,j \quad  a_{ij} = 0 $$}
\item{Macierzą kwadratową\\
Macierzą kwadratową nazywamy taką macierz $ A = [a_ij]_{m \times n} $, że $m = n$. Mówimy, że $A$ jest stopnia $n$. O elementach $a_{11},a_{22},\ldots,a_{nn}$ mówimy, że towrzą przekątną główną.}
\item{Macierz trójkątna \\
Macierzą trójkątną górną (dolna) nazywamy macierz kwadratową w której wszystkie elementy powyżej (poniżej) przekątnej głównej są zerami.}
\item{Macierzą diagonalną nazywamy macierz kwadratową, która jednocześnie jest trójkątną górną i dolną.}
\item{Macierzą jednostkową nazywamy macierz diagonalną gdzie $$ \forall\ i \quad a_{ii} = 0 $$ oznaczamy również $ I = 0 $}
\end{enumerate}


\subsection{Uwaga}
Z definicji wynika, że elementu są symetyczne względem przekątnej są takie same.
\section{Wyznacznik macierzy kwadratowej}
\subsection{Wstęp w $\mathbb{R}^2$}
Mamy wyznacznik $$\begin{bmatrix} a_{1,1} & a_{1,2} \\ a_{2,1} & a_{2,2} \end{bmatrix} = a_{1,1}\cdot a_{2,2} - a_{1,2} \cdot a_{2,1}$$
Jest to wartość (z dokładnością do znaku) jest równe polu równoległoboku rozpiętego na wektorach $(a_{1,1},a{1,2})$ i $(a_{2,1},a_{2,2})$
Przykład geometryczny.
%grafika z strony http://oen.dydaktyka.agh.edu.pl/dydaktyka/matematyka/a_algebra_analiza/algelin2/node215.html , jest w katalogu
$$ S - (a_{1,2} + a_{2,2})(a_{1,1} + a_{2,1}) - 2\cdot a_{1,2} a_{2,1} - a_{2,1}a_{2,2} - a_{1,1}a_{1,2} = \begin{Vmatrix}
	11 & 12\\
	21 & 22\\
\end{Vmatrix}$$
\subsection{Wstęp w $\mathbb{R}^3$}
%rysunek rółnoległościanu na trzech wektorach
\subsection{Wstęp w $\mathbb{R}^n$}
Mamy wektory $v_1,v_2,\ldots,v_n$ - wektory liniowo niezależne rozpinająca $n$-wymiarowy równoległościan $\Omega$ wtedy 
$$ |\Omega| = \begin{Vmatrix}
	\hbox{współrzędne } v_1 \\
	\hbox{współrzędne } v_2 \\
	\vdots \\
	\hbox{współrzędne } v_n \\
\end{Vmatrix}$$
\subsection{Definicja}
Wyznacznikiem macieży kwadratowej $$ A = \begin{bmatrix}
	11 & 12 & 13\\
	21 & 22 & 23\\
	31 & 32 & 33\\
\end{bmatrix}$$
którego wyznacznik oznaczamy $ \det{A} $ lub $$\begin{vmatrix}
	11 & 12 & 13\\
	21 & 22 & 23\\
	31 & 32 & 33\\
\end{vmatrix}$$
nazywamy liczbę określoną następująco :
\begin{enumerate}
\item{dla $n=1$ $\det{A} = a_{1,1}$}
\item{dla $n \geq 2$ $$ \det{A} = \begin{vmatrix}
	11 & 12 & 13\\
	21 & 22 & 23\\
	31 & 32 & 33\\
\end{vmatrix} = \sum_{k=1}^{n}{(-1)^{k+1} a_{1,k} \det{A_{1,k}}}$$
Gdzie $A_{i,j}$ - to macierz powstała po wykreśleniu $i$-tego wiersza oraz $j$-tej kolumny.}
\end{enumerate}
\subsection{Specyficzne przypadki}
\begin{enumerate}
\item{n=2}
\item{n=3}
\end{enumerate}
\subsection{Reguła Sarrusa}
Stosujemy 
% Powinna być grafika ilutrująca regułę sarrusa.
\subsection{Jak liczyć wyznacznik}
\subsection{Minor}
\subsubsection{Definicja}
Minorem stopnia $k$ macierzy $A_{m \times n}$ nazywamy wyznacznik dowolnej pod macierzy kwadratowej kwadratowej stopnia $k ( 0 < k <\leq \min{m,n})$, czyli powstałej z $A$ poprzez wykreślenie $(m-k)$ wierszy oraz $(n-k)$ kolumn.
\subsubsection{Przykład}
$$\begin{bmatrix}
	0 & 2 & -3 & 5\\
	1 & -2 & 4 & -5\\
	-1 & 3 & -4 & -6\\
\end{bmatrix}$$
Minorów stopnia 3 jest 4.
Minorów stopnia 2 jest ${4 \choose 2} \cdot 3 = 18$
\subsection{Minor dopowiadający}
Jeśli $ A \in M_{m \times n}$, to wyznacznik podmacierzy powstałej z $A$ przez wykreślnie $i$-tego wierszy oraz $j$-tej kolumny ( tę podmacierz oznaczamy $A_{i,j}$) nazywamy minorem odpowiadającym elementowi $a_{i,j}$. Oznaczamy jako $M_{i,j} := \det A_{i,j}$.
\subsection{Twierdzenie o rozwinięciu Laplace'a}
Jeśli $A \in M_{n \times n}$, to prawdziwe są następujące wzory.
\begin{enumerate}
\item{Rozwinięcie względem $i$-tego wiersza}
$$ \det A = a_{i1} D_{i1} + a_{i2} D_{i2} + \ldots + a_{ib}D_{in}$$
\item{Rozwinięcie względem $j$-tej kolumny}
$$ \det A = \sum_{k=1}^m{a_1k D_1k} $$
\end{enumerate}
Uwaga: Pierwszy wzór dla $i=1$ to dokładnie definicja wyznacznika.
\section{Przykład}
$\begin{vmatrix}
	1 & 2 & 3 & 4\\
	-1 & -1 & 0 & 1\\
	0 & 0 & -1 & 0\\
	4 & 1 & 0 & 2\\
\end{vmatrix} = ( -1) (-1)^{3+3} \begin{vmatrix}
	1 & 2 & 4\\
	-1 & -1  & 1\\
	4 & 1  & 2\\
\end{vmatrix} = -21
$
\subsection{Wioski z twierdzenia Laplaca'a}
Wyznacznik macierzy trójkątnej dolnej (górnej) jest równa iloczynowi elementół na przekątnej głównej.
\subsubsection{Wniosek z wniosku}
$$ \det I_{n\times n} = 1$$
\subsection{Przykłady}
$\begin{vmatrix}
	1 & 2 & 3 & 4\\
	5 & 6 & 7 & 8\\
	9 & 10 & 11 & 12\\
	13 & 14 & 15 & 16\\
\end{vmatrix}$
albo
$\begin{vmatrix}
	-5 & 2 & 3 & 4 & 5\\
	1 & -4 & 3 & 4 & 5\\
	1 & 2 & -3 & 4 & 5\\
	1 & 2 & 3 & -2 & 5\\
	1 & 2 & 3 & 4 & -1\\
\end{vmatrix}$
\subsection{Twierdzenie o wyznaczniku macierzy transponowanej}
$$ \forall \ A \in M_{n \times n} \quad \det (A^{T}) = \det A$$
\subsection{Twierdzenia o własnościach wyznacznika macierzy}
Niech $A \in M_{n \times n},\ A=[a_{ij}]_{n\times n} = [ k_1,k_2,\ldots,k_n]$ - mowa o kolumnach $k_1,k_2,\ldots ,k_n$
\begin{enumerate}
\item{$\det [ k_1,\ldots,0,\ldots,k_n] = 0$}
\item{$\det [ k_1,\ldots,k_j' + k_j'',\ldots,k_n] =
 [ k_1,\ldots,k_j' +,\ldots,k_n] + \det[ k_1,\ldots, k_j'',\ldots,k_n] $}
 \item{$\det [k_1,\ldots,\alpha k_j , \ldots , k_n] = \alpha \det[k_1,\ldots,k_j,\ldots, k_n]$}
 \item{$\det(\alpha A) = \alpha^n \det A \quad \forall \ \alpha \in \mathbb{R}$}
 \item{$\det [k_1,\ldots,k_s,\ldots,k_t,\ldots,k_n] = - \det [k_1,\ldots,k_t,\ldots,k_s,\ldots,k_n]$}
 \item{$\det [k_1,\ldots,k_j,\dots,k_j,\ldots,k_n]$}
 \item{$\det [k_1,\ldots,\alpha_1 k_1 + \ldots + \alpha_n k_n, \ldots, k_n] = 0$}
 \item{$\det [k_1,\ldots,k_j + \alpha_1 k_1 + \ldots + \alpha_n k_n, \ldots, k_n] = \det [ k_1,\ldots,k_j,\ldots,k_n]$}
 \end{enumerate}
 \subsection{Operacje elementarne na wierszach (kolumnach=) macierzy}
 \begin{enumerate}
 \item{Zamiana miejscami wieszy (kolumnami) macierzy.}
 \item{Dodanie do wiesza ( kolumny) kombinacji liniowej innych wiesz(kolumn)}
 \item{Pomnożenie wiesza przez skalar}
 \end{enumerate}
 \subsection{Jedziemy  z trasznymi przykładami}
$\begin{vmatrix}
	1 & 2 & 3 & 4\\
	5 & 6 & 7 & 8\\
	9 & 10 & 11 & 12\\
	13 & 14 & 15 & 16\\
\end{vmatrix}$ odejmujemy pierwszy wiesz od drugiego i trzeci od czawrtego, po czym dostajmy wa wiesze 
\subsection{Twierdzenie Coucheg'ego}
$$ \forall\ A,B \in M_{n\times n} = \det(A \cdot B) = \det{(A)} \cdot \det{(B)} $$

\end{document}
