\documentclass[11pt]{article}
\usepackage[utf8]{inputenc}
%\usepackage[T1]{fontenc}
\usepackage{amssymb}
\usepackage{amsmath}
\usepackage{enumerate}
\usepackage{fullpage}
\usepackage{polski}  
\usepackage{indentfirst} 
\usepackage[pdftex]{graphicx}
\usepackage{multirow}
\usepackage{placeins}

\author{Łukasz Dubiel}
\title{Całkowanie funkcji wymiernych}
\date{22 listopada 2011}

\begin{document}

\maketitle
\section{Ułamki podstawowe}
\subsection{Definicja}
Ułamkami prostymi nazywamy funkcje wymierne następującej postaci.
\subsubsection{Ułamki proste pierwszego rodzaju}
$$\frac{1}{(x + \alpha)^n} \quad \alpha \in \mathbb{R}, n \in \mathbb{N}^+$$
\subsubsection{Ułamki proste drugiego rodzaju}
$$\frac{T'(x)}{(T(x))^n} \quad n \in \mathbb{N}^+ $$
T(x) - trójmian kwadratowt nierozkładalny ( $ T(x) = ax^2 + bx + c $ , $a \not=0 $, $\Delta = b^2 - 4ac < 0$
\subsubsection{Ułamki proste trzeciego rodzaju}
$$\frac{1}{(T(x))^n} \quad n \in \mathbb{N}^+$$
T(x) - j.w. 
\subsection{Przykład ułamków pierwszego rodzaju}
$$\int \frac{1}{(x+\alpha)^n}dx$$
Stosujemy zamianę zmiennych $ t = x + \alpha $
$$\int \frac{1}{t^n}dt $$
Co jest równoważne
\begin{displaymath}
\begin{cases}
\ln{|x+\alpha|} , \quad n = 1 \\
\frac{1}{1-n}(x+\alpha)^{1-n}, \quad n \geq 2
\end{cases}
\end{displaymath}

\subsection{Przykład ułamków drugiego rodzaju}
$$\int \frac{T'(x)}{(T(x))^n}dx $$
Ponownie stosujemy zamianę zmiennych $ t = T(x) $
$$\int \frac{dt}{t^n}$$
czyli
\begin{displaymath}
\begin{cases}
\ln{|T(x)|} , \quad n = 1 \\
\frac{1}{1-n}(T(x))^{1-n}, \quad n \geq 2
\end{cases}
\end{displaymath}

\subsection{Przykład ułamków trzeciego rodzaju}
$$\int \frac{1}{(T(x))^n}dx = \int \frac{dx}{(ax^2 + bx + c)^n} $$
Sprowadzamy $T(x)$ do postaci kanonicznej. 
$$ax^2 + bx + c = a\left[\left(x+\frac{b}{2a}\right)^2 + r^2\right]  =  ar^2 \left[\left(\frac{x + \frac{b}{2a}}{r}\right)^2 + 1\right]$$
$$\int \frac{dx}{(ax^2 + bx + c)^n} = \int \frac{dx}{(ar^2)^n\left[\left(\frac{x + \frac{b}{2a}}{r}\right)^2 + 1\right]^n} $$
I teraz stosujemy podstawienie $\frac{x+\frac{b}{2a}}{r} = t$
I teraz dostajemy
$$I_n = \int \frac{dt}{(t^2+1)^n}$$

Gdy $n = 1$ \bigskip

$I_1 = \arctan{t} + C$

\bigskip
gdy $n \geq 2$
$$I_n = \int \frac{1+t^2 - t^2}{(t^2+1)^n}dt = \int \frac{dt}{(1+t^2)^{n-1}} + \int \frac{-t^2}{(1+t^2)^n}dt = I_{n-1} - \frac{1}{2} \int \frac{2t^2}{(1+t^2)^n}dt $$
$$\int \frac{2t t}{(2+t^2)^n}dt$$
Używamy zamiany zmiennych gdzie $ u = t$ , $v' = \frac{2t}{(1+t^2)^n} \to u' = 1$, $v = \frac{(t^2+1)^{1-n}}{1-n}$
$$\frac{t(1+t^2)^{1-n}}{1-n} - \int \frac{(1+t^2)^{1-n}}{1-n}dt  = \frac{t(1+t^2)^{1-n}}{1-t} - \frac{1}{1-n}\int\frac{dt}{(1+t^2)^{n-1}}$$
Ostatecznie dostajemy
$$I_n = I_{n-1} - \frac{t}{(n-1)(1+t^2)^{n-1}} - \frac{1}{2(n-1)}I_{n-1} $$
Czyli po sprowadzeniu do wspólnego mianownika
$$I_n = \frac{2n -3}{2n -2}I_{n-1} - \frac{t}{(n-1)(1 + t^2)^{n-1}} $$

\subsection{Dygresja}
$$\frac{1}{x^2} + \frac{1}{x} + \frac{2}{x-1} + \frac{2x}{x^2+1} + \frac{1}{(x^2 +1)^2} = \frac{}{(x^2(x-1)(x^2+1)^2}$$
\subsection{Fakt ( na razie) - Zasadnicze twierdzenie algebry}
Dowolny wielomian o współczynikach rzeczywistych można rozłożyć na iloczyn dwumianów liniowych oraz nierozkładalnych trójmianów kwadratowych.

\subsection{Definicja - funkcja wymierna właściwa}
Funkcję wymierną $\frac{L(x)}{M(x)} = f(x)$ nazywamy funkcją wymierną właściwą, wtedy i tylko wtedy gdy stopień licznika jest mniejszy niż stopień mianownika, czyli $ \deg L < \deg M $

\subsection{Twierdzenie}
Każda funkcja wymierna jest sumą wielomianu i funkcji wymiernej właściwej ( wystarczy podzielić licznik przez mianownik, gdy $\deg L > \deg M$, a z kolei funkcja wymierna właściwa jest kombinacją liniową ( sumą z ewentualnymi stałymi ) ułamków podstawowych.

\subsection{Algorytm całkowania funkcji wymiernych}
\begin{enumerate}
\item{Dzielimy licznik przez mianownik} 
Jeśli się da $W(x) = P(x)Q(x) + R(x)$
$$\int \frac{W(x)}{Q(x)} = \int P(x) dx + \int \frac{R(x)}{Q(x)}dx $$
\item{Mianownik funkcji wymiernej właściwej na czynniki pierwsze ( rozkładamy $Q(x)$)}
\item{$\frac{R(x)}{Q(x)}$ przedstawiamy jako kombinację liniową ułamków podstawowych}
Zasady rozkładu ( bo przecież na każdej wojnie są jakieś zasady )\\
-czynikowm $(x+\alpha)^n$ odpowiada n ułamków podstawowych pierwszego rodzaju : $$\frac{A_1}{x+\alpha} + \frac{A_2}{(x+\alpha)^2} + \frac{A_3}{(x+\alpha)^3} + \frac{A_{n-1}}{(x+\alpha)^{n-1}}$$
Czynnikom $(T(x))^n$, gdzie $T(x) = ax^2 + bx + c$, odpowiada $2n$ ułamków podstawowych II i III rodzaju w postaci :
$$\frac{\alpha_1}{T(X)} + \frac{\beta_1 T'(x)}{T(x)} + \frac{\alpha_2}{(T(X))^2} + \frac{\beta_2 T'(x)}{(T(x))^2} + \frac{\alpha_3}{(T(X))^3} + \frac{\beta_3 T'(x)}{(T(x))^3} + \ldots + \frac{\alpha_n}{(T(X))^n} + \frac{\beta_n T'(x)}{(T(x))^n}   $$
\item{Scałkować wszystkie ułamki oraz wielomian i dodać wynik}
\end{enumerate}

\subsection{Uwaga - o stopniu mianownika i ilości ułamków}
Zatem $\frac{R(x)}{Q(x)}$ jest sumą $k$ ułamków podstawowych, gdzie $ k = \deg Q(x) $

Bo $$Q(x) = (x+a_1)^n_1 ... (x + a_l)^{n_l}(T_1(x))^{m_1}(T_s(x))^{m_s}  $$
$$ n_1 + n_2 + \ldots + n_l + 2m_1 + 2m_2 + \ldots + 2m_s = \deg Q(x) $$

\subsection{Przykład}
$$\int \frac{-x^4 + 5x^2 + 8x + 20}{x^4 - 2x^3 + 5x^2}dx$$
Dzielimy wielomian
$$-\int1dx + \int \frac{-2x^3 + 10x^2 - 8x +20}{x^2(2x^2 -2x +5)}dx $$
Rozkładamy funkcję wymierną na ułamki podstawowe
$$\frac{-2x^3 + 10x^2 - 8x +20}{x^2(2x^2 -2x +5)} = \frac{A_1}{x} + \frac{A_2}{x^2} + \frac{\alpha_1}{x^2-2x+5} + \frac{\beta(2x-2)}{x^2-2x+5} $$
Po sprowadzeniu prawej strony do wspólnego mianownika
dostajemy
$$-2x^3+10x^2-8x+20 = (A_1 +2\beta_1)x^3 + (-2A_1 + A_2 + \alpha_1 - 2\beta_1)x^2 + (5A_1-2A_2)x + 5A_2$$
Dostajemy układ równań
\begin{displaymath}
\begin{cases}
(A_1 +2\beta_1) = -2\\
(-2A_1 + A_2 + \alpha_1 - 2\beta_1) = 10 \\
(5A_1-2A_2) = 8 \\
 5A_2 = 20
\end{cases}
\end{displaymath}
Rozwiązaniem to 
\begin{displaymath}
\begin{cases}
A_1 = 0\\
A_2 = 4 \\
\alpha_1 = 4 \\
\beta_1 = -1
\end{cases}
\end{displaymath}
Czyli naszym wynikiem jest
$$-x + 4\int \frac{dx}{x^2}+4\int \frac{dx}{x^2-2x+5}-\int \frac{2x-2}{x^2-2x+5}dx = -x -\frac{4}{x} - \ln{|x^2-2x+5|} + 4\int \frac{dx}{x^2-2x + 5}$$
Wyznaczmy ostatnią całkę 
$$\int \frac{dx}{x^2-2x + 5} = \int \frac{dx}{(x-1)^2-4} = \frac{1}{4}\int \frac{dx}{(\frac{x-1}{2})^2+1}$$
Stosując zamianę zmiennych $t = \frac{x-1}{2}$ dostajemy
$$\frac{1}{2}\int\frac{dx}{t^2+1} = \frac{1}{2}\arctan{t} + C = \frac{1}{2} = \frac{1}{2}\arctan{\frac{x-1}{2}} + C $$
Tak więc wynikiem jest ostatecznie.
$$-x -\frac{4}{x} - \ln{|x^2-2x+5|} + \frac{1}{2}\arctan{\frac{x-1}{2}} + C $$

\subsection{Przykład 2}
$$\int \frac{6x-1}{x^2+2x+3}dx  $$
\end{document}
