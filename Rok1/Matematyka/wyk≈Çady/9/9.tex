\documentclass{article}
\usepackage[utf8]{inputenc}
%\usepackage[T1]{fontenc}
\usepackage{amssymb}
\usepackage{amsmath}
\usepackage{enumerate}
\usepackage{fullpage}
\usepackage{polski}  
\usepackage{indentfirst} 
\usepackage[pdftex]{graphicx}
\usepackage{multirow}
\usepackage{placeins} 

\title{Matematyka - wykład 10}
%\author{Lech Adamus}
\begin{document}
\maketitle
\section{Ekstrema funkcji}
\subsection{Definicja}
Niech $f$-funckja rzeczywista, że $u(x_0,r) \subset  \mathbb{D}_f$ dla pewneg $r > 0$.\\
Mówimy, że $f$ ma w $x_0$ minimum lokalne (silne), wtedy i tylko wtedy, gdy istnieje $0<r_1<r$ , takie że
dla wszystkich $$\forall x \in S(x_0,r_1) \ f(x) \geq f(x_0) \quad ( f(x) > f(x_0))$$.

Niech $f$-funckja rzeczywista, że $u(x_0,r) \subset  \mathbb{D}_f$ dla pewneg $r > 0$.\\
Mówimy, że $f$ ma w $x_0$ maksimum lokalne (silne), wtedy i tylko wtedy, gdy istnieje $0<r_1<r$ , takie że
$$\forall x \in S(x_0,r_1) \ f(x) \leq f(x_0) ( f(x) < f(x_0)$$.

Mówimy, że ff ma ekstremum lokalne w $x \in \mathbb{R}$, wtedy i tylko wtedy, gdy ma tam minimum lub maksimum.
\subsection{Twierdzenia Fermata (WK istnienia ekstremum lokalnego)}
Niech $f \in \mathbb{D}(x_0)$

Jeśli $f$ ma ekstremum w $x_0$, to $f'(x) = 0$

\bigskip
Uwaga 1: Twierdzenie odwrotne nie jest prawdziwe.

Uwaga 2: Funkcja ciągła może zatem mieć ekstremum lokalne w punktach których zeruje się pochodna, lub w których pochodna nie istnieje. Punkty te ( podejrzane o ekstremum ) nazywamy stacjonarnymi.

\subsection{ Pierwszy warunek wystarczający istnienia ekstremum }
Niech $f$-funkcja rzeczywista taka, żę $u(x_0,r) \subset \mathbb{D}_f$ dla pewnego $r>0 , x \in \mathbb{R}$, oraz $f \in C(x_0)$, a także $f \in (s(x_0,r_1))$ dla pewnego $0<r_1<r$.

1) Jeśli $f'(x) < 0$ [dla każdego $x \in (x_0 - r_1 , x_0)$] oraz $f'(x) > 0$ [dla każdego $x \in (x_0,x_0+r_1)$] to $f$ ma minimum lokalne w $x_0$

2) Jeśli $f'(x) > 0$ [dla każdego $x \in (x_0 - r_1 , x_0)$ ] oraz $f'(x) < 0$ [dla każdego $x \in (x_0,x_0+r_1)$] to $f$ ma maksimum lokalne w $x_0$

\subsection{ Drugi warunek wystarczający istnienia ekstremum}
Niech $f \in C^{n+1}(I)$, i $ I\subset \mathbb{R} $ - przedział otwarty, $ x_0 \in I $ oraz $ f^{(n)}(x_0) = 0 $ [dla każdego $k \in {1..n}$ ] oraz $2|n$.

1)Jeśli $f^{(n+1)}(x_c) > 0$ to f ma minimum lokalne w $x_0$

2)Jeśli $f^{(n+1)}(x_c) < 0$ to f ma maksimum lokalne w $x_0$

\newpage
\subsubsection{Szkic dowodu}
Z wzoru Taylora
$$f(x_0 + h) = f(x_0) + $$ [Duże zero] + $$\frac{f^{(n+1)}(x + \theta h)}{(x+1)!}h^{n+1}$$

Znak $(f^{(n+1)}(x_0 + \theta h)$ decyduje o znaku $f(x_0 +h) - f(x)$

\subsection{ Drugi warunek wystarczający istnienia ekstremum - wersja szczególna}

Niech $f \in C^2{I}, I \subset \mathbb{R}$ -przedział otwarty, $x_0 \ in I$ oraz $f'(x_0) = 0$

1)Jeśli $f''(x_0) > 0$ , to $f$ ma minimum lokalne w $x_0$

2)Jeśli $f''(x_0) < 0$ , to $f$ ma maksimum lokalne w $x_0$

Uwaga 1: W praktyce, przy wyznaczaniu ekstremów lokalnych funkcji:
\begin{enumerate}
\item{Wyznaczamy punkty stacjonarne}
\item{Dla każdego nich sprawdzamy WW}
\end{enumerate}

\subsubsection{Przykład 1}
Wyznacz ekstrema lokalne
$$f(x) = \sqrt[3]{6x^2 - x^3}, \ \ \mathbb{D}_f = \mathbb{R}$$
$$f'(x) = \frac{1}{3}(6x^2 -x^3)^{-\frac{2}{3}}(12x -3x^2) = \frac{4x-x^2}{\sqrt[3]{6x^2 - x^3}}, \mathbb{D}'_f = \mathbb{R} \backslash \{0,6\} $$

$ f'(x) = 0 $ <=> $(4x-x^2) = 0$  dostajemy $x=4\ v\ x=0$

Mamy trzy punkty stacjonarne $\{ 0,4,6 \}$. Pochodna zmienia znak w punktach $0$ i $4$, ale nie zmienia w $6$.
Zatem w zero mamy minimum lokalne równe 0, w 4 mamy maksimum równe $2\sqrt{2}$

\subsubsection{Przykład 2}
$$f(x) = \cos{x} - \frac{1}{3}\cos{2x}, \ \ x \in <0,\frac{\pi}{2}>$$
$$f'(x) = -\sin{x} - \frac{2}{3}\sin{2x} = -\sin{x} + \frac{2}{3}2\sin{x}\cos{x} = \sin{x}\left(\frac{4}{3}\cos{x} -1\right) = 0$$
$$ x = 0 \not\in \mathbb{D}_f \vee x = \arccos{\frac{3}{4}}$$
Korzystamy z 2WW
$$f''(x) = -cos(x) + \frac{4}{3}\cos{2x} = \frac{2}{3}\cos^2{x} - \cos{x} - \frac{3}{4}$$
$$f''(\arccos{\frac{3}{4}}) = -\frac{7}{12} < 0$$
funkcja ma max lokalne w punkcie $\arccos{\frac{3}{4}}$

\subsection{Ekstremum globalne}
Problem:
$ X \subset \mathbb{R} $\\
$ f:X \to \mathbb{R}$ - ciągła \\
Chcemy znaleźć wartość największą i najmniejszą $f$ na $X$.

Z twierdzenia Wererstrassa jeśli $X = <a,b>$, to problem powyższy ma rozwiązanie.

\subsubsection{Algorymt}
\begin{enumerate}
\item{Znaleźć wszystkie punkty stacjonarne}
\item{Obliczyć wartości w punktach stacjonarnych oraz na brzegach przedziałów}
\item{Porównać wartości i wybrać największą i najmniejszą }

\section{Zadania optymalizacyjne}
\subsection{Przykład}
W kulę o promieniu $R > 0$ wpisać walec o największej objętości.

Oznaczmy $H$ - wysokość walca , $r$ - promień walca
Z twierdzenia Pitagorasa mamy
$$ R^2 = r^2 + \left(\frac{H}{2}\right)^2 $$
$$ H = \sqrt{4R^2 - 4r^2}$$
$$ V = \pi r^2 H$$
$$ V(r) = \pi r^2 \sqrt{4R^2 - 4r^2} $$
Szukamy maksimum globalnego dla V dla przedziału $(0,R)$
$$ V'(x) = 2\pi r \sqrt{4R^2 - 4r^2} + \pi r^2  \frac{-4r}{\sqrt{4R^2 - 4r^2}} = \frac{4\pi r ( 2R^2 - 3r^2)}{\sqrt{4R^2 - 4r^2}} $$
$V(r)$ - różniczkowalna w $(0,R)$
$$V'(r) = 0 $$
Rozwiązania to $r = 0 \vee 2R^2 = 3R^2$ czyli tylko $r=\sqrt{\frac{2}{3}}R$ jest punktem stacjonarnym.
Mianownik jest dodatni więc znak pochodnej zmienia się jak funkcja kwadratowa.

$V(r)$ osiąga maksiumum dla $r = \sqrt{\frac{2}{3}}R$
,punkt r jest globalny bo jest tylko jeden punkt stacjonarny.
\end{enumerate}
\end{document}