\documentclass{article}
\usepackage[utf8]{inputenc}
%\usepackage[T1]{fontenc}
\usepackage{amssymb}
\usepackage{amsmath}
\usepackage{enumerate}
\usepackage{fullpage}
\usepackage{polski}  
\usepackage{indentfirst} 
\usepackage[pdftex]{graphicx}
\usepackage{multirow}
\usepackage{placeins}


%\usepackage{rotating} 
%\addtolength{\hoffset}{0cm} \addtolength{\textwidth}{0cm}
%\addtolength{\voffset}{-0,2cm} \addtolength{\textheight}{1,2cm}
\author{Łukasz Dubiel}
\title{Matematyka - ćwiczenia 6}
\begin{document}
\maketitle

$$f(x) = x \sin{\frac{1}{x}}\ \ \  \mathbb{D} = \mathbb{R} - \{0\} $$
$$\lim_{x \in 0^+}{x \sin{\frac{1}{x}}} $$
Z tego że $\sin{x}$ jest ograniczony to granica jest zero.
Z prawej strony jest tak samo przez co istaniejej granica.
$$\lim_{ x \to 0}{f(x)} = 0 $$
Ale funckja nie jest ciągła w R bo nie jest ciągła w $\mathbb{R}$.

\bigskip

\section{Nieciągłości}
Mogą być dwa rodzaje nieciagłości
\begin{enumerate}
\item{Skok}

Mamy jakąś wartość dla x-ów do pewnego $x_0$, apotem zaczyna się coś innego.

\item{Przerwa}

Mamy definicje funkcji oraz inne cuda.


\end{enumerate}

\section{Zadanie - autor rozwiązania Anna}
$$
\begin{cases}
\frac{\sqrt{1 +x} - 1}{x} , x \not= 0 \\
t , x = 0\\
\end{cases}
$$

\section{Pochodne}
\subsection{1a - rozwiązanie by Rurek}
$$f(x) = 9x^7+ \frac{3}{2}x^2 - \frac{1}{3}x^3 + 3x^{-5} - 3 x^{-11} $$
$$f'(x) = 63x^6 + 6x^3 - x^2 - 15x^{-6} + 33x^{-12}$$
\subsection{1b - rozwiazanie by Marek}
$$f(x) = \frac{5x^2 + x - 2}{x^2+7} $$
$$f'(x) = \frac{(10x +1)(x^2 + 7) - (2x)(5x^2 + x -2)}{(x^2 + 7)^2} $$
Reszta to czysta arytmetyka
\subsection{1c - rozwiązanie by Skrzypu}
$$f(x) = \frac{x+1}{\sqrt{1-x}} $$
Wykorzystamy dwa wzory, na pochodną ilorazu oraz pochodną funkcji wewnętrzenej.
$$f(x) = \frac{g(x)}{h(x)}$$
$$g'(x) = 1$$
$$h'(x) = \frac{-1}{2\sqrt{-x+1}} $$
$$f'(x) = \frac{g'(x)h(x) - g(x)f'(x)}{(g(x))^2}$$
Reszta to arytmetyka ( nie chce mi się przepisywać )
$$f'(x) = \frac{\sqrt{1-x} + \frac{1}{2}(x+1)(-x+1)^{\frac{1}{2}}}{|x-1|}$$

\subsection{1d - rozwiązanie by Prowadząca}
$$f(x) = \tan^4{2x} $$
Mamy potrójnie złożoną funckję
$$f'(x) = 4\tan^3{2x} * \frac{1}{\cos^2{x}} 2 $$
$$f'(x) = 8\frac{\sin^3{2x}}{\cos^5{2x}}$$

\subsection{1e - rozwiązanie by Goral-niskopienny}
$$f(x) = x\arcsin{x} + \ln{1+x^2} $$
$$f'(x) = \arcsin{x} + \frac{x}{\sqrt{1-x^2}} + \frac{2x}{1+x^2} $$

\subsection{1f - rozwiązanie by Goral-niskopienny}
$$f(x) = \sin{(3 \cos{x})}\cos{ (5 \sin^3{7x})} $$
$$f'(x) = \cos{[(3\cos{x})(-3\sin{x})]} * \cos{[5 \sin^3{7x}]} + \sin{[5\sin^3{7x}]} - \sin{5\sin^3{7x}}15\sin^2{7x}*\cos{7x}*7 $$

\subsection{1g - rozwiązanie by Daniel}
$$f(x) = \ln{\sqrt{\frac{1-\sin{x}}{1-\cos{x}}}} $$
$$f'(x) = \frac{1}{\sqrt{\frac{1-\sin{x}}{1-\cos{x}}}} \frac{1}{2\sqrt{\frac{1-\sin{x}}{1-\cos{x}}}} * \frac{-\cos{x} *(1-\cos{x}) - (1-\sin{x})\sin{x}}{(1-\cos{x})^2}$$
$$f'(x) = \frac{1-\cos{x}}{2-2\sin{x}} * \frac{1-\cos{x}-\sin{x}}{(1-\cos{x})^2} = \frac{1}{2-2\sin{x}} * \frac{1-\cos{x}-\sin{x}}{(1-\cos{x})}$$
\end{document}