\documentclass[11pt]{article}
\usepackage[utf8]{inputenc}
%\usepackage[T1]{fontenc}
\usepackage{amssymb}
\usepackage{amsmath}
\usepackage{enumerate}
\usepackage{fullpage}
\usepackage{polski}  
\usepackage{indentfirst} 
\usepackage[pdftex]{graphicx}
\usepackage{multirow}
\usepackage{placeins}

\author{Łukasz Dubiel}
\title{Matematyka Ćwiczenia - zestaw 9\#1}

\begin{document}

\maketitle

\section{Dygresja}
$$ \int_0^2 \frac{dx}{x^2} = \frac{-1}{x} |_0^2 $$
I nie w ten sposób
$$ \lim_{a \to 0^+}{\int_a^2 \frac{1}{x^2}dx} = \lim_{a \to 0^+}{\frac{1}{x}|_a^2 = \infty} $$
Bo widać z prostego podstawienia
\section{Zadanie 1 - przykład 1}
$$ \int_1^3 \frac{dx}{\sqrt{(x-1)(3-x)}} = \lim_{\epsilon \to 0} \int_{1+\epsilon}^{3-\epsilon} \frac{dx}{\sqrt{-x^2 + 4x -3 }} = \lim_{\epsilon \to 0} \int_{1+\epsilon}^{3-\epsilon} \frac{dx}{\sqrt{-x^2 + 4x - 4 + 1 }} $$
$$ \lim_{\epsilon \to 0} \int_{1+\epsilon}^{3-\epsilon} \frac{dx}{\sqrt{1 - (x-2)^2}}  $$
Dygresja
Teraz policzmy sobie całkę nieoznaczoną po użyciu postawienia $ t = x -2 $
$$ \int \frac{dx}{\sqrt{1-t^2}} = \arcsin{t} = \arcsin{x-2} + C $$
Powrót
$$ \lim_{\epsilon \to 0} \int_{1+\epsilon}^{3-\epsilon} \frac{dx}{\sqrt{1 - (x-2)^2}}  = \lim_{\epsilon \to 0}{\arcsin{x-2}|_{1+\epsilon}^{3-\epsilon} } = \lim_{\epsilon \to 0}{\arcsin{1-\epsilon} - \arcsin{-1+\epsilon}}$$

\section{Zadanie 1 - przykład 3}
$$ \int_1^{\infty} \frac{dx}{\sqrt{x}} = \lim_{a \to \infty}{ \int_1^a \frac{dx}{\sqrt{x}} }$$

Używamy podstawienia $ x = t^2, \quad dx = 2tdt $
$$2 \int dt = 2t = 2\sqrt{x} $$
Wracając
$$ \lim_{a \to \infty}{ \frac{dx}{\sqrt{x} |_1^a}} = 
\lim_{a \to \infty}{ 2\sqrt{a} - 2 = \infty}$$

\section{Zadanie 1 - przykład 4}
$$ \int_{-\infty}^{-1} \frac{dx}{x\sqrt{x^2-1}} $$
Na obu granicach całkowania mamy problem. Dlatego trzeba to rozbić na fragmenty.
$$ \int_{-\infty}^{c} \frac{dx}{x\sqrt{x^2-1}} + \int_{c}^{-1} \frac{dx}{x\sqrt{x^2-1}} = \lim_{a \to -\infty}{\int_{a}^{c} \frac{dx}{x\sqrt{x^2-1}}} + \lim_{b \to -1^-}{\int_{c}^{b} \frac{dx}{x\sqrt{x^2-1}}}$$
Teraz liczymy sobie całkę nieoznaczoną
$$ \int \frac{dx}{x\sqrt{x^2-1}} $$
Wykorzystujemy I postawienie Euluera. 
$$ \sqrt{x^2-1} = t - x$$
$$ x^2 -1 = t^2 - 2x + x^2$$
$$ x = \frac{t^2-1}{2}$$
$$ dx = \frac{t^2-1}{2t^2} $$
$$ \sqrt{x^2-1} = t - \frac{t^2-1}{2} = \frac{-t^2 +2t -1}{2} $$
$$ \int \frac{dx}{x\sqrt{x^2-1}} = \int \frac{\frac{t^2-1}{t^2}dt}{\frac{t^2+1}{2t} \cdot \frac{t^2+1}{2t} } = 2 \int \frac{dt}{t^2+1} = \arctan{t} + c = \arctan{\sqrt{x^2-1}+x} $$
I teraz rozwiązujemy po części.




\end{document}