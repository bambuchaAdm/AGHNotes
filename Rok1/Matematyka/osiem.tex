\documentclass{article}
\usepackage[utf8]{inputenc}
%\usepackage[T1]{fontenc}
\usepackage{amssymb}
\usepackage{amsmath}
\usepackage{enumerate}
\usepackage{fullpage}
\usepackage{polski}  
\usepackage{indentfirst} 
\usepackage[pdftex]{graphicx}
\usepackage{multirow}
\usepackage{placeins} 

\title{Matematyka - ćwiczenia}
\author{Łukasz Dubiel}
\begin{document}
\maketitle

\section{Zbiór 4 - Zadanie 3}
Zbadaj ciągłość i różniczkowalność funkcji
\subsection{b}
\begin{displaymath}g(x) =
\begin{cases}
\frac{x(x-1)}{2} , x < 1 \\
\sqrt{x} - 1 , x \geq 1 \\
\end{cases}
\end{displaymath}
Funkcja dana jest nam przez części. Obie części są funckjami elementarnymi tak więc są ciągłe i różniczkowalne.
Problemem może okazać się punkt $1$. Tak trzeba policzyć granice obustronne w $1$.
$$\lim_{x \to 1^-}{\frac{x(x-1)}{2}} = 0 $$
$$\lim_{x \to 1^+}{\sqrt{x} - 1} = 0$$
Tak więc wiemy że istnieje granica tej funkcji w $1$
Pozostaje policzyć czy $$ g(1) = \lim_{x \to 1}{g(x)} $$
$$g(x) = 0 $$
Tak więc funkcja jest ciągła w całej swojej dziedzinie.

No to pora na sprawdzenie różniczkowalności w $x_0 = 1$, gdyż w funkcja jest elementarna więc jest różniczkowalna.
$$\lim_{x \to x_0}{\frac{g(x) - g(x_0)}{x - x_0}} $$
No nie widać od razu, więc trzeba przez części.
$$\lim_{x \to 1^-}{\frac{\frac{x(x-1)}{2}}{x-1}} = \frac{1}{2}$$
Teraz druga strona
$$\lim_{x \to 1^+}{\frac{\sqrt{x}-1}{x - 1}} = \lim_{ x \to 1^+}{\frac{1}{\sqrt{x}+1}} = \frac{1}{2} $$
$$g'(x) = \frac{1}{2}$$

\newpage
\section{Zestaw 4 - zadnie 4}
Wyznacz parametry $ p , q \in \mathbb{R} $ by funkcja była różniczkowalna i ciągła.
\subsection{a}
$$f(x) =
\begin{cases}
\sin{x}, x < 0 \\
px^2 + qx , x \geq 0\\
\end{cases}$$
$$\lim_{x \to 0^-}{\sin{x}} = 0$$
$$\lim_{x \to 0^+}{px^2 + qx} = \lim_{x \to 0^+}{x(px + q)} = 0 $$
Tak więc dla dowolnych $p$,$q$ granica w $0$ istnieje.Wartość $f(0) = 0 $ tak więc funkcja jest ciągła niezależnie od $p$,$q$.

Przejdźmy do różniczkowalności. Nie widać czy granica istnieje od razu, tak więc musimy przejść na strony.
$$\lim_{x \to 0^-}{\frac{f(x) - f(x_0)}{x-x_0}} = \lim_{x \to 0^-}{\frac{\sin{x}}{x}} = 1   $$
$$\lim_{x \to 0^+}{\frac{px^2 + qx - 0}{x-0}} = \lim_{x \to 0^+}{\frac{x(px + q)}{x}} = \lim_{x \to 0^+}{px+q} = q$$
Tak więc $q = 1$ by funckja mogła być różniczkowalna.

\subsection{b}
$$f(x) =
\begin{cases}
pe^x + q, x \leq 0 \\
2 - x , x > 0 \\
\end{cases}
$$
Badamy ciągłość funkcji.
$$\lim_{x \to 0^-}{pe^x+q} = p + q$$
$$\lim_{x \to 0^+}{2-x} = 2$$
Czyli $$ p + q = 2 $$

Oczywiście dobieramy się do różniczkowalności.
$$\lim_{x \to 0^-}{\frac{p(e^x-1)}{x}} = p$$
Wykorzystujemy granice elementarną ( patrz pięc rozdział 3 )
i teraz drugą granicę
$$\lim_{x \to 0^-}{\frac{2-x-(p+q)}{x}} = 1$$

\newpage
\section{Zestaw 5 - zadanie 1}
\subsection{a}
$$\lim_{x \to 0}{\frac{x-\sin{x}}{x-\tan{x}}}$$
Licząc granicę wprost uzyskujemy symbol nieoznaczony 
$$\lim_{x \to 0}{\frac{x-\sin{x}}{x-\tan{x}}} = \left[\frac{0}{0}\right]$$
Korzystając z reguły a'Hospitala, dostajemy.
$$\lim_{x \to 0}{\frac{1-\cos{x}}{1-\cos^2{x}}}$$
$$\lim_{x \to 0}{\frac{1-\cos{x}}{\frac{cos^2-1}{\cos^2}}}$$
$$\lim_{x \to 0}{\frac{(1-\cos{x})\cos^2{x}}{(\cos{x}-1)(\cos{x}+1)}} = \lim_{x \to 0}{\frac{-\cos^2{x}}{\cos{x}+1}} = \frac{1}{2} $$

\subsection{b}
$$\lim_{ x \to -\infty}{\frac{x^2}{e^{x^2}}} $$
Licząc wprost dostaniemy $[\frac{\infty}{\infty}]$ , więc możemy użyć reguły d'Hospitala.
$$\lim_{ x \to -\infty}{\frac{2x}{2x e^{x^2}}} = \lim_{ x \to -\infty}{\frac{1}{ e^{x^2}}} = 0$$

\subsection{c}
$$\lim_{x \to \infty}{x(e^{\frac{1}{x} - 1})}$$


\end{document}