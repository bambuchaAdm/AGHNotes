\documentclass[11pt]{article}
\usepackage[utf8]{inputenc}
\usepackage{polski}



\author{Łukasz Dubiel}
\title{Matematyka-ćwiczenia}
\begin{document}
\maketitle
\section{Metody}
\subsection{$\infty - \infty$}
$$f(x) - g(x) = \frac{\frac{1}{f(x)} - \frac{1}{g(x)}}{\frac{1}{f(x)g(x)}}$$
\subsection{$0 * \infty$}
$$f(x)g(x) = \frac{f(x)}{\frac{1}{g(x)}}$$
\subsection{$0^0, \infty^0, 1^{\infty}$}
$$f(x)^{g(x)} = e^{g(x) \ln{f(x)}} $$
oraz 
$$\lim_{x \to x_0}{e^{f(x)}} = e^{\lim_{x \to x_0}}{f(x)}$$

\section{Zbór 5 - Zadanie 1}
\subsection{d}
$$\lim_{x\to0}{\frac{e^x-1-x}{x(e^x-1)}} = \left[\frac{0}{0}\right]$$
Z reguły a'Hospitala
$$\lim_{x\to0}{\frac{e^x-1}{(e^x-1)+xe^x}} = \left[\frac{0}{0}\right]$$
ponownie reguła
$$\lim_{x\to0}{\frac{e^x}{e^x+e^x+xe^x}}=\frac{1}{2}$$
\subsection{c}
$$\lim_{x\to0}{\left[\ln{(1+x)}\right]^x}$$
Ta granica jest równoważna granicy
$$\lim_{x\to0}{e^{x\ln{(\ln{(1+x)})}}}$$
Korzystając z twierdzenia o przeniesieniu granicy do argumentu funkcji powyższa granica jest równa.
$$e^{\lim_{x\to0}{{x\ln{(\ln{(1+x)})}}}}$$
Tak więc wyznaczmy pomocniczą granicę
$$\lim_{x\to0}{{x\ln{(\ln{(1+x)})}}} = \left[0 * -\infty \right]$$
Tak więc przekształcamy do 
$$\lim_{x\to0}{\frac{\ln{(\ln{(1+x)})}}{\frac{1}{x}}} = \left[\frac{\infty}{\infty}\right]$$
I korzystamy z reguły a'Hospitala.
$$\lim_{x\to0}{\frac{\frac{1}{\ln{(1+x)}}\frac{1}{x}*1}{\frac{-1}{x^2}}}$$

\subsection{f}
$$\lim_{x\to1}{x^{\frac{1}{1-x}}} = 1^{\infty}$$
$$\lim_{x\to1}{e^{\frac{1}{1-x}\ln{x}}}$$
Co jest równoważne
$$e^{\lim_{x\to1}{{\frac{1}{1-x}\ln{x}}}}$$
Tak więc musimy policzyć granicę 
$$\lim_{x\to1}{{\frac{\ln{x}}{1-x}}} = \left[\frac{0}{0}\right]$$
Wykorzystując regułę a'Hospitala ta granica jest równa
$$\lim_{x\to1}{\frac{\frac{1}{x}}{x}} = 1$$

\subsection{g}
$$\lim_{x\to-\infty}{\left[(x-1)e^{\frac{1}{x-1}} - x\right]} = \left[\infty - \infty\right]$$
Zapisujemy równoważnie
$$\lim_{x\to-\infty}{\frac{\frac{1}{x} - \frac{1}{(x-1)e^{\frac{1}{x-1}}}}{\frac{1}{x(x-1)e^{\frac{1}{x-1}}}}}$$
$$\lim_{x\to-\infty}{\frac{\frac{(x-1)e^{\frac{1}{x-1}}-x}{x(x-1)e^{\frac{1}{x-1}}}}{\frac{1}{x(x-1)e^{\frac{1}{x-1}}}}}$$
No to chyba nie tędy droga... Trzeba podejść inaczej.
$$\lim_{x\to-\infty}{\left[xe^{\frac{1}{x-1}} - e^{\frac{1}{x-1}} - x\right]}$$
$$-1+\lim_{x\to-\infty}{\left[x(e^{\frac{1}{x-1}} - 1)\right]}$$
Ale w granicy dostajemy symbol nieoznaczony $[-\infty*0]$
Dlatego zajmiemy się tylko tą granicą.
$$\lim_{x\to-\infty}{\frac{e^{\frac{1}{x-1}} - 1}{\frac{1}{x}{}}}$$
I mamy symbol nieoznaczony $\frac{0}{0}$, więc stosujemy regułę a'Hospitala.
$$\lim_{x \to -\infty}{\frac{e^{\frac{1}{x-1}}*\frac{-1}{(x-1)^2}}{\frac{-1}{x^2}}} = \lim_{x\to-\infty}{\frac{e^{\frac{1}{x-1}}x^2}{(x-1)^2}}$$
I mamy $\infty * \infty$
$$\lim_{x\to-\infty}{\frac{e^{\frac{1}{x-1}}\left[\frac{-x^2}{(x-1)^2} + 2x \right]}{2(x-1)}}$$
Po sprowadzeniu do wspólnego mianowniku
$$\lim_{x\to-\infty}{e^{\frac{1}{x-1}} \frac{2x^3 - 5x^2 + 2x}{2x^3-5x^2+6x-2}} = 1$$

\newpage
Drugi sposób, polega na sprytnym przekształcenia. 
Wyciągamy x przed nawias.
$$\lim_{x\to -\infty}{x\left[(1-\frac{1}{x})e^{\frac{1}{1-x}} -1 \right]}$$
I jazda d'Hospitalem.

\subsection{h}
$$\lim_{x\to0}{(1+2x)^{\frac{1}{x}}} = \lim_{x\to0}{e^{\frac{\ln{(1+2x)}}{x}}}$$


\end{document}
