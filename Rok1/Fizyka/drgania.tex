\documentclass[11pt]{article}
\usepackage[utf8]{inputenc}
%\usepackage[T1]{fontenc}
\usepackage{amssymb}
\usepackage{amsmath}
\usepackage{enumerate}
\usepackage{fullpage}
\usepackage{polski}  
\usepackage{indentfirst} 
\usepackage[pdftex]{graphicx}
\usepackage{multirow}
\usepackage{placeins}

\author{Łukasz Dubiel}
\title{Drgania }

\begin{document}
\maketitle
\section{Drgania elektryczne - RLC}
\subsection{Opór krytyczny}
Drganie periodyczne przejdzie w drganie aperiodyczne tylko i wyłącznie gdy
$$ \frac{1}{\sqrt{LC}} \geq \frac{R}{2L} $$
Przez co dosdtajemy
$$ \beta_k = \frac{1}{CL} $$

\subsection{Drgania wymuszone}
Działanie na układ $$F_z = F_0 \cos{\Omega t}$$
Równie przejdzie w postaci
$$ \frac{d}{d^2x} + 2\beta\frac{d}{dx} + \omega_0^2 = y \cos{\Omega t}$$

$$ A(\Omega) = \frac{y}{\sqrt{(\omega_0^2 - \Omega^2)^2 - 4 \beta^2 \Omega^2}} $$

\subsubsection{Brak oporów ( $ \beta = 0 $ )}

$$ A(\Omega) = \frac{y}{\sqrt{(\omega_0^2 \Omega^2)^2}} $$

\subsubsection{Opory $ \beta \not = 0$}
$$ \Omega_r = \sqrt{\omega_0^2 - 2\beta^2} $$
Gdy tłumienie rośnie, częstotliwość rezonansowa spada.

\end{document}
