\documentclass[11pt]{article}
\usepackage[utf8]{inputenc}
%\usepackage[T1]{fontenc}
\usepackage{amssymb}
\usepackage{amsmath}
\usepackage{enumerate}
\usepackage{fullpage}
\usepackage{polski}  
\usepackage{indentfirst} 
\usepackage[pdftex]{graphicx}
\usepackage{multirow}
\usepackage{placeins}

\author{Łukasz Dubiel}
\title{Fizyka ćwiczenia - moment bezwładności}
\begin{document}
\maketitle

\section{Zadanie 1 - rozwiązanie dynamiczne}
Oznaczenia\\
$m_1$ - masa pierwszeg \\
$m_2$ - masa drugiego \\
$Q_1$ - ciężar pierwszego \\
$Q_2$ - ciężar drugiego \\
$N_1$ - siła reakcji liny\\
$N_2$ - siła reakcji liny\\
$N_{1s}$ - siła reakcji styczna \\
$N_{2s}$ - siła reakcji styczna \\
$m_1$ = 1kg\\
$m_2$ = 0.8kg\\
$m$ = 0.2kg''

$$ F_{w1} = Q_1 - N_1$$
$$ F_{w2} = N_2 - Q_2$$
$$\begin{cases}m_1a = Q_1 - N_1\\
m_2a = N_1 - Q_1\end{cases}$$
Teraz uwzględniamy obrót koła
$$\epsilon = \frac{a}{R}$$
jak również
$$\epsilon = \frac{M}{I}$$
Uwzględniając fakt
$$ M = r(N_1 - N_2) $$
Łącząc trzy poprzednie równania
$$ \frac{r(N_1 - N_2)}{\frac{1}{2}mr^2} = \frac{a}{r}$$
Upraszczając dostajemy
$$ N_1 - N_2 = \frac{1}{2}Ma $$
Łącząc te trzy równiania dostajemy układ
$$\begin{cases}m_1a = Q_1 - N_1 \label{jeden}\\
m_2a = N_1 - Q_1 \label{dwa}\\
N_1 - N_2 = \frac{1}{2}Ma \label{trzy}\end{cases}$$
Rozwiązać układ równań

\section{Zadanie 1 - rozwiązanie przy pomocy ZZE}
Gdy układ się przesunieto
$$ E_p = M_1gh - M_2gh = \frac{M_1v^2}{2} +\frac{M_2v^2}{2} + \frac{I\omega^2}{2}$$
Uwzględniając że 
$$ \omega = \frac{v}{r}$$
dostajemy
$$ gh(M_1 - M_2) = \frac{M_1v^2}{2} +\frac{M_2v^2}{2} + \frac{Iv^2}{2r^2}$$
$$ gh(M_2 - M_2) = \frac{M_1v^2}{2} +\frac{M_2v^2}{2} + \frac{\frac{1}{2}mr^2v^2}{2r^2}$$
$$ gh(M_2 - M_2) = \frac{M_1v^2}{2} +\frac{M_2v^2}{2} + \frac{mv^2}{4}$$
$$ gh(M_2 - M_2) = \frac{v^2}{2}\left(M_1 + M_2 + \frac{m}{2}\right)$$
Uwzględniając 
$$ v = at $$
$$ h = \frac{at^2}{2} $$
$$ \frac{at^2}{2}g(M_1 - M_2) = \frac{1}{2}a^2 t^2 \left( M_1 + M_2 + \frac{m}{2}\right)$$
Wyznając a z tego wzoru
$$ a = \frac{g(M_1 - M_2)}{M_1 + M_2 + m} $$
i po postawieniu dostajemy
$$ a = 0,105g $$

\newpage
\section{Zadanie 3}
$$ \vec{\tau} = -D\vec{\varphi} \label{sila}$$
oraz
$$  \vec{\epsilon} = \frac{\tau}{I} \label{przyspieszenie}$$
$$ \vec{\epsilon} = \frac{d^2\varphi}{dt^2} \label{pochodna}$$
I teraz podstawiając \eqref{pochodna} do \eqref{przyspieszenie} oraz \eqref{sila} do \eqref{przyspieszenie} dostajemy
$$ \frac{d^2\varphi}{dt^2} + \frac{D\vec{\varphi}}{I} = 0$$
Rozwiązując to zwyczajne równanie różniczkowe
$$ \varphi = A\cos{k t} $$
$$ \frac{d^2(A\cos{k t})}{dt^2} + \frac{DA\cos{k t}}{I} = 0$$
$$ -Ak^2\cos{kt} + \frac{DA\cos{kt}}{I} = 0 $$
Dzielimy przez $A\cos{kt}$ dostajemy
$$ -k^2 + \frac{D}{I} = 0$$
$$ k = \pm\sqrt{\frac{D}{I}} $$
Podstawiając 
$$ \varphi = \varphi_0 \cos{\sqrt{\frac{D}{I}} t} $$
Czyli nasz okres wynosi
$$ T =\frac{2\pi}{\sqrt{\frac{D}{I}}} $$
$$ T = 2\pi\sqrt{\frac{I}{D}} $$
$D$ jest to moment siły , gdyż $\phi$ jest bezwymiarowe
Czyli $D ~ r$, gdzie r jest miejscem przyczepu sprężyny

\newpage
\section{Zadanie 4}
Praca to
$$ W = \int F \circ dx $$
$$ W = \int_0^X -kx \cos{0} dx $$
$$ W = - k \frac{x^2}{2}|_0^X = -k\frac{X^2}{2} $$
$$ x(t) = x_0\cos{\sqrt{\frac{k}{m}} t} $$
$$ E_p(t) = -k\frac{x_0 \cos{\sqrt{\frac{m}{k}}t}}{2} $$

Teraz energię kinetyczną
$$ E_k = m\frac{v^2}{2} $$

Pokazać że E

Zadanie - energia potencjalna na x

\end{document}