\documentclass[16pt]{article}
\usepackage[utf8]{inputenc}
\usepackage{polski}
\usepackage{enumerate}

\author{Łukasz Dubiel}
\title{Plan rozwoju DnDP}
\begin{document}
\maketitle
\newpage
\section{Wstęp}

Obecnie DnD Project ( w skrucie DnDP, chodź nazwa zostanie zmieniona)
jest systemem wspomagającym graczy podczas sesji.

Ma za zadanie przeliczać premie i współczyniki, pozwalając graczom weść głębiej w rozgrywkę.

Docelowo ma się składać z trzech elementów
\begin{enumerate}
\item{DnD Client}

Aplikacja dla gracza w świecie. Właśnie tutaj gracz kontroluje swoją postać oraz może odczytywać wszyskie informacje o grze udostępnione przez mistrza gry.

\item{DnD Game Master}

Aplikacja przeznaczona dla mistrza gry z której kontroluje karty graczy. To właśnie tutaj nakładane są premie i kary oraz wszelakie modyfikatory zależne od karprys szczęścia i mistrza gry. 

W dalekiej przyszłości miałby symulować części rozgrywki

\item{DnD Server}

Miesce przechowywania przedmiotów, postaci, stanów gry oraz innych mechanizmów. Podczas szykania przez graczy mistrza gry ma pełnić rolę drogowskazu, oraz pewnego rodzaju VPN'a w razie gdyby oboje byli za NATem.
\end{enumerate}



\section{Rozwój}

Obecnie DnDP wymaga uporządkowania, poprzez oczyszczenie z resztek starych rozwiązań oraz użycie bibliotek i narzędzi dedykowanych do danych czyności

Do końca stycznia sądzę, że jestem wstanie wykonać następujące zmiany.
\begin{enumerate}
\item{Przepisanie testów}

Użycie mocków w testach umożliwi oddzielenie testów jednostkowych od całości środowiska
(obecnie na potrzeby testu jest tworzone całe potrzebne środowisko) rozluźni to wiązanie między testami a "testowanymi". 

Jest to jedna z rzeczy które są śmieciami po poprzedniej, złej koncepcji.


Planowane narzędzia/biblioteki : JUnit (już jest), EasyMock lub Mockito lub Powermock

Szacowany czas : max 6 godzin przy wszystkich testach

\item{Dokończenie modułów}

System ma się podzielić na moduły. Większa część jest już napisana, jednak brakuje wykończenia. Alternatywą jest przejśćie na którąś implementację OSGI.

Szacowany czas: max 6 godzin

\item{Przpisanie GUI}

Planowane narzędzia/biblioteki: Apache Pivot (całkowita nowość) lub Swing

Szacowany czas : max 20 godzin programowania, 2 godziny GIMPa

\item{Stwirzenie GUI dla przedmiotów}

Obcnie wirtualna karta postaci nie jest pełna. Zawiera graficzny interfejs do większości współczyników lecz nie ekwipunku. Powinno się to zmienić w następnym wydaniu.

Planowane narzędzia/biblioteki: patrz pkt wyżej

Szacowany czas : 10 godzin programowania

\item{Rozpoczęcie używania DI}

Pozwoli na rozluźnienie wiązania pomiędzy poszczególnymi modułami. Pozwoli na prawie transparentną zmiennę wewnętrznych implementacji ( o ile nie zmienią API )

Planowane narzędzia/biblioteki : Google Guice

Szacowany czas: max 4 godziny kodowania ( nie liczę konsultacji z innymi developerami oraz eksperymentów)

\item{Dodanie listy znanych języków dla postaci}

W fantastycznych krainach instnieje wiele rówżnych ras jak również języków. System powinien również uwzględniać ten fakt, a obecnie tego nie robi.

Jest to ticket który dotyczy bezpośrednio rozszerzenia funkcjonalnośći. W obszarze projektu jest ich kilka, każdy o równie niskiej wadze. Wpisuje ten jako ich reprezentanta.

Szacowany czas: max 2 godziny

\end{enumerate}

\section{Wnioski}
Planowanej pracy jest sporo. Lista nie jest finalna, a wałście jest przepisaniem cześci wiszących ticketów w projekcie. Uważam, że jestem wstanie wykonać cały przedstawiony plan do końca stycznia. Jeżeli całość zostałaby wykonana oznaczałoby to dość duży skok technologiczny porównywalny z przejściem z ant'a na mavena.




\end{document}